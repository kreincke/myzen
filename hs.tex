% hannya shingyo cloak file
%
% (c) Karsten Reincke, Frankfurt a.M. 2016, ff.
%
% This file is licensed under the Creative Commons Attribution 3.0 Germany
% License (http://creativecommons.org/licenses/by/3.0/de/): Feel free to share
% (to copy, distribute and transmit) or to remix (to adapt) it, if you respect
% how you must attribute the work in the manner specified by the author(s):
%
% In an internet based reuse please link the reused parts to 
% http://www.fodina.de/myzen/ and mention the original author Karsten Reincke in 
% a suitable manner.
% 
% In a paper-like reuse please insert into your preface etc. a short hint to
% http://www.fodina.de/myzen/ and to the original author, Karsten Reincke, .
% 
% For normal quotations please use the scientific standard to cite.

\documentclass[
DIV=calc,
BCOR=5mm,
11pt,
headings=small,
oneside,
bibtotocnumbered,
english,ngerman]{scrartcl}

\usepackage[utf8]{inputenc}

\usepackage[]{a4,babel}
\selectlanguage{ngerman}

\usepackage{csquotes}

\usepackage[see]{jurabib}
\bibliographystyle{jurabib}
% mycsrf German jurabib configuration include module file 
%
% (c) Karsten Reincke, Frankfurt a.M. 2012, ff.
%
% This text is licensed under the Creative Commons Attribution 3.0 Germany
% License (http://creativecommons.org/licenses/by/3.0/de/): Feel free to share
% (to copy, distribute and transmit) or to remix (to adapt) it, if you respect
% how you must attribute the work in the manner specified by the author(s):
% \newline
% In an internet based reuse please link the reused parts to mycsrf.fodina.de
% and mention the original author Karsten Reincke in a suitable manner. In a
% paper-like reuse please insert a short hint to mycsrf.fodina.de and to the
% original author, Karsten Reincke, into your preface. For normal quotations
% please use the scientific standard to cite.

% the first time cite with all data, later with shorttitle
\jurabibsetup{citefull=first}

%%% (1) author / editor list configuration
%\jurabibsetup{authorformat=and} % uses 'und' instead of 'u.'
% therefore define your own abbreviated conjunction: 
% an 'and before last author explicetly written conjunction

% for authors in citations
\renewcommand*{\jbbtasep}{\ u.\ } % bta = between two authors sep
\renewcommand*{\jbbfsasep}{,\ } % bfsa = between first and second author sep
\renewcommand*{\jbbstasep}{\ u.\ }% bsta = between second and third author sep
% for editors in citations
\renewcommand*{\jbbtesep}{\ u.\ } % bta = between two authors sep
\renewcommand*{\jbbfsesep}{,\ } % bfsa = between first and second author sep
\renewcommand*{\jbbstesep}{\ u.\ }% bsta = between second and third author sep

% for authors in literature list
\renewcommand*{\bibbtasep}{\ u.\ } % bta = between two authors sep
\renewcommand*{\bibbfsasep}{,\ } % bfsa = between first and second author sep
\renewcommand*{\bibbstasep}{\ u.\ }% bsta = between second and third author sep
% for editors  in literature list
\renewcommand*{\bibbtesep}{\ u.\ } % bte = between two editors sep
\renewcommand*{\bibbfsesep}{,\ } % bfse = between first and second editor sep
\renewcommand*{\bibbstesep}{\ u.\ }% bste = between second and third editor sep

% use: name, forname, forname lastname u. forname lastname
\jurabibsetup{authorformat=firstnotreversed}
\jurabibsetup{authorformat=italic}

%%% (2) title configuration
% in every case print the title, let it be seperated from the 
% author by a colon and use the slanted font
\jurabibsetup{titleformat={all,colonsep}}
%\renewcommand*{\jbtitlefont}{\textit}

%%% (3) seperators in bib data
% separate bibliographical hints and page hints by a comma
\jurabibsetup{commabeforerest}

%%% (4) specific configuration of bibdata in quotes / footnote
% use a.a.O if possible
\jurabibsetup{ibidem=strict}
% replace ugly a.a.O. by ders., a.a.O. resp. ders., ebda.
% but if there are more than one author or girl writers?
\AddTo\bibsgerman{
  \renewcommand*{\ibidemname}{Ds.,\ a.a.O.}
  \renewcommand*{\ibidemmidname}{ds.,\ a.a.O.}
}
\renewcommand*{\samepageibidemname}{Ds.,\ ebda.}
\renewcommand*{\samepageibidemmidname}{ds.,\ ebda.}

%%% (5) specific configuration of bibdata in bibliography
% ever an in: before journal and collection/book-titles 

\renewcommand*{\bibjtsep}{in:\ }
\renewcommand*{\bibbtsep}{in:\ }

% ever a colon after author names 
\renewcommand*{\bibansep}{:\ }
% ever a semi colon after the title 
\renewcommand*{\bibatsep}{;\ }
% ever a comma before date/year
\renewcommand*{\bibbdsep}{,\ }

% let jurabib insert the S. and p. information
% no S. necessary in bib-files and in cites/footcites
\jurabibsetup{pages=format}

% use a compressed literature-list using a small line indent
\jurabibsetup{bibformat=compress}
\setlength{\jbbibhang}{1em}

% which follows the design of the cites and offers comments
\jurabibsetup{biblikecite}

% print annotations into bibliography
\jurabibsetup{annote}
\renewcommand*{\jbannoteformat}[1]{{ \itshape #1 }}

\renewcommand*{\biburlprefix}{ ( $\rightarrow$ }
\renewcommand*{\biburlsuffix}{ )}

%refine the prefix of url download
\AddTo\bibsgerman{\renewcommand*{\urldatecomment}{Referenzdownload: }}

% we want to have the year of articles in brackets
\renewcommand*{\bibaldelim}{(}
\renewcommand*{\bibardelim}{)}

%Umformatierung des Reihentitels und der Reihennummer
\DeclareRobustCommand{\numberandseries}[2]{%
\unskip\unskip%,
\space\bibsnfont{(=~#2}%
\ifthenelse{\equal{#1}{}}{)}{, [Bd./Nr.]~#1)}%
}%

%Umformatierung Referenzverweises
\usepackage{xpatch}
\AfterFile{dejbbib.ldf}{%
  \xapptocmd{\bibsgerman}{%
     \def\inname{\ifjboxford in:\else\ifjbchicago in:\else in:\fi\fi}%
    \def\incollinname{\ifjboxford in:\else\ifjbchicago in:\else in:\fi\fi}%
  }{}{}%
}



%zenk Hyphenation Include Module text
%
% (c) Karsten Reincke, Frankfurt a.M. 2012, ff.
%
% This text is licensed under the Creative Commons Attribution 3.0 Germany
% License (http://creativecommons.org/licenses/by/3.0/de/): Feel free to share
% (to copy, distribute and transmit) or to remix (to adapt) it, if you respect
% how you must attribute the work in the manner specified by the author(s):
% \newline
% In an internet based reuse please link the reused parts to zen.fodina.de
% and mention the original author Karsten Reincke in a suitable manner. In a
% paper-like reuse please insert a short hint to zen.fodina.de and to the
% original author, Karsten Reincke, into your preface. For normal quotations
% please use the scientific standard to cite.
%


\hyphenation{ my-keds there-fo-re}




% package for improving the grey value and the line feed handling
\usepackage{microtype}

%%% (3) layout page configuration %%%

% select the visible parts of a page
% S.31: { plain|empty|headings|myheadings }
\pagestyle{plain}
%\pagestyle{empty}

% select the wished style of page-numbering
% S.32: { arabic,roman,Roman,alph,Alph }
\pagenumbering{arabic}
\setcounter{page}{1}

% select the wished distances using the general setlength order:
% S.34 { baselineskip| parskip | parindent }
% - general no indent for paragraphs
\setlength{\parindent}{0pt}
\setlength{\parskip}{1.2ex plus 0.2ex minus 0.2ex}


%%% (4) general package activation %%%

%- start(footnote-configuration)

\deffootnote[1.5em]{1.5em}{1.5em}{\textsuperscript{\thefootnotemark)\ }}

% package for macking tables with broken lines
\usepackage{multirow}

%for using label as nameref
\usepackage{nameref}

%integrate nomenclature
% zen  Deutsch Nomenclation Declaration Include Module 
%
% (c) Karsten Reincke, Frankfurt a.M. 2012, ff.
%
% This text is licensed under the Creative Commons Attribution 3.0 Germany
% License (http://creativecommons.org/licenses/by/3.0/de/): Feel free to share
% (to copy, distribute and transmit) or to remix (to adapt) it, if you respect
% how you must attribute the work in the manner specified by the author(s):
% \newline
% In an internet based reuse please link the reused parts to zen.fodina.de
% and mention the original author Karsten Reincke in a suitable manner. In a
% paper-like reuse please insert a short hint to zen.fodina.de and to the
% original author, Karsten Reincke, into your preface. For normal quotations
% please use the scientific standard to cite.

\usepackage[intoc]{nomencl}
\let\abbr\nomenclature
% Deutsche Überschrift
%\renewcommand{\nomname}{Periodicals, Shortcuts, and Overlapping Abbreviations}
\renewcommand{\nomname}{Periodika, ihre Kurzformen und generelle Abkürzungen}

\setlength{\nomlabelwidth}{.20\hsize}
\renewcommand{\nomlabel}[1]{#1 \dotfill}
% reduce the line distance
\setlength{\nomitemsep}{-\parsep}
\makenomenclature


% Hyperlinks
\usepackage{hyperref}
\hypersetup{bookmarks=true,breaklinks=true,colorlinks=true,citecolor=blue,draft=false}

\usepackage[encapsulated]{CJK}
\newcommand{\cntext}[1]{\begin{CJK}{UTF8}{gbsn}#1\end{CJK}}

\newcommand{\cnbkai}[1]{\begin{CJK}{UTF8}{bkai}#1\end{CJK}}
\newcommand{\cnbsmi}[1]{\begin{CJK}{UTF8}{bsmi}#1\end{CJK}}
\newcommand{\cngbsn}[1]{\begin{CJK}{UTF8}{gbsn}#1\end{CJK}}
\newcommand{\jpwada}[1]{\begin{CJK}{UTF8}{wadalab}#1\end{CJK}}
\newcommand{\jpsong}[1]{\begin{CJK}{UTF8}{song}#1\end{CJK}}
\newcommand{\jpbkai}[1]{\begin{CJK}{UTF8}{bkai}#1\end{CJK}}


\usepackage{arydshln}

\usepackage{geometry}


\begin{document}

\newgeometry{left=1.8cm,right=1.8cm,top=1cm,bottom=2.5cm}

%% use all entries of the bliography
\nocite{*}

%%-- start(titlepage)
\titlehead{Mit Dank an meine Frau für ihren wunderbaren Mut zu neuen guten
Wegen:}
%\subject{ZEN}
\title{Hannya Shingyō Lerntext}
\author{Karsten Reincke% zen License Include Module
%
% (c) Karsten Reincke, Frankfurt a.M. 2012, ff.
%
% All files of myzen are licensed under the Creative Commons Attribution 3.0
% Germany License (http://creativecommons.org/licenses/by/3.0/de/): Feel free to 
% share (to copy, distribute and transmit) or to remix (to adapt) it, if you 
% respect how you must attribute the work in the manner specified by the author:
% 
% In an internet based reuse please link the reused parts to 
% http://www.fodina.de/myzen/ and mention the original author Karsten Reincke in 
% a suitable manner.
% 
% In a paper-like reuse please insert into your preface etc. a short hint to 
% http://www.fodina.de/myzen/ and to the original author, Karsten Reincke, .
% 
% For normal quotations please use the scientific standard to cite.

\footnote{This text is licensed under the Creative Commons Attribution 3.0
License (http://creativecommons.org/licenses/by/3.0/): Feel free \enquote{to
share (to copy, distribute and transmit)} or \enquote{to remix (to adapt)}. As
a compensation, \enquote{you must attribute (your modified) work in the manner
specified by the author(s) [\ldots]}): In each reuse, mention the original
author -- Karsten Reincke -- and insert a link/hint to
\texttt{http://www.fodina.de/myzen/} }


 
 - Release % hannya shingyo cloak file
%
% (c) Karsten Reincke, Frankfurt a.M. 2016, ff.
%
% This file is licensed under the Creative Commons Attribution 3.0 Germany
% License (http://creativecommons.org/licenses/by/3.0/de/): Feel free to share
% (to copy, distribute and transmit) or to remix (to adapt) it, if you respect
% how you must attribute the work in the manner specified by the author(s):
%
% In an internet based reuse please link the reused parts to 
% http://www.fodina.de/myzen/ and mention the original author Karsten Reincke in 
% a suitable manner.
% 
% In a paper-like reuse please insert into your preface etc. a short hint to
% http://www.fodina.de/myzen/ and to the original author, Karsten Reincke, .
% 
% For normal quotations please use the scientific standard to cite.

\documentclass[
DIV=calc,
BCOR=5mm,
11pt,
headings=small,
oneside,
bibtotocnumbered]{scrartcl}

\usepackage[utf8]{inputenc}

\usepackage[]{a4,ngerman}
\usepackage[english,ngerman]{babel}
\selectlanguage{ngerman}

\usepackage{csquotes}

\usepackage[see]{jurabib}
\bibliographystyle{jurabib}
% mycsrf German jurabib configuration include module file 
%
% (c) Karsten Reincke, Frankfurt a.M. 2012, ff.
%
% This text is licensed under the Creative Commons Attribution 3.0 Germany
% License (http://creativecommons.org/licenses/by/3.0/de/): Feel free to share
% (to copy, distribute and transmit) or to remix (to adapt) it, if you respect
% how you must attribute the work in the manner specified by the author(s):
% \newline
% In an internet based reuse please link the reused parts to mycsrf.fodina.de
% and mention the original author Karsten Reincke in a suitable manner. In a
% paper-like reuse please insert a short hint to mycsrf.fodina.de and to the
% original author, Karsten Reincke, into your preface. For normal quotations
% please use the scientific standard to cite.

% the first time cite with all data, later with shorttitle
\jurabibsetup{citefull=first}

%%% (1) author / editor list configuration
%\jurabibsetup{authorformat=and} % uses 'und' instead of 'u.'
% therefore define your own abbreviated conjunction: 
% an 'and before last author explicetly written conjunction

% for authors in citations
\renewcommand*{\jbbtasep}{\ u.\ } % bta = between two authors sep
\renewcommand*{\jbbfsasep}{,\ } % bfsa = between first and second author sep
\renewcommand*{\jbbstasep}{\ u.\ }% bsta = between second and third author sep
% for editors in citations
\renewcommand*{\jbbtesep}{\ u.\ } % bta = between two authors sep
\renewcommand*{\jbbfsesep}{,\ } % bfsa = between first and second author sep
\renewcommand*{\jbbstesep}{\ u.\ }% bsta = between second and third author sep

% for authors in literature list
\renewcommand*{\bibbtasep}{\ u.\ } % bta = between two authors sep
\renewcommand*{\bibbfsasep}{,\ } % bfsa = between first and second author sep
\renewcommand*{\bibbstasep}{\ u.\ }% bsta = between second and third author sep
% for editors  in literature list
\renewcommand*{\bibbtesep}{\ u.\ } % bte = between two editors sep
\renewcommand*{\bibbfsesep}{,\ } % bfse = between first and second editor sep
\renewcommand*{\bibbstesep}{\ u.\ }% bste = between second and third editor sep

% use: name, forname, forname lastname u. forname lastname
\jurabibsetup{authorformat=firstnotreversed}
\jurabibsetup{authorformat=italic}

%%% (2) title configuration
% in every case print the title, let it be seperated from the 
% author by a colon and use the slanted font
\jurabibsetup{titleformat={all,colonsep}}
%\renewcommand*{\jbtitlefont}{\textit}

%%% (3) seperators in bib data
% separate bibliographical hints and page hints by a comma
\jurabibsetup{commabeforerest}

%%% (4) specific configuration of bibdata in quotes / footnote
% use a.a.O if possible
\jurabibsetup{ibidem=strict}
% replace ugly a.a.O. by ders., a.a.O. resp. ders., ebda.
% but if there are more than one author or girl writers?
\AddTo\bibsgerman{
  \renewcommand*{\ibidemname}{Ds.,\ a.a.O.}
  \renewcommand*{\ibidemmidname}{ds.,\ a.a.O.}
}
\renewcommand*{\samepageibidemname}{Ds.,\ ebda.}
\renewcommand*{\samepageibidemmidname}{ds.,\ ebda.}

%%% (5) specific configuration of bibdata in bibliography
% ever an in: before journal and collection/book-titles 

\renewcommand*{\bibjtsep}{in:\ }
\renewcommand*{\bibbtsep}{in:\ }

% ever a colon after author names 
\renewcommand*{\bibansep}{:\ }
% ever a semi colon after the title 
\renewcommand*{\bibatsep}{;\ }
% ever a comma before date/year
\renewcommand*{\bibbdsep}{,\ }

% let jurabib insert the S. and p. information
% no S. necessary in bib-files and in cites/footcites
\jurabibsetup{pages=format}

% use a compressed literature-list using a small line indent
\jurabibsetup{bibformat=compress}
\setlength{\jbbibhang}{1em}

% which follows the design of the cites and offers comments
\jurabibsetup{biblikecite}

% print annotations into bibliography
\jurabibsetup{annote}
\renewcommand*{\jbannoteformat}[1]{{ \itshape #1 }}

\renewcommand*{\biburlprefix}{ ( $\rightarrow$ }
\renewcommand*{\biburlsuffix}{ )}

%refine the prefix of url download
\AddTo\bibsgerman{\renewcommand*{\urldatecomment}{Referenzdownload: }}

% we want to have the year of articles in brackets
\renewcommand*{\bibaldelim}{(}
\renewcommand*{\bibardelim}{)}

%Umformatierung des Reihentitels und der Reihennummer
\DeclareRobustCommand{\numberandseries}[2]{%
\unskip\unskip%,
\space\bibsnfont{(=~#2}%
\ifthenelse{\equal{#1}{}}{)}{, [Bd./Nr.]~#1)}%
}%

%Umformatierung Referenzverweises
\usepackage{xpatch}
\AfterFile{dejbbib.ldf}{%
  \xapptocmd{\bibsgerman}{%
     \def\inname{\ifjboxford in:\else\ifjbchicago in:\else in:\fi\fi}%
    \def\incollinname{\ifjboxford in:\else\ifjbchicago in:\else in:\fi\fi}%
  }{}{}%
}



%zenk Hyphenation Include Module text
%
% (c) Karsten Reincke, Frankfurt a.M. 2012, ff.
%
% This text is licensed under the Creative Commons Attribution 3.0 Germany
% License (http://creativecommons.org/licenses/by/3.0/de/): Feel free to share
% (to copy, distribute and transmit) or to remix (to adapt) it, if you respect
% how you must attribute the work in the manner specified by the author(s):
% \newline
% In an internet based reuse please link the reused parts to zen.fodina.de
% and mention the original author Karsten Reincke in a suitable manner. In a
% paper-like reuse please insert a short hint to zen.fodina.de and to the
% original author, Karsten Reincke, into your preface. For normal quotations
% please use the scientific standard to cite.
%


\hyphenation{ my-keds there-fo-re}




% package for improving the grey value and the line feed handling
\usepackage{microtype}

%%% (3) layout page configuration %%%

% select the visible parts of a page
% S.31: { plain|empty|headings|myheadings }
\pagestyle{plain}
%\pagestyle{empty}

% select the wished style of page-numbering
% S.32: { arabic,roman,Roman,alph,Alph }
\pagenumbering{arabic}
\setcounter{page}{1}

% select the wished distances using the general setlength order:
% S.34 { baselineskip| parskip | parindent }
% - general no indent for paragraphs
\setlength{\parindent}{0pt}
\setlength{\parskip}{1.2ex plus 0.2ex minus 0.2ex}


%%% (4) general package activation %%%

%- start(footnote-configuration)

\deffootnote[1.5em]{1.5em}{1.5em}{\textsuperscript{\thefootnotemark)\ }}

% package for macking tables with broken lines
\usepackage{multirow}

%for using label as nameref
\usepackage{nameref}

%integrate nomenclature
% zen  Deutsch Nomenclation Declaration Include Module 
%
% (c) Karsten Reincke, Frankfurt a.M. 2012, ff.
%
% This text is licensed under the Creative Commons Attribution 3.0 Germany
% License (http://creativecommons.org/licenses/by/3.0/de/): Feel free to share
% (to copy, distribute and transmit) or to remix (to adapt) it, if you respect
% how you must attribute the work in the manner specified by the author(s):
% \newline
% In an internet based reuse please link the reused parts to zen.fodina.de
% and mention the original author Karsten Reincke in a suitable manner. In a
% paper-like reuse please insert a short hint to zen.fodina.de and to the
% original author, Karsten Reincke, into your preface. For normal quotations
% please use the scientific standard to cite.

\usepackage[intoc]{nomencl}
\let\abbr\nomenclature
% Deutsche Überschrift
%\renewcommand{\nomname}{Periodicals, Shortcuts, and Overlapping Abbreviations}
\renewcommand{\nomname}{Periodika, ihre Kurzformen und generelle Abkürzungen}

\setlength{\nomlabelwidth}{.20\hsize}
\renewcommand{\nomlabel}[1]{#1 \dotfill}
% reduce the line distance
\setlength{\nomitemsep}{-\parsep}
\makenomenclature


% Hyperlinks
\usepackage{hyperref}
\hypersetup{bookmarks=true,breaklinks=true,colorlinks=true,citecolor=blue,draft=false}

\usepackage[encapsulated]{CJK}
\newcommand{\cntext}[1]{\begin{CJK}{UTF8}{gbsn}#1\end{CJK}}

\newcommand{\cnbkai}[1]{\begin{CJK}{UTF8}{bkai}#1\end{CJK}}
\newcommand{\cnbsmi}[1]{\begin{CJK}{UTF8}{bsmi}#1\end{CJK}}
\newcommand{\cngbsn}[1]{\begin{CJK}{UTF8}{gbsn}#1\end{CJK}}
\newcommand{\jpwada}[1]{\begin{CJK}{UTF8}{wadalab}#1\end{CJK}}
\newcommand{\jpsong}[1]{\begin{CJK}{UTF8}{song}#1\end{CJK}}
\newcommand{\jpbkai}[1]{\begin{CJK}{UTF8}{bkai}#1\end{CJK}}


\usepackage{arydshln}

\usepackage{geometry}


\begin{document}

\newgeometry{left=1.8cm,right=1.8cm,top=1cm,bottom=2.5cm}

%% use all entries of the bliography
\nocite{*}

%%-- start(titlepage)
\titlehead{Mit Dank an meine Frau für ihren wunderbaren Mut zu neuen guten
Wegen:}
%\subject{ZEN}
\title{Hannya Shingyō Lerntext}
\author{Karsten Reincke% zen License Include Module
%
% (c) Karsten Reincke, Frankfurt a.M. 2012, ff.
%
% All files of myzen are licensed under the Creative Commons Attribution 3.0
% Germany License (http://creativecommons.org/licenses/by/3.0/de/): Feel free to 
% share (to copy, distribute and transmit) or to remix (to adapt) it, if you 
% respect how you must attribute the work in the manner specified by the author:
% 
% In an internet based reuse please link the reused parts to 
% http://www.fodina.de/myzen/ and mention the original author Karsten Reincke in 
% a suitable manner.
% 
% In a paper-like reuse please insert into your preface etc. a short hint to 
% http://www.fodina.de/myzen/ and to the original author, Karsten Reincke, .
% 
% For normal quotations please use the scientific standard to cite.

\footnote{This text is licensed under the Creative Commons Attribution 3.0
License (http://creativecommons.org/licenses/by/3.0/): Feel free \enquote{to
share (to copy, distribute and transmit)} or \enquote{to remix (to adapt)}. As
a compensation, \enquote{you must attribute (your modified) work in the manner
specified by the author(s) [\ldots]}): In each reuse, mention the original
author -- Karsten Reincke -- and insert a link/hint to
\texttt{http://www.fodina.de/myzen/} }


 
 - Release % hannya shingyo cloak file
%
% (c) Karsten Reincke, Frankfurt a.M. 2016, ff.
%
% This file is licensed under the Creative Commons Attribution 3.0 Germany
% License (http://creativecommons.org/licenses/by/3.0/de/): Feel free to share
% (to copy, distribute and transmit) or to remix (to adapt) it, if you respect
% how you must attribute the work in the manner specified by the author(s):
%
% In an internet based reuse please link the reused parts to 
% http://www.fodina.de/myzen/ and mention the original author Karsten Reincke in 
% a suitable manner.
% 
% In a paper-like reuse please insert into your preface etc. a short hint to
% http://www.fodina.de/myzen/ and to the original author, Karsten Reincke, .
% 
% For normal quotations please use the scientific standard to cite.

\documentclass[
DIV=calc,
BCOR=5mm,
11pt,
headings=small,
oneside,
bibtotocnumbered]{scrartcl}

\usepackage[utf8]{inputenc}

\usepackage[]{a4,ngerman}
\usepackage[english,ngerman]{babel}
\selectlanguage{ngerman}

\usepackage{csquotes}

\usepackage[see]{jurabib}
\bibliographystyle{jurabib}
% mycsrf German jurabib configuration include module file 
%
% (c) Karsten Reincke, Frankfurt a.M. 2012, ff.
%
% This text is licensed under the Creative Commons Attribution 3.0 Germany
% License (http://creativecommons.org/licenses/by/3.0/de/): Feel free to share
% (to copy, distribute and transmit) or to remix (to adapt) it, if you respect
% how you must attribute the work in the manner specified by the author(s):
% \newline
% In an internet based reuse please link the reused parts to mycsrf.fodina.de
% and mention the original author Karsten Reincke in a suitable manner. In a
% paper-like reuse please insert a short hint to mycsrf.fodina.de and to the
% original author, Karsten Reincke, into your preface. For normal quotations
% please use the scientific standard to cite.

% the first time cite with all data, later with shorttitle
\jurabibsetup{citefull=first}

%%% (1) author / editor list configuration
%\jurabibsetup{authorformat=and} % uses 'und' instead of 'u.'
% therefore define your own abbreviated conjunction: 
% an 'and before last author explicetly written conjunction

% for authors in citations
\renewcommand*{\jbbtasep}{\ u.\ } % bta = between two authors sep
\renewcommand*{\jbbfsasep}{,\ } % bfsa = between first and second author sep
\renewcommand*{\jbbstasep}{\ u.\ }% bsta = between second and third author sep
% for editors in citations
\renewcommand*{\jbbtesep}{\ u.\ } % bta = between two authors sep
\renewcommand*{\jbbfsesep}{,\ } % bfsa = between first and second author sep
\renewcommand*{\jbbstesep}{\ u.\ }% bsta = between second and third author sep

% for authors in literature list
\renewcommand*{\bibbtasep}{\ u.\ } % bta = between two authors sep
\renewcommand*{\bibbfsasep}{,\ } % bfsa = between first and second author sep
\renewcommand*{\bibbstasep}{\ u.\ }% bsta = between second and third author sep
% for editors  in literature list
\renewcommand*{\bibbtesep}{\ u.\ } % bte = between two editors sep
\renewcommand*{\bibbfsesep}{,\ } % bfse = between first and second editor sep
\renewcommand*{\bibbstesep}{\ u.\ }% bste = between second and third editor sep

% use: name, forname, forname lastname u. forname lastname
\jurabibsetup{authorformat=firstnotreversed}
\jurabibsetup{authorformat=italic}

%%% (2) title configuration
% in every case print the title, let it be seperated from the 
% author by a colon and use the slanted font
\jurabibsetup{titleformat={all,colonsep}}
%\renewcommand*{\jbtitlefont}{\textit}

%%% (3) seperators in bib data
% separate bibliographical hints and page hints by a comma
\jurabibsetup{commabeforerest}

%%% (4) specific configuration of bibdata in quotes / footnote
% use a.a.O if possible
\jurabibsetup{ibidem=strict}
% replace ugly a.a.O. by ders., a.a.O. resp. ders., ebda.
% but if there are more than one author or girl writers?
\AddTo\bibsgerman{
  \renewcommand*{\ibidemname}{Ds.,\ a.a.O.}
  \renewcommand*{\ibidemmidname}{ds.,\ a.a.O.}
}
\renewcommand*{\samepageibidemname}{Ds.,\ ebda.}
\renewcommand*{\samepageibidemmidname}{ds.,\ ebda.}

%%% (5) specific configuration of bibdata in bibliography
% ever an in: before journal and collection/book-titles 

\renewcommand*{\bibjtsep}{in:\ }
\renewcommand*{\bibbtsep}{in:\ }

% ever a colon after author names 
\renewcommand*{\bibansep}{:\ }
% ever a semi colon after the title 
\renewcommand*{\bibatsep}{;\ }
% ever a comma before date/year
\renewcommand*{\bibbdsep}{,\ }

% let jurabib insert the S. and p. information
% no S. necessary in bib-files and in cites/footcites
\jurabibsetup{pages=format}

% use a compressed literature-list using a small line indent
\jurabibsetup{bibformat=compress}
\setlength{\jbbibhang}{1em}

% which follows the design of the cites and offers comments
\jurabibsetup{biblikecite}

% print annotations into bibliography
\jurabibsetup{annote}
\renewcommand*{\jbannoteformat}[1]{{ \itshape #1 }}

\renewcommand*{\biburlprefix}{ ( $\rightarrow$ }
\renewcommand*{\biburlsuffix}{ )}

%refine the prefix of url download
\AddTo\bibsgerman{\renewcommand*{\urldatecomment}{Referenzdownload: }}

% we want to have the year of articles in brackets
\renewcommand*{\bibaldelim}{(}
\renewcommand*{\bibardelim}{)}

%Umformatierung des Reihentitels und der Reihennummer
\DeclareRobustCommand{\numberandseries}[2]{%
\unskip\unskip%,
\space\bibsnfont{(=~#2}%
\ifthenelse{\equal{#1}{}}{)}{, [Bd./Nr.]~#1)}%
}%

%Umformatierung Referenzverweises
\usepackage{xpatch}
\AfterFile{dejbbib.ldf}{%
  \xapptocmd{\bibsgerman}{%
     \def\inname{\ifjboxford in:\else\ifjbchicago in:\else in:\fi\fi}%
    \def\incollinname{\ifjboxford in:\else\ifjbchicago in:\else in:\fi\fi}%
  }{}{}%
}



%zenk Hyphenation Include Module text
%
% (c) Karsten Reincke, Frankfurt a.M. 2012, ff.
%
% This text is licensed under the Creative Commons Attribution 3.0 Germany
% License (http://creativecommons.org/licenses/by/3.0/de/): Feel free to share
% (to copy, distribute and transmit) or to remix (to adapt) it, if you respect
% how you must attribute the work in the manner specified by the author(s):
% \newline
% In an internet based reuse please link the reused parts to zen.fodina.de
% and mention the original author Karsten Reincke in a suitable manner. In a
% paper-like reuse please insert a short hint to zen.fodina.de and to the
% original author, Karsten Reincke, into your preface. For normal quotations
% please use the scientific standard to cite.
%


\hyphenation{ my-keds there-fo-re}




% package for improving the grey value and the line feed handling
\usepackage{microtype}

%%% (3) layout page configuration %%%

% select the visible parts of a page
% S.31: { plain|empty|headings|myheadings }
\pagestyle{plain}
%\pagestyle{empty}

% select the wished style of page-numbering
% S.32: { arabic,roman,Roman,alph,Alph }
\pagenumbering{arabic}
\setcounter{page}{1}

% select the wished distances using the general setlength order:
% S.34 { baselineskip| parskip | parindent }
% - general no indent for paragraphs
\setlength{\parindent}{0pt}
\setlength{\parskip}{1.2ex plus 0.2ex minus 0.2ex}


%%% (4) general package activation %%%

%- start(footnote-configuration)

\deffootnote[1.5em]{1.5em}{1.5em}{\textsuperscript{\thefootnotemark)\ }}

% package for macking tables with broken lines
\usepackage{multirow}

%for using label as nameref
\usepackage{nameref}

%integrate nomenclature
% zen  Deutsch Nomenclation Declaration Include Module 
%
% (c) Karsten Reincke, Frankfurt a.M. 2012, ff.
%
% This text is licensed under the Creative Commons Attribution 3.0 Germany
% License (http://creativecommons.org/licenses/by/3.0/de/): Feel free to share
% (to copy, distribute and transmit) or to remix (to adapt) it, if you respect
% how you must attribute the work in the manner specified by the author(s):
% \newline
% In an internet based reuse please link the reused parts to zen.fodina.de
% and mention the original author Karsten Reincke in a suitable manner. In a
% paper-like reuse please insert a short hint to zen.fodina.de and to the
% original author, Karsten Reincke, into your preface. For normal quotations
% please use the scientific standard to cite.

\usepackage[intoc]{nomencl}
\let\abbr\nomenclature
% Deutsche Überschrift
%\renewcommand{\nomname}{Periodicals, Shortcuts, and Overlapping Abbreviations}
\renewcommand{\nomname}{Periodika, ihre Kurzformen und generelle Abkürzungen}

\setlength{\nomlabelwidth}{.20\hsize}
\renewcommand{\nomlabel}[1]{#1 \dotfill}
% reduce the line distance
\setlength{\nomitemsep}{-\parsep}
\makenomenclature


% Hyperlinks
\usepackage{hyperref}
\hypersetup{bookmarks=true,breaklinks=true,colorlinks=true,citecolor=blue,draft=false}

\usepackage[encapsulated]{CJK}
\newcommand{\cntext}[1]{\begin{CJK}{UTF8}{gbsn}#1\end{CJK}}

\newcommand{\cnbkai}[1]{\begin{CJK}{UTF8}{bkai}#1\end{CJK}}
\newcommand{\cnbsmi}[1]{\begin{CJK}{UTF8}{bsmi}#1\end{CJK}}
\newcommand{\cngbsn}[1]{\begin{CJK}{UTF8}{gbsn}#1\end{CJK}}
\newcommand{\jpwada}[1]{\begin{CJK}{UTF8}{wadalab}#1\end{CJK}}
\newcommand{\jpsong}[1]{\begin{CJK}{UTF8}{song}#1\end{CJK}}
\newcommand{\jpbkai}[1]{\begin{CJK}{UTF8}{bkai}#1\end{CJK}}


\usepackage{arydshln}

\usepackage{geometry}


\begin{document}

\newgeometry{left=1.8cm,right=1.8cm,top=1cm,bottom=2.5cm}

%% use all entries of the bliography
\nocite{*}

%%-- start(titlepage)
\titlehead{Mit Dank an meine Frau für ihren wunderbaren Mut zu neuen guten
Wegen:}
%\subject{ZEN}
\title{Hannya Shingyō Lerntext}
\author{Karsten Reincke% zen License Include Module
%
% (c) Karsten Reincke, Frankfurt a.M. 2012, ff.
%
% All files of myzen are licensed under the Creative Commons Attribution 3.0
% Germany License (http://creativecommons.org/licenses/by/3.0/de/): Feel free to 
% share (to copy, distribute and transmit) or to remix (to adapt) it, if you 
% respect how you must attribute the work in the manner specified by the author:
% 
% In an internet based reuse please link the reused parts to 
% http://www.fodina.de/myzen/ and mention the original author Karsten Reincke in 
% a suitable manner.
% 
% In a paper-like reuse please insert into your preface etc. a short hint to 
% http://www.fodina.de/myzen/ and to the original author, Karsten Reincke, .
% 
% For normal quotations please use the scientific standard to cite.

\footnote{This text is licensed under the Creative Commons Attribution 3.0
License (http://creativecommons.org/licenses/by/3.0/): Feel free \enquote{to
share (to copy, distribute and transmit)} or \enquote{to remix (to adapt)}. As
a compensation, \enquote{you must attribute (your modified) work in the manner
specified by the author(s) [\ldots]}): In each reuse, mention the original
author -- Karsten Reincke -- and insert a link/hint to
\texttt{http://www.fodina.de/myzen/} }


 
 - Release % hannya shingyo cloak file
%
% (c) Karsten Reincke, Frankfurt a.M. 2016, ff.
%
% This file is licensed under the Creative Commons Attribution 3.0 Germany
% License (http://creativecommons.org/licenses/by/3.0/de/): Feel free to share
% (to copy, distribute and transmit) or to remix (to adapt) it, if you respect
% how you must attribute the work in the manner specified by the author(s):
%
% In an internet based reuse please link the reused parts to 
% http://www.fodina.de/myzen/ and mention the original author Karsten Reincke in 
% a suitable manner.
% 
% In a paper-like reuse please insert into your preface etc. a short hint to
% http://www.fodina.de/myzen/ and to the original author, Karsten Reincke, .
% 
% For normal quotations please use the scientific standard to cite.

\documentclass[
DIV=calc,
BCOR=5mm,
11pt,
headings=small,
oneside,
bibtotocnumbered]{scrartcl}

\usepackage[utf8]{inputenc}

\usepackage[]{a4,ngerman}
\usepackage[english,ngerman]{babel}
\selectlanguage{ngerman}

\usepackage{csquotes}

\usepackage[see]{jurabib}
\bibliographystyle{jurabib}
\input{btexmat/zenJbibCfgDeInc}

\input{btexmat/zenHyphenationDeInc}


% package for improving the grey value and the line feed handling
\usepackage{microtype}

%%% (3) layout page configuration %%%

% select the visible parts of a page
% S.31: { plain|empty|headings|myheadings }
\pagestyle{plain}
%\pagestyle{empty}

% select the wished style of page-numbering
% S.32: { arabic,roman,Roman,alph,Alph }
\pagenumbering{arabic}
\setcounter{page}{1}

% select the wished distances using the general setlength order:
% S.34 { baselineskip| parskip | parindent }
% - general no indent for paragraphs
\setlength{\parindent}{0pt}
\setlength{\parskip}{1.2ex plus 0.2ex minus 0.2ex}


%%% (4) general package activation %%%

%- start(footnote-configuration)

\deffootnote[1.5em]{1.5em}{1.5em}{\textsuperscript{\thefootnotemark)\ }}

% package for macking tables with broken lines
\usepackage{multirow}

%for using label as nameref
\usepackage{nameref}

%integrate nomenclature
\input{btexmat/zenNclMetaDefineDeInc}

% Hyperlinks
\usepackage{hyperref}
\hypersetup{bookmarks=true,breaklinks=true,colorlinks=true,citecolor=blue,draft=false}

\usepackage[encapsulated]{CJK}
\newcommand{\cntext}[1]{\begin{CJK}{UTF8}{gbsn}#1\end{CJK}}

\newcommand{\cnbkai}[1]{\begin{CJK}{UTF8}{bkai}#1\end{CJK}}
\newcommand{\cnbsmi}[1]{\begin{CJK}{UTF8}{bsmi}#1\end{CJK}}
\newcommand{\cngbsn}[1]{\begin{CJK}{UTF8}{gbsn}#1\end{CJK}}
\newcommand{\jpwada}[1]{\begin{CJK}{UTF8}{wadalab}#1\end{CJK}}
\newcommand{\jpsong}[1]{\begin{CJK}{UTF8}{song}#1\end{CJK}}
\newcommand{\jpbkai}[1]{\begin{CJK}{UTF8}{bkai}#1\end{CJK}}


\usepackage{arydshln}

\usepackage{geometry}


\begin{document}

\newgeometry{left=1.8cm,right=1.8cm,top=1cm,bottom=2.5cm}

%% use all entries of the bliography
\nocite{*}

%%-- start(titlepage)
\titlehead{Mit Dank an meine Frau für ihren wunderbaren Mut zu neuen guten
Wegen:}
%\subject{ZEN}
\title{Hannya Shingyō Lerntext}
\author{Karsten Reincke\input{btexmat/zenLicenseFootnoteInc} 
 - Release \input{hs.rel} }

\maketitle
%%-- end(titlepage)
\section{Der Anlass}
 
Wäre es nicht schön, das \emph{Hannya Shingyo} -- mit anderen zusammen -- auch
auswendig vortragen zu können? Immerhin hat die Retization dieses Textes im
(Zen)-Buddhismus eine große Tradition!

Der Weg zum flüssigen Mitsprechen ist holprig: Wie lernt man solch eine sperrige
Folge japanischer Silben, wie einen so erratischen Textblock? Das Lernen dürfte
leichter fallen, wenn eine Struktur erkennbar wäre, etwa in einer
mehrspaltigen, mehrsprachigen, sinnhaft gegliederten Aufbereitung.

Dazu müsste der japanische Text jedoch recht wortgetreu übersetzt sein.
Denn nur so ließe sich die Übersetzung in einer Zeile mit dem übersetzten
Satzteil arrangieren. Würden die deutschen mit den japanisch-chinesischen
Phrasen so auch optisch korrespondieren, erschlössen sich die
Sinneinheiten direkt.

Trotzdem sollte die Übersetzung auch noch elegant sein: Das \emph{Hannya
Shingyo} ist ein Lehrtext, ein Sutra. Zuerst dürfte es mündlich vorgetragen
worden sein, als Ansprache an die Schüler. Mithin wird man darin -- ganz
sprachunabhängig -- auch rhetorische Elemente finden: Einen Interesse weckenden
\emph{Einstieg} etwa. Oder eine aufrüttelnde \emph{Kernthese}, die allmähliche
\emph{Entfaltung} ihrer Feinheiten, und die sich daran anschließende Begründung
der \emph{Konsequenzen}. Und natürlich einen einprägsamen \emph{Schluss}. Wäre es
nicht schön, wenn ein \emph{Hannya-Shingyo-Lerntext} auch das noch erkennen
ließe?

Gleichwohl müsste die Übertragung immer genau bleiben, von der Bedeutung und der
syntaktischen Struktur her\footnote{Doris Wolter hat dankenswerterweise
verschiedene Übersetzungen ins Deutsche zusammengetragen.
(\cite[vgl.][\nopage]{Wolter2010a}) Vergleicht man diese Versionen, offenbaren
sich erhebliche Unterschiede. Insbesondere das letzte Drittel des Hannya
Shingyos scheint dabei zu besonders 'poetischen' Übertragungen einzuladen.
Angesichts der existentiellen philosophischen Dimension des Zen-Buddhismus und
des Anspruchs auf letztgültige Wahrheiten im \emph{Hannya Shingyo} selbst ist
das schlicht unzufriedenstellend.}. Sie sollte so wenig als möglich
interpretieren.

Es gibt wunderbare Übersetzungen: z.B. die von
Deshimaru\footcite[vgl.][]{Deshimaru1988a}, die eher ein philosophischer
Hintergrundbericht sein will, als eine pure Übersetzung. Oder die universitär
abgesicherte, elegante Übertragung von Scheid\footcite[vgl.][]{Scheid2016a}.
Oder die wortgetreue von Boeck\footcite[vgl.][]{Boeck2016a}.

Nur liefern sie alle leider keinen mehrsprachigen, sinnhaft gegliederten
Lerntext, der bei aller Worttreue auch noch die elegante Rhetorik des Originals
erahnen ließe. Wie wäre es also mit folgender Variante?

\newpage
\section{Der Text} 

\sffamily

\begin{center}
\begin{tabular}{r|rl|rl|rl}
~ & \multicolumn{6}{l}{\textsc{Der Titel:}}\\
\hline
{\tiny\texttt{001}}&
  \multicolumn{2}{l|}{\cnbsmi{摩}  \cnbsmi{訶} \cnbsmi{般} \cnbsmi{若}} &
  \multicolumn{2}{l|}{\textbf{ma kā}  \textbf{han nya}} &
  \textrm{\emph{[Die]}}& \textrm{maha prajñā \emph{= höchste Weisheit}}\\
{\tiny\texttt{002}}&
  ~ & \cnbsmi{波} \cnbsmi{羅} \cnbsmi{蜜} \cnbsmi{多} & 
  ~ & \textbf{ha ra mi tā} & 
  {\tiny \textrm{($\rightarrow$)}} & 
    \textrm{pāramitā\emph{, die über sich hinausführt,}}\\
{\tiny\texttt{003}}& 
  ~ & \cnbsmi{心} \cnbsmi{經} &
  ~ & \textbf{shin gyō} & 
  \textrm{\emph{[als das]}} & \textrm{essentielle Sutra \emph{[schlechthin]}} \\
\hline
~ & \multicolumn{6}{l}{\textsc{Das Manifest:}}\\
\hline
{\tiny\texttt{004}}& 
~ & ~  & ~ & ~ &
  \textrm{Indem} {\tiny \textrm{($\rightarrow$)}} & \textrm{\emph{[ein der]}} \\
{\tiny\texttt{005}}&
  ~ & \cnbsmi{觀} \cnbsmi{自} \cnbsmi{在} & 
  ~ & \textbf{kan ji zai} & 
  {\tiny \textrm{($\rightarrow$)}} &
    \textrm{freien Sicht \emph{[zugewandter]}} \\
{\tiny\texttt{006}}&
  ~ & \cnbsmi{菩} \cnbsmi{薩} \cnbsmi{。}& 
  ~ & \textbf{bo} \textbf{sa}.& 
  ~ & \textrm{\emph{[lebender Buddha, ein]} Bodhisattva} \\  
{\tiny\texttt{007}}& 
  ~ & \cnbsmi{行} \cnbsmi{深} &
  ~ & \textbf{gyō} \textbf{jin} & 
  ~ & \textrm{tief \emph{[und gründlich]} praktizierend} \\  
{\tiny\texttt{008}}& 
  ~ & \cnbsmi{般} \cnbsmi{若} & 
  ~ & \textbf{han nya} & 
  ~ & \textrm{\emph{[die]} Prajñā \emph{, Weisheit}} \\  
{\tiny\texttt{009}} &
  ~ & \cnbsmi{波} \cnbsmi{羅} \cnbsmi{蜜} \cnbsmi{多}& 
  ~ & \textbf{ha ra mi ta} & 
  ~ & \textrm{Pāramitā \emph{, die über sich hinausführt,}} \\  
{\tiny\texttt{010}}&
  ~ & ~  & ~ & ~ &  ~ & \textrm{\emph{[lebt]}} \\
{\tiny\texttt{011}}&
  \cnbsmi{時}&\cnbsmi{。} &
  \textbf{ji}. & ~ &
  {\tiny \textrm{($\rightarrow$)}} & ~ \\
{\tiny\texttt{012}}& 
  ~ & ~ & ~ & ~ & ~ & \textrm{\emph{[kommt es bei ihm zum]}} \\
{\tiny\texttt{013}}& 
  ~ & \cnbsmi{照} \cnbsmi{見} &
  ~ & \textbf{shō ken} &
  ~ & \textrm{erleuchteten Sehen \emph{[, dass die]}} \\  
{\tiny\texttt{014}}& 
  ~ & \cnbsmi{五} \cnbsmi{蘊} & 
  ~ & \textbf{go on} & 
  {\tiny \textrm{($\rightarrow$)}} & \textrm{5 Skandas}  \\
{\tiny\texttt{015}}&
  ~ & \cnbsmi{皆} \cnbsmi {空} \cnbsmi{。} &
  ~ & \textbf{kai kū}. & 
  ~ & \textrm{alle leer \emph{[sind]}} \\
{\tiny\texttt{016}}&
  \cnbsmi{度} & ~ &
  \textbf{do} & ~ &
  \textrm{\emph{[und]} so} & ~ \\
{\tiny\texttt{017}}&
  ~ & \cnbsmi{一} \cnbsmi{切} &
  ~ & \textbf{is sai} &
  ~ & \textrm{entfernt \emph{[er]}} \\
{\tiny\texttt{018}}&
  ~ & \cnbsmi{苦} \cnbsmi{厄} \cnbsmi{。} & ~ &
  \textbf{ku yaku}. & 
  ~ & \textrm{Leiden \emph{[und]} Unheil.} \\
\hline
  ~ & \multicolumn{6}{l}{\textsc{Die Kernthese:}}\\
\hline
{\tiny\texttt{019}}&
  \multicolumn{2}{l|}{\cnbsmi{舍} \cnbsmi{利} \cnbsmi{子}\cnbsmi{。}}  &
  \multicolumn{2}{l|}{\textbf{sha ri shi}.} & ~ &
  \textrm{Shariputra!}\\
\hline  
{\tiny\texttt{020}}&
  ~ & ~ & ~ & ~ & 
  \multicolumn{2}{l}{\textrm{\emph{[Die 1. der 5 Skandas, nämlich die]}}} \\
{\tiny\texttt{021}}&
  ~ & \cnbsmi{色} & 
  {\tiny \textrm{($\rightarrow$)}} & \textbf{shiki} &
  ~ & \textrm{Erscheinung} \\  
{\tiny\texttt{022}}&
  \cnbsmi{不} & \cnbsmi{異} & 
  \textbf{fu} & \textbf{i} &
  \textrm{\emph{[ist]} nicht} & \textrm{getrennt \emph{[von]}} \\  
{\tiny\texttt{023}}&
  ~ & \cnbsmi{空} \cnbsmi{。} &
  {\tiny \textrm{($\rightarrow$)}} & \textbf{kū}.  &
  {\tiny \textrm{($\rightarrow$)}} & \textrm{kū, \emph{[der Leere]}} \\
\hdashline
{\tiny\texttt{024}}&
  ~ & \cnbsmi{空} &
  ~ & \textbf{kū} & 
  \textrm{\emph{[und]}} & \textrm{kū, \emph{[die Leere]}} \\
{\tiny\texttt{025}}&
  \cnbsmi{不} & \cnbsmi{異} &
  \textbf{fu} & \textbf{i} &
  \textrm{\emph{[ist]} nicht} & \textrm{getrennt \emph{[von]}} \\  
{\tiny\texttt{026}}&
  ~ & \cnbsmi{色} \cnbsmi{。} &
  ~ & \textbf{shiki}. &
  ~ & \textrm{\emph{[der]} Erscheinung.} \\
\hline
{\tiny\texttt{027}}&
  ~ & ~ & ~ & ~ & \multicolumn{2}{l}{\textrm{~\emph{Ja, mehr noch:}}}  \\  
{\tiny\texttt{028}}&
  ~ & \cnbsmi{色} &
  ~ & \textbf{shiki} & 
  ~ & \textrm{\emph{[Die]} Erscheinung} \\  
{\tiny\texttt{029}}&
  \cnbsmi{即} & \cnbsmi{是} & 
  \textbf{soku} & \textbf{ze} &
  \textrm{ist} & \textrm{eigentlich} \\  
{\tiny\texttt{030}}&
  ~ & \cnbsmi{空} \cnbsmi{。} &
  ~ & \textbf{kū}. &
  ~ & \textrm{kū, \emph{[die Leere]}} \\
 \hdashline
 {\tiny\texttt{031}}&
  ~ & \cnbsmi{空} &
  ~ & \textbf{kū} & 
  \textrm{\emph{[und]}} & \textrm{kū, \emph{[die Leere]}} \\
{\tiny\texttt{032}}&
  \cnbsmi{即} & \cnbsmi{是} & 
  \textbf{soku} & \textbf{ze} &
  \textrm{ist} & \textrm{eigentlich} \\  
{\tiny\texttt{033}}&
 ~ & \cnbsmi{色} \cnbsmi{。} &
 ~ & \textbf{shiki}. &~
 ~ & \textrm{\emph{[die]} Erscheinung.} \\
 \hline
 {\tiny\texttt{034}}&
    ~ & ~ & ~ & ~ & \multicolumn{2}{l}{
    \textrm{\emph{[Und bei den anderen 4 Skandas, also beim]}}}\\
 {\tiny\texttt{035}}&
  ~ & \cnbsmi{受} &
  ~ & \textbf{ju} &
  ~ & \textrm{Empfinden,} \\
{\tiny\texttt{036}}&
  ~ & \cnbsmi{想} &
  ~ & \textbf{sō} &
  ~ & \textrm{Wahrnehmen,} \\
 {\tiny\texttt{037}}&
  ~ & \cnbsmi{行} &
  ~ & \textbf{gyō} &
  ~ & \textrm{Wollen \emph{[und]}} \\
 {\tiny\texttt{038}}&
  ~ & \cnbsmi{識} &
  ~ & \textbf{shiki}. &
  ~ & \textrm{Unterscheiden}, \\
{\tiny\texttt{039}}&
  \cnbsmi{亦} & \cnbsmi{復} \cnbsmi{如} \cnbsmi{是} \cnbsmi{。} &
  \textbf{yaku} & \textbf{bu nyo ze}. &
  \textrm{auch \emph{[da]}} & \textrm{ist \emph{[es]} wieder gleich}. \\
\hline
  ~ & \multicolumn{6}{l}{\textsc{Die ex negativo Definition von \textrm{kū}:}}\\
\hline
{\tiny\texttt{040}} &
  \multicolumn{2}{l|}{\cnbsmi{舍} \cnbsmi{利} \cnbsmi{子} \cnbsmi{。}} &
  \multicolumn{2}{l|}{\textbf{sha ri shi}.} &
  ~ & \textrm{Shariputra!}\\
\hline
{\tiny\texttt{041}}&
  \cnbsmi{是} & \cnbsmi{諸} &
  \textbf{ze} & \textbf{sho} &
  \textrm{\emph{[Es]} ist} & \textrm{alles} \\
{\tiny\texttt{042}}&
  ~ & \cnbsmi{法} &
  ~ & \textbf{hō} &
  ~ & \textrm{Seiende} \\
{\tiny\texttt{043}}&
  ~ & \cnbsmi{空} \cnbsmi{相} \cnbsmi{。} &
  ~ & \textbf{kū sō}. &
  ~ & \textrm{\emph{[ein]} Aspekt \emph{[von]} kū}: \\
\hdashline
{\tiny\texttt{044}}&
  \cnbsmi{不} & \cnbsmi{生} &
  \textbf{fu} & \textbf{shō} & 
  \textrm{nicht} & \textrm{geboren \emph{[bzw.]} geschaffen} \\
{\tiny\texttt{045}}&
  \cnbsmi{不} & \cnbsmi{滅} \cnbsmi{。} &
  \textbf{fu} & \textbf{metsu}. &
  \textrm{nicht} & \textrm{gestorben \emph{[bzw.]} ausgelöscht},\\
\hdashline
{\tiny\texttt{046}}&
  \cnbsmi{不} & \cnbsmi{垢} &
  \textbf{fu} & \textbf{ku} &
  \textrm{nicht} & \textrm{befleckt} \\
{\tiny\texttt{047}}&
  \cnbsmi{不} & \cnbsmi{淨} &
  \textbf{fu} & \textbf{jō}. &
  \textrm{nicht} & \textrm{rein}, \\
\hdashline
\end{tabular}

\begin{tabular}{r|rl|rl|rl}
\hdashline
{\tiny\texttt{048}}&
  \cnbsmi{不} & \cnbsmi{增} &
  \textbf{fu} & \textbf{zō} &
  \textrm{nicht} & \textrm{zunehmend} \\
{\tiny\texttt{049}}&
  \cnbsmi{不} & \cnbsmi{減} &
  \textbf{fu} & \textbf{gen}. &
  \textrm{nicht} & \textrm{abnehmend}. \\
\hline
{\tiny\texttt{050}}&
  \cnbsmi{是} & \cnbsmi{故} &
  \textbf{ze} & \textbf{ko} &
  \textrm{Mithin} & {\tiny ($\rightarrow$)} \textrm{\emph{[gibt es]}}\\
{\tiny\texttt{051}}&
  ~ & \cnbsmi{空} \cnbsmi{中} \cnbsmi{。} &
  ~ & \textbf{kū chū}. &
  ~ & \textrm{in kū} \\
{\tiny\texttt{052}}&
  ~ & ~  & ~ & ~ &  ~ & \textrm{\emph{[keines der 5 Skandhas, also]}} \\
{\tiny\texttt{053}}&
  \cnbsmi{無} & \cnbsmi{色} \cnbsmi{。} &
  \textbf{mu} & \textbf{shiki} &
  \textrm{kein} & \textrm{Erscheinen}, \\
{\tiny\texttt{054}}&
  \cnbsmi{無} & \cnbsmi{受} &
  \textbf{mu} & \textbf{ju} & 
  \textrm{kein} & \textrm{Empfinden,} \\
{\tiny\texttt{055}}&
  ~ & \cnbsmi{想} &
  ~ & \textbf{sō} &
  ~ & \textrm{Wahrnehmen,} \\
{\tiny\texttt{056}}&
  ~ & \cnbsmi{行} &
  ~ & \textbf{gyō} &
  ~ & \textrm{Wollen \emph{[oder]}} \\
{\tiny\texttt{057}}&
  ~ & \cnbsmi{識} \cnbsmi{。} &
  ~ & \textbf{shiki}. & 
  ~ & \textrm{Unterscheiden}, \\
\hdashline
 {\tiny\texttt{058}}&
  \cnbsmi{無} & \cnbsmi{眼} &
  \textbf{mu} & \textbf{gen} &
  \textrm{keine} & \textrm{Augen,} \\
{\tiny\texttt{059}}&
  ~ & \cnbsmi{耳} &
  ~ & \textbf{ni} &
  ~ & \textrm{Ohren,} \\
{\tiny\texttt{060}}&
  ~ & \cnbsmi{鼻} &
  ~ & \textbf{bi} &
  ~ & \textrm{Nase,} \\
{\tiny\texttt{061}}&
  ~ & \cnbsmi{舌} &
  ~ & \textbf{ze} &
  ~ & \textrm{Zunge,} \\
{\tiny\texttt{062}}&
  ~ & \cnbsmi{身} &
  ~ & \textbf{shin} &
  \textrm{\emph{[keinen]}} & \textrm{Tastsinn \emph{[und]}} \\
{\tiny\texttt{063}}&
  ~ & \cnbsmi{意} \cnbsmi{。} &
  ~ & \textbf{i}. & 
  \textrm{\emph{[kein]}} & \textrm{Denkvermögen}. \\
\hdashline
{\tiny\texttt{064}}&
   \cnbsmi{無} & \cnbsmi{色} &
   \textbf{mu} & \textbf{shiki} &
   \textrm{keine} & \textrm{Farbe,} \\
{\tiny\texttt{065}}&
   ~ & \cnbsmi{聲} &
   ~ & \textbf{shō} &
   \textrm{\emph{[keinen]}} & \textrm{Klang,} \\
{\tiny\texttt{066}}&
   ~ & \cnbsmi{香} &
   ~ & \textbf{kō} &
   ~ & \textrm{Geruch,} \\
{\tiny\texttt{067}}&
  ~ & \cnbsmi{味} &
  ~ & \textbf{mi} &
  ~ & \textrm{Geschmack,} \\
{\tiny\texttt{068}}&
  ~ & \cnbsmi{觸} &
  ~ & \textbf{soku} & 
  \textrm{\emph{[keine]}} & \textrm{Berührung \emph{[und]}} \\
{\tiny\texttt{069}}&
  ~ & \cnbsmi{法} \cnbsmi{。} &
  ~ & \textbf{hō}.&
  \textrm{\emph{[keinen]}} & \textrm{Gedanken}; \\
\hline
{\tiny\texttt{070}}&
  ~ & ~ & ~ & ~ & ~ & \textrm{\emph{[Also gibt es in kū]}} \\
\hdashline
{\tiny\texttt{071}}&
  \cnbsmi{無} & \cnbsmi{眼} \cnbsmi{界} \cnbsmi{。} &
  \textbf{mu} & \textbf{gen kai} &
  \textrm{nicht} & \textrm{die sichtbare Welt} {\tiny ($\rightarrow$)} \\
{\tiny\texttt{072}}&
  \cnbsmi{乃}\cnbsmi{至} & ~ & 
  \textbf{nai shi} & ~ &
  \textrm{\emph{[und]} {\tiny ($\rightarrow$)}} & 
   \textrm{darum insbesondere [auch]}\\
{\tiny\texttt{073}}&
  \cnbsmi{無} & \cnbsmi{意} \cnbsmi{識} \cnbsmi{界} \cnbsmi{。}&
  \textbf{mu} & \textbf{i shiki kai}.&
  \textrm{nicht} & 
    \textrm{die Welt der Vorstellungen {\tiny ($\rightarrow$)}} \\
\hdashline
{\tiny\texttt{074}}&
  \cnbsmi{無} & \cnbsmi{無} \cnbsmi{明} \cnbsmi{。} &
  \textbf{mu} & \textbf{mu myō} &
  \textrm{kein} & \textrm{Nicht-Wissen \emph{[und]}} \\
{\tiny\texttt{075}}&
  \cnbsmi{亦} & ~ & 
  \textbf{yaku} & ~ &
  \textrm{auch} & ~ \\  
{\tiny\texttt{076}}&
  \cnbsmi{無} & \cnbsmi{無} \cnbsmi{明} \cnbsmi{盡} \cnbsmi{。} &
  \textbf{mu} & \textbf{mu myō jin.} &
  \textrm{kein} & \textrm{Ende vom Nicht-Wissen} {\tiny ($\rightarrow$)} \\
\hdashline
{\tiny\texttt{077}}&
  \cnbsmi{乃}\cnbsmi{至} & ~ & 
  \textbf{nai shi} & ~ &
  \textrm{\emph{[und]} {\tiny ($\rightarrow$)}} & 
   \textrm{darum insbesondere \emph{[auch]}}\\
\hdashline
{\tiny\texttt{078}}&
  \cnbsmi{無} & \cnbsmi{老} \cnbsmi{死} \cnbsmi{。} &
  \textbf{mu} & \textbf{rō shi} &
  \textrm{kein} & \textrm{Altern und Tod \emph{[und]}} \\
{\tiny\texttt{079}}&
  \cnbsmi{亦} & ~ &
  \textbf{yaku} & ~ &
  \textrm{auch} & ~ \\
{\tiny\texttt{080}}&
  \cnbsmi{無} & \cnbsmi{老} \cnbsmi{死} \cnbsmi{盡} \cnbsmi{。} &
  \textbf{mu} &
  \textbf{rō shi jin}. &
  \textrm{kein} & \textrm{Ende von Altern und Tod {\tiny ($\rightarrow$)}}\\
\hline
{\tiny\texttt{081}}&
  \cnbsmi{無} & \cnbsmi{苦} & 
  \textbf{mu} & \textbf{ku} &
  \textrm{kein} & \textrm{Leiden,} \\  
{\tiny\texttt{082}}&
  ~ & \cnbsmi{集} &
  ~ & \textbf{shū} & 
  ~ & \textrm{Anhäufen,} \\  
{\tiny\texttt{083}}&
  ~ & \cnbsmi{滅} &
  ~ & \textbf{metsu} & 
  ~ & \textrm{Verlöschen \emph{[und]}} \\  
{\tiny\texttt{084}}&
  ~ & \cnbsmi{道} \cnbsmi{。} & 
  ~ & \textbf{dō}. &
  \textrm{\emph{[keinen]}} & \textrm{Weg,} \\
\hdashline 
{\tiny\texttt{085}}&
  \cnbsmi{無} & \cnbsmi{智} &
  \textbf{mu} & \textbf{chi} &
  \textrm{keine} & \textrm{Erkenntnis \emph{[und]}} \\  
{\tiny\texttt{086}}&
  \cnbsmi{亦} & ~ &
  \textbf{yaku} & ~ &
  \textrm{auch} & ~ \\
{\tiny\texttt{087}}&
  \cnbsmi{無} & \cnbsmi{得} \cnbsmi{。} &
  \textbf{mu} & \textbf{toku}. &
  \textrm{keinen} & \textrm{Gewinn,} \\  
{\tiny\texttt{088}}&
  \cnbsmi{以} & ~ &
  \textbf{I} & ~ &
  \textrm{weil} &  \textrm{\emph{[kū]}} \\  
{\tiny\texttt{089}}&
  \cnbsmi{無} & \cnbsmi{所} \cnbsmi{得}  &
  \textbf{mu} & \textbf{sho tok}u  &
  \textrm{kein} & \textrm{Ort \emph{[des]} Gewinnens \emph{[ist]}.} \\
\hline
  ~ & \multicolumn{6}{l}{\textsc{Die praktische Konsequenz:}}\\
\hline
{\tiny\texttt{090}}&
  \cnbsmi{故}\cnbsmi{。} & 
    \cnbsmi{菩} \cnbsmi{提} \cnbsmi{薩} \cnbsmi{捶}\cnbsmi{。}  &
  \textbf{ko}. & \textbf{bo dai sat ta.} &
  \textrm{Darum} & 
    \textrm{\emph{[gilt:] [Ein]} Bodhisattva \emph{[zu sein,]}}\\
{\tiny\texttt{091}}&
  ~ & \cnbsmi{依} &
  ~ & \textbf{e} & 
  ~ & \textrm{bedingt \emph{[die]}} \\
\newline
{\tiny\texttt{092}}&
  ~ & \cnbsmi{般} \cnbsmi{若}  &
  ~ & \textbf{han nya} &
  ~ & \textrm{Prajñā \emph{Weisheit}} \\  
{\tiny\texttt{093}}&
  ~ & \cnbsmi{波} \cnbsmi{羅} \cnbsmi{蜜} \cnbsmi{多} &
  ~ & \textbf{ha ra mi ta} &
  ~ & \textrm{Pāramitā\emph{, die über sich hinausführt.}} \\
\hdashline
{\tiny\texttt{095}}&
  \cnbsmi{故}\cnbsmi{。} & 
    \cnbsmi{心} \cnbsmi{無} \cnbsmi{罫} \cnbsmi{礙} \cnbsmi{。} &
  \textbf{ko.} & \textbf{shin mu kei ge} &
  \textrm{Darum} & 
    \textrm{\emph{[wird sein]} Geist nicht behindert.} \\
\hdashline
{\tiny\texttt{096}}&
  ~ & \cnbsmi{無} \cnbsmi{罫} \cnbsmi{礙} & 
  ~ & \textbf{mu kei ge} & 
 \multicolumn{2}{l}
  {\textrm{\emph{[Und da der]} nicht behindert \emph{[wird]},}}\\
{\tiny\texttt{097}}&
  \cnbsmi{故}\cnbsmi{。} & \cnbsmi{無} \cnbsmi{有} & 
  \textbf{ko.} & \textbf{mu u }& 
  \textrm{darum} & \textrm{hat \emph{[der Bodhisattva]} keine} \\
{\tiny\texttt{098}}&
  ~ & \cnbsmi{恐} \cnbsmi{怖}\cnbsmi{。} &
  ~ & \textbf{ku fu} &
  ~ & \textrm{Furcht}. \\ 
\hline
\end{tabular}

\begin{tabular}{r|rl|rl|rl}
\hdashline
{\tiny\texttt{099}}&
  ~ & \cnbsmi{遠} \cnbsmi{離} &
  ~ & \textbf{on ri} &
  \textrm{\emph{[Das]}} & \textrm{übersteigend\emph{[, was er sich]}} \\      
{\tiny\texttt{100}}&
  ~ & \cnbsmi{一} \cnbsmi{切} &
  ~ & \textbf{is sai} & 
  ~ & \textrm{entfernt \emph{[hat -- nämlich]} }\\
{\tiny\texttt{101}}&
  ~ & \cnbsmi{顛} \cnbsmi{倒} &
  ~ & \textbf{ten dō} &
  ~ & \textrm{Täuschungen \emph{[und]}} \\      
{\tiny\texttt{102}}&
  ~ & \cnbsmi{夢} \cnbsmi{想} \cnbsmi{。} &
  ~ & \textbf{mu sō}. &
  ~  & \textrm{Illusionen \emph{[--]}} \\      
{\tiny\texttt{103}}&
  ~ & \cnbsmi{究} \cnbsmi{竟} &
  ~ & \textbf{ku gyō} &
  ~ & \textrm{erreicht \emph{[er]} schließlich } \\      
{\tiny\texttt{104}}&
  ~ & \cnbsmi{涅} \cnbsmi{槃} \cnbsmi{。}&
  ~ & \textbf{ne han}. &
  ~  & \textrm{\emph{[das]} Nirvana.} \\      
\hline   
{\tiny\texttt{105}}&
  ~ & \cnbsmi{三} \cnbsmi{世} &
  ~ & \textbf{san ze} &
  \textrm{\emph{[Zudem]}} & \textrm{\emph{[gilt seit]} drei Zeitaltern} \\      
{\tiny\texttt{106}}&
  ~ & \cnbsmi{諸} \cnbsmi{佛} \cnbsmi{。} &
  ~ & \textbf{sho butsu} &
  \textrm{\emph{für}} & 
    \textrm{alle Buddhas: \emph{ihre Buddhaschaft}} \\     
 {\tiny\texttt{107}}&
  ~ & \cnbsmi{依} &
  ~ & \textbf{e} &
  ~ & \textrm{bedingt \emph{[die]}} \\  
{\tiny\texttt{108}}&
  ~ & \cnbsmi{般} \cnbsmi{若}  &
  ~ & \textbf{han nya} &
  ~ & \textrm{Prajñā \emph{(Weisheit)}} \\  
{\tiny\texttt{109}}&
  ~ & \cnbsmi{波} \cnbsmi{羅} \cnbsmi{蜜} \cnbsmi{多} &
  ~ & \textbf{ha ra mi ta} &
  ~ & \textrm{Pāramitā\emph{, die über sich hinausführt.}} \\
\hline  
{\tiny\texttt{110}}&
  \cnbsmi{故}\cnbsmi{。} & \cnbsmi{得} &
  \textbf{ko}. & \textbf{toku} &
  \textrm{Darum} & \textrm{gewinnen sie die} \\  
{\tiny\texttt{111}}&
  ~ & \cnbsmi{阿} \cnbsmi{耨} \cnbsmi{多} \cnbsmi{羅} & 
  ~ & \textbf{a noku ta ra} &
  {\tiny ($\rightarrow$)} & \textrm{anuttara} \textrm{\emph{höchste}} \\
{\tiny\texttt{112}}&
  ~ & \cnbsmi{三} \cnbsmi{藐} &
  ~ & \textbf{san myaku} &
  {\tiny ($\rightarrow$)} & \textrm{samyak} \textrm{\emph{vollkommene}} \\      
{\tiny\texttt{113}}&
  ~ & \cnbsmi{三} \cnbsmi{菩} \cnbsmi{提} \cnbsmi{。} &
  ~ & \textbf{san bo dai}. &
  {\tiny ($\rightarrow$)} & \textrm{sambodhi} \textrm{\emph{Erleuchtung}} \\ 
 \hline 
{\tiny\texttt{114}}&
  \cnbsmi{故} & \cnbsmi{知} &
  \textbf{ko} & \textbf{chi} &
  \textrm{Darum} &  \textrm{wisse \emph{[nun Du Deinerseits:]}} \\  
{\tiny\texttt{115}}&
  ~ & \cnbsmi{般} \cnbsmi{若} &
  ~ & \textbf{han nya} &
  \textrm{\emph{[Das]}} & \textrm{Prajñā} {\tiny ($\rightarrow$)} \\  
{\tiny\texttt{116}}&
  ~ & \cnbsmi{波} \cnbsmi{羅} \cnbsmi{蜜} \cnbsmi{多}&
  ~ & \textbf{ha ra mi ta}.&
  ~ & \textrm{Pāramitā} {\tiny ($\rightarrow$)} \\ 
{\tiny\texttt{117}}&
  \cnbsmi{是} & \cnbsmi{大} \cnbsmi{神} \cnbsmi{咒} \cnbsmi{。} &
  \textbf{ze} & \textbf{dai jin shu}. &
  \textrm{ist} & \textrm{\emph{[ein]} großes wunderbares Mantra}; \\  
{\tiny\texttt{118}}&
  \cnbsmi{是} & \cnbsmi{大} \cnbsmi{明} \cnbsmi{咒} &
  \textbf{ze} & \textbf{dai myō shu}. &
  \textrm{\emph{[es]} ist} & \textrm{\emph{[ein]} großes leuchtendes Mantra}, \\ 
{\tiny\texttt{119}}&
  \cnbsmi{是} & \cnbsmi{無} \cnbsmi{上} \cnbsmi{咒} \cnbsmi{。} &
  \textbf{ze} & \textbf{mu jō shu}. &
  \textrm{\emph{[es]} ist} & 
    \textrm{\emph{[das]} {\tiny ($\rightarrow$)} höchste Mantra} \\
{\tiny\texttt{120}}&
  \cnbsmi{是} & \cnbsmi{無} \cnbsmi{等} \cnbsmi{等} \cnbsmi{咒} \cnbsmi{。}  &
  \textbf{ze} &  \textbf{mu tō dō shu}.  &
  \textrm{\emph{[es]} ist} & \textrm{\emph{[das]} nicht übersteigbare Mantra} \\  
{\tiny\texttt{121}}&
  ~ & \cnbsmi{能} & 
  ~ & \textbf{nō} &  
  ~ \textrm{\emph{[es]}} & \textrm{dient \emph{[dem]}} \\
{\tiny\texttt{122}}&
  ~ & \cnbsmi{除} \cnbsmi{一} \cnbsmi{切} & 
  ~ & \textbf{jo is sai} & 
  ~ & \textrm{Beseitigen \emph{[und]} Abschneiden} \\
{\tiny\texttt{123}}&
  ~ & \cnbsmi{苦} \cnbsmi{。} & 
  ~ & \textbf{ku}. & 
 ~ & \textrm{\emph{[von]} Leiden.} \\
\hline 
  ~ & \multicolumn{6}{l}{\textsc{Das Fazit \ldots}}\\
\hline
{\tiny\texttt{124}}&
  ~ & ~ & ~ & ~ &  & \textrm{\emph{[Und weil dies]}} \\      
{\tiny\texttt{125}}&
  ~ & \cnbsmi{真} \cnbsmi{實} &
  ~ & \textbf{shin jitsu}  &
  ~ & \textrm{wirklich \emph{[und]} {\tiny ($\rightarrow$)}} \\  
{\tiny\texttt{126}}&
  ~ & \cnbsmi{不} \cnbsmi{虛} \cnbsmi{。}&
  ~ & \textbf{fu ko} &
  ~ & \textrm{nicht unwahr \emph{[ist,]}} \\
{\tiny\texttt{127}}&
  \cnbsmi{故} & ~ &
  \textbf{ko} & ~ &
  \textrm{darum} & \textrm{\emph{[wird die]} }\\  
{\tiny\texttt{128}}&
  ~ & \cnbsmi{說} & 
  ~ & \textbf{setsu} &
  ~ & \textrm{Bedeutung \emph{[der]}} \\  
 {\tiny\texttt{129}}&
  ~ & \cnbsmi{般} \cnbsmi{若} &
  ~ & \textbf{han nya} &
  ~ & \textrm{Prajñā} \\  
{\tiny\texttt{130}}&
  ~ & \cnbsmi{波} \cnbsmi{羅} \cnbsmi{蜜} \cnbsmi{多}&
  ~ & \textbf{ha ra mi ta} &
  ~ & \textrm{Pāramitā} \\ 
{\tiny\texttt{131}}&
  ~ & \cnbsmi{咒} &
  ~ & \textbf{shu}. & 
  \textrm{\emph{[als]}} & \textrm{Mantra} \\
{\tiny\texttt{132}}&
  \cnbsmi{即} & ~ & 
  \textbf{soku} & ~ &
  \multicolumn{2}{l}{\textrm{eigentlich \emph{[auch durch die]}}} \\  
{\tiny\texttt{133}}&
  ~ & \cnbsmi{說} &
  ~ & \textbf{setsu} & 
  ~ & \textrm{Bedeutung \emph{[des nun}}\\
{\tiny\texttt{134}}&
  ~ & \cnbsmi{咒} &
  ~ & \textbf{shu}  & 
  ~ & \textrm{\emph{folgenden]} Mantras} \\
{\tiny\texttt{135}}&
  ~ & \cnbsmi{曰} &
  ~ & \textbf{watsu} & 
  ~ & \textrm{ausgesagt:} \\ 
\hline
\end{tabular}

\begin{tabular}{r|rl|rl|rl}
\hline
  ~ & \multicolumn{6}{l}{\textsc{\ldots in Form eines Mantras:}}\\
\hline
{\tiny\texttt{136}}&
  ~ & ~ & ~ & ~ & ~ & \textrm{\emph{Lasst uns}} \\
{\tiny\texttt{137}}&
  ~ & \cnbsmi{羯} \cnbsmi{諦}&
  ~ & \textbf{gya tei} &
  ~ & \textrm{hinübergehen,} \\  
{\tiny\texttt{138}}&
  ~ & \cnbsmi{羯} \cnbsmi{諦}&
  ~ & \textbf{gya tei} &
  ~ & \textrm{hinübergehen,} \\  
{\tiny\texttt{139}}&
  \cnbsmi{波} \cnbsmi{羅} & \cnbsmi{羯} \cnbsmi{諦}&
  \textbf{ha ra} & \textbf{gya tei} &
  \multicolumn{2}{l}{\textrm{mit anderen hinübergehen,}} \\
{\tiny\texttt{140}}&
  \cnbsmi{波} \cnbsmi{羅} \cnbsmi{僧} & \cnbsmi{羯} \cnbsmi{諦}&
  \textbf{ha ra sō} & \textbf{gya tei} &
  \multicolumn{2}{l}{\textrm{mit anderen vollständig hinübergehen,}} \\
 \hdashline
 {\tiny\texttt{141}}&
  \cnbsmi{菩} \cnbsmi{提} \cnbsmi{薩} & \cnbsmi{婆} \cnbsmi{訶}&
  \textbf{bo ji} & \textbf{so wa ka} &
  \textrm{\emph{auf dem}} & \textrm{Weg \emph{zur} Vollendung.} \\
 \hline
   ~ & \multicolumn{6}{l}{\textsc{Punkt}}\\
 \hline
 {\tiny\texttt{142}}& 
  ~ & \cnbsmi{般} \cnbsmi{若} &
  ~ & \textbf{han nya} &
  \textrm{\emph{[So die]}}& \textrm{prajñā \emph{, Weisheit}}\\
{\tiny\texttt{143}}& 
  ~ & \cnbsmi{心} \cnbsmi{經} &
  ~ & \textbf{shin gyō}. & 
  \textrm{\emph{[als]}} & \textrm{essentielles Sutra} \\
 \hline 
%\end{longtable}
\end{tabular}
\end{center}
\rmfamily

\section{Die Gestaltung} 

In der linken Spalte meiner Lernversion des Hannya Shingyos steht der
chinesische Text. Er folgt dem universitär abgesicherten Text von
Scheid\footcite[vgl.][\nopage]{Scheid2016a} und ist -- entsprechend der
europäischen Tradition -- von links nach rechts und von oben nach unten zu
lesen. Er unterscheidet sich von den chinesischen Versionen, die die anderen
hier zitierten Autoren präsentieren, höchstens in der Punktion.

Die mittlere Spalte meiner Lernversion präsentiert den japanischen Text in
europäischer Umschrift. Sie folgt -- mit drei Ausnahmen -- dem Text von
Deshimaru\footcite[vgl.][30]{Deshimaru1988a} und ist ebenfalls von links nach
rechts und von oben nach unten zu lesen. Die erste Ausnahme betrifft das Wort
\emph{bo sa} in Zeile [006]. Hier steht bei Deshimaru \emph{bo satsu}. Die
zweite Ausnahme betrifft das Wort \emph{ze} in Zeile [061]. Hier steht bei
Deshimaru \emph{ze(tsu)}. Da in der Sangha, zu der ich mich hingezogen
fühle\footcite[vgl.][\nopage]{DaiShinZen2016a}, die Silbe \emph{tsu} nicht
gesprochen wird, habe ich mir erlaubt, es in meiner Lernversion zu unterdrücken.
Inhaltlich entsteht dadurch keine Veränderung, phonetisch nur eine geringe: das
auslautend \emph{u} wird im Japanischen fast nicht gesprochen, jedenfalls noch
weniger als das deutsche Auslaut-e in \emph{Stange} oder \emph{Karte}. Die
dritte Ausnahme betrifft die Groß- und Kleinschreibung: ich habe die konsequente
Kleinschreibung der Version von Scheid übernommen. Die Großschreibung nach einem
Punkt signalisiert harte syntaktische Abschlüsse, die semantisch so nicht
stimmen.

Meine Übersetzung ins Deutsche folgt in der Regel der anregenden, wortweisen
Übersetzung von Boeck\footcite[vgl.][\nopage]{Boeck2016a}, allerdings im
Abgleich mit den Erläuterungen von Deshimaru und Scheid. Mein eigenes Zutun
wollte von Anfang an nicht mehr bieten als eine geschickte Anordnung, bei der
eine möglichst wortgetreue Übersetzung zeilenmäßig in der Nähe der zu
übersetzenden Phrase bleibt. Das Hannya Shingyo sollte in sinnhaften Einheiten
lernbar gemacht werden. Um das zu erreichen, habe ich die großen syntaktischen
Freiheiten der deutschen Sprache genutzt: im Zweifel habe ich die etwas
geschrobenere Formulierung mit genauer Zuordnung der eleganteren, aber
entfernenden vorgezogen.

Um meine eher syntaktisch motivierten Zutaten als solche zu kennzeichnen, habe
ich sie in eckige Klammern eingeschlossen und kursiv gesetzt. Der deutsche Text
sollte sich mit diesen Zutaten schlüssig von links nach rechts und oben nach
unten lesen lassen. Unmarkierte deutsche Wörter sollten in der Zeile stehen, in
denen auch die chinesischen und japanischen Korrelate stehen - jedoch nicht
immer in derselben Reihenfolge, wie die Originale.

Und noch zwei letzte typographische Aufschlüsselung: 

\begin{enumerate}
  \item Die chinesische Schrift ist eine Begriffsschrift. Trotzdem enthält sie
  auch syntaktische Konnektoren, etwa die Negationen \emph{mu} (= \cnbsmi{無})
  und  \emph{fu} (= \cnbsmi{不}), die additive Konjunktion \emph{yaku} (=
  \cnbsmi{亦} = auch), die einfache Schlussfolgerung \emph{ko} (= \cnbsmi{故} =
  darum) oder die betonte Schlussfolgerung \emph{nai shi} (= \cnbsmi{乃}
  \cnbsmi{至} = darum insbesondere)\footnote{Boeck übersetzt \emph{fu} mit der
  deutschen Vorsilbe \emph{un-} und \emph{mu} mit der expliziten Negation
  \emph{nicht}. \emph{yaku} übersetzt er ebenfalls als auch. \emph{ko} übersetzt
  er wörtlich als \emph{Ursache}. Und \emph{nai shi} übersetzt er als \emph{dann
  extrem}, was ich als \emph{darum inbesondere}
  übernehme.\cite[vgl][\nopage]{Boeck2016a}}. Diese Patikel strukturieren den
  Text logisch. Deshalb habe ich sie in der linearen Anordnung jeweils nach
  links herausgezogen. Im selben Sinne habe ich auch einige andere, gliedernde
  Partikel optisch arrangiert.
  \item Im Text erscheint gelegentlich ein verweisender Pfeil
  \emph{$\rightarrow$}. Zu diesen Zeilen gibt es eine Erläuterung der
  Übersetzung. Die Zeilennummern werden im Kapitel mit den Übersetzungshinweisen
  als Referenz benutzt.
\end{enumerate}

\section{Die Übersetzung} 

Einige Entscheidungen habe ich im folgenden erläutert. Mit ist natürlich klar,
dass eine wirklich wissenschaftliche Aufbereitung viele Aspekte und Behauptungen
nachweisen müsste, auf die ich hier ohne Nachweis zurückgreife. Sie sind das
Ergebnis der Arbeit der anderen Autoren. Ihnen gebührt dafür Respekt,
Anerkennung und Dank, nicht mir. In einer späteren Version werde ich die
Nachweise sicher nachholen. Bis dahin möge man mir nachsehen, dass ich einfach
nur eine besser zu lernende Version erstellen wollte.

\begin{description}

  \item[001-003:] Das Hannya Shingyo ist ursprünglich in Sanskrit geschrieben,
  von dort ins Chinesische übertragen und von da aus ist es dann noch einmal ins
  Japanische übersetzt worden. Das Chinesische selbst ist eine Begriffsschrift,
  sodass sich die Übersetzung ins Japanische auf die Definition einer 'anderen'
  Aussprache konzentrieren konnte. Allerdings hatte die chinesische Version
  einige ursprüngliche Formulierung als 'wörtliche Zitate' bewahrt. Dabei ist
  die Aussprache des Sanskrit mit chinesischen Silben lautlich nachgebildet
  worden. Die Übertragung ins Japanische hat diese Idee übernommen.
  Damit entsteht jedoch eine 'Doppeldeutigkeit'. Denn die das Sanskrit mehr oder
  minder gut nachbildenden japanischen Wörter und Silben haben natürlich eine
  eigene unabhängige Bedeutung. Dem entsprechend wird gelegentlich gesagt, die
  Übertragungen hätten die Bedeutung des Hannya Shingyos
  \enquote{vertieft}\footcite[vgl.][56]{Deshimaru1988a}. Das \emph{Hannya
  Shingyo} als Name des Textes ist jedenfalls das erste Zitat aus dem Sanskrit.

  \item[005-006:] Der Ausdruck \emph{kan ji zai bo sa} bildet auch ein solches
  lautliches Zitat, allerdings in etwas \emph{verschleierter Form}: er soll den
  Ausdruck \emph{Boddhisattva Avalokiteshvara} wiedergeben. Dabei beziehen sich
  die Silben \emph{bo sa} direkt auf auf den Titel \emph{Boddhisattva}.
  Titelträger ist im Original \emph{Avalokitesvara}, ein Schüler von Buddha.
  Dieser hat einen Beinamen gehabt, auf den sich die Silben \emph{kan} (=
  \emph{beobachten}) und \emph{ji zai} (= \emph{Freiheit})  beziehen . Darum
  kann man den Namen nicht unübersetzt in einen deutschen Text übernehmen: es
  wird hier eben nicht über eine konkrete Einzelperson gesprochen. Vielmehr
  fungiert diese konkrete Person als Typus. Die so verallgemeinerte Aussage
  erlaubt es dem Hörer, sich einbezogen zu fühlen. Um das im Deutschen
  nachzubilden, nutze ich den unbestimmten Artikel und folge ansonsten der
  Deutung von Deshimaru\footcite[vgl.][57 et passim]{Deshimaru1988a}.

  \item[004,011:] \emph{ji} (= \cnbsmi{時}) soll \emph{Zeit} bedeuten und wird
  als Konjunktion zumeist mit \emph{als} oder \emph{während} übersetzt. Im
  deutschen kennen wir zwei Arten der 'zeitlichen' Verbindung zweier Fakten. Die
  eine betont eher die Zufälligkeit, die andere die Ursächlichkeit:
  \emph{\underline{als} ich Zucker aß, bekam ich Kopfschmerzen} meint etwas
  anderes als, \emph{\underline{indem} ich Zucker aß, bekam ich Kopfschmerzen}.
  Im Hannya Shingyo ist eine ursächliche Verknüpfung gemeint: \emph{Das
  Praktizieren der Höchsten Wahrheit führt zu der Erkenntnis, dass \ldots}. Das
  Wort \emph{indem} markiert diese ursächliche Beziehung gut.

  \item[014:] Die 5 Skandhas -- nämlich \emph{Empfindung, Wahrnehmung, Gedanken,
  Handeln und Bewusstsein} -- bilden eine zentrale Achse des Textes:
  zuerst wird ihr Oberbegriff \emph{go on} (= \cnbsmi{五} \cnbsmi{蘊}) genannt.
  Danach wird von jeder einzelnen gesagt, sie sei nicht nur nicht getrennt von
  \emph{kū}, sondern sie sei \emph{kū} (020-039). Schließlich wird auch gesagt,
  dass es sie in \emph{kū} ansich nicht gäbe (050-054), genauso wenig, wie
  entsprechenden Organe (055-060) oder deren Resulte (061-66). Dem liegt ein
  Weltbild zugrunde, das sicher nicht mehr unseres ist. Deshalb ist es
  angemessen, den fremden Begriff 'Skandha' als \emph{Fremdwort} in die
  Übersetzung zu übernehmen. Allerdings: die Pointe des Hannya Shingyos, dass es
  das, was dieses fremde Weltbild beschreibt, in \emph{kū} nicht gäbe, ließe
  sich umstandlos auch mit unserem heutigen physisch / psychischen Weltbild
  formulieren. Man muss sich also die 'veraltete' Sichtweise nicht zu eigen
  machen, um das Hannya Shingyo zu verstehen und seine Aussage zu bejahen. Das
  Hannya Shingyo ist -- so gesehen -- sehr modern.
  
  \item[021:] Es ist üblich, \emph{shiki} mit \emph{Form} zu übersetzen. Das
  wird der rhetorischen Form des Textes aber nicht gerecht: \emph{shiki} ist die
  erste der 5 Skandas. Die anderen 4 werden in den Zeilen [035-038] aufglistet.
  Die Übersetzung von \emph{shiki} muss auch das 1. Skandha schon als Teil einer
  Reihe erscheinen lassen. Dazu eignet sich das Wort \emph{Form} nicht.

  \item[021-033:] Außerdem wird diese ganze Sentenz gelegentlich zu der Aussage
  verknappt, \emph{Form sei Leere und Leere sei Form}. Damit geht eine -- auch
  rhetorisch entscheidende -- Pointe des Originals verloren: Zuerst sagt das
  Hannya Shingyo, \emph{shiki}, die \emph{Erscheinung} sei nicht getrennt von
  \emph{kū}. Dies muss den Hörer verwirren. Denn das normale Verständnis besagt
  doch wohl eher, dass es sich dabei um verschiedene Dinge handelt. Und mit
  diesem Erwartungshorizont spielt der Text. Denn er setzt danach -- sozusagen
  -- 'noch eins drauf': Er verschäft die Situation, in dem er sagt, dass die
  \emph{Erscheinung} und \emph{kū} nicht nur nicht getrennt seien, sondern dass
  das eine realiter auch das andere \emph{sei}. Rhethorisch gesehen präsentiert
  das Hannya Shingyo also zuerst eine 'steile' These, die es im folgenden wird
  erläutern und begründen müssen. Auf jeden Fall -- und das ist der rednerische
  Zweck dieses Vorgehens -- hat es mit dieser Konstruktion die Aufmerksamkeit
  seiner Hörer geweckt. Darum ist es notwendig, diese rhethorische Verschärfung
  auch in der Übersetzung zu erhalten.
  
  \item[022:] Oft wird \emph{i} mit \emph{verschieden} übersetzt. Das wird dem
  Original nicht gerecht. Denn tatsächlich geht es im folgenden Text [044-087],
  in dem \emph{kū} ex negativo definiert wird, um nichts anderes, als die
  Feststellung von Unterschieden.  Die Pointe des Hannya Shingyos ist aber, dass
  \emph{kū} trotz aller Verschiedenartigkeit dennoch -- irgendwie -- mit den 5
  Skandhas zusammenfällt, also trotz aller Verschiedenartigkeit nicht getrennt
  ist von \emph{shiki}. Darum habe ich mich für das Übersetzung \emph{getrennt}
  entschieden; es unterstreicht die intellektuelle Brisanz des Hannya Shingyos.

  \item[023ff:] Es ist üblich, \emph{kū} mit dem Wort \emph{Leere} zu
  übersetzen. Allerdings bringt das Wort \emph{Leere} eigene Konnotationen mit,
  die dem eigentlich Gemeinten entgegenstehen. Das Problem schillernder Begriffe
  kennt pikanterweise sogar das Hannya Shingyo selbst, mehr noch: es spielt
  sogar mit dem Phänomen: Es nimmt nämlich einen dem Gemeinten nahestehenden,
  vermeintlich klaren Begriff \emph{kū} und schärft diesen mittels Aussagen
  darüber, was das Gemeinte alles \emph{nicht} ist. Solch ein Verfahren nennt
  man eine \emph{Ex-Negativo-Definition}. Tatsächlich besteht das Hannya Shingyo
  im Kern aus einer Liste von negierenden Abgrenzungen [044-087]. Aus diesem
  Grund ist es besser, nicht das auch durch die europäische Philosophie
  aufgeheizte Wort \emph{Leere} durch vielfache Wiederholgungen zum Kern zu
  machen, sondern das Original -- also \emph{kū} --  zu verwenden und dessen
  Bedeutung gerade über Negationen klarwerden zu lassen.

  \item[050:] \emph{ze ko} (= \cnbsmi{是} \cnbsmi{故}) steht für \emph{sein
  Ursache}. Während ich später in Zeile [095ff] \emph{ko} konsequent als
  \emph{darum} übersetze, um den repitiven Charakter zu erhalten, wähle ich hier
  - zu Beginn der Deduktion - das stärkere und elegantere \emph{mithin} als
  Übersetzung.
  \item[072:] \emph{nai shi} besagt für sich genommen \emph{dann extrem}. Es
  geht also um eine besonders wichtige Schlussfolgerung. Solch ein sprachliches
  Konstrukt kennen wir auch im Deutschen, nämlich die einleitende Formel:
  \emph{Darum ist/wird/\ldots inbesondere \ldots}.

  \item[071-073:] Die Kombination \emph{gen kai} (= \cnbsmi{眼} \cnbsmi{界}) steht
  wörtlich für \emph{[Auge Welt]}, die Sequenz \emph{i shiki kai} (= \cnbsmi{意}
  \cnbsmi{識} \cnbsmi{界}) hingegen für \emph{Denkvermögen Unterscheiden Welt}.
  Erstere meint also die sichtbare, die erscheinende Welt, letztere die Welt der
  trennenden Vorstellungen und Konzepte. Auch in dieser Gegenüberstellung trifft
  man indirekt die fünf Skandas wieder: Zeile [058] hat schon \emph{gen} (= das
  \emph{Auge}) dem ersten Skandha \emph{shiki} (= \cnbsmi{色} =
  \emph{Erscheinen}) aus Zeile [053] als Organ zugeordnet. Für das fünfte
  Skandha, das Unterscheiden als intellektuelles Tun -- japanisch ebenfalls
  \emph{shiki} genannt -- wird ein anderes Zeichen benutzt als für das erste
  Skandha, nämlich \cnbsmi{識} (Zeile [057]. Und eben dieses zweite \emph{shiki}
  erscheint auch in Zeile [073]. Die rhetorische Konstruktion 'von \emph{gen
  kai} bis \emph{shiki kai} spannt also indirekt erneut den ganzen Bogen über
  alle fünf Skandhas auf.

  \item[074-080:] Die rhetorische Konstruktion \emph{Es gibt in kū nicht XYZ}
  und \emph{Es gibt in kū kein Ende von XYZ} ist besonders aufreizend für
  (europäische) Logiker: Ersteres negiert die Existenz von XYZ; letzteres setzt
  seine Existenz voraus und betont diese durch den impliziten Hinweis auf seine
  Ewigkeit, ausgedrückt durch eine doppelte Verneinung. Damit widersetzt sich
  das Hannya Shingyo der formalen Logik, in dem es dem europäischen Verständnis
  sein \emph{tertium datur} entgegenstellt, nicht ohne diese Logik allerdings
  selbst souverän zu benutzen. Dem Zen entsprechend ist das kein Widerspruch,
  sondern geradezu der Sinn allen Tuns: alle gedanklichen Konstrukte müssen
  aufgehoben werden, wenn \emph{kū} selbst im  Akt der Erleuchtung erfahrbar
  werden soll.

  \item[111-113:] Auch die Sentenz \emph{anokutara sanmyaku sanbodai} ist eine
  zitierende Sanskritnachahmung und meint \emph{höchste, vollkommene
  Erleuchtung}\footcite[vgl.][\nopage Anm. 10]{Scheid2016a}. Welches der Worte
  was bedeutet, habe ich den Quellen bisher nicht entnehmen können. Meine
  Zuordnung ist also willkürlich, folgt aber der Tradition.

  \item[115-121:] Hier findet eine rhetorisch geniale Umdeutung statt, die eine
  große Auswirkung auf den Buddhismus hat: Bisher war der Begriff \emph{han nya
  ha ra mi ta} beschreibend. Er stand für die \emph{die höchste Weisheit, die
  über sich hinausführt}. Jetzt wird der Ausdruck zum Namen des Textes selbst:
  indem er mehrfach als herausgehobenes \emph{Mantra} bezeichnet wird,
  verschiebt sich seine Bedeutung: der Terminus \emph{han nya ha ra mi ta} wird
  zum Namen des Textes. Und in dem diesem dann auch noch eine Wirkung
  zugesprochen wird, wird seine Rezitation zu einem Mittel. Kein Wunder also,
  dass alle Buddhisten diesen Text rezitieren: es steckt in ihm selbst.
 
  \item[119:] Die Phrase \emph{mu jō shu} verwendet wieder einmal eine der im
  \emph{Hannya Shingyo} so gern genutzte 'negative Zuschreibungen':
  \emph{mu} (= \cnbsmi{無}) ist die bekannte Verneinigung; und \emph{shu} (=
  \cnbsmi{咒}) steht für das \emph{Mantra}. Also wird \emph{jō} (= \cnbsmi{上})
  ein Attribut sein, das negiert dem Objekt \emph{Mantra} zugsprochen wird:
  Ein chinesisch-deutsches Internetlexikon sagt, das \cnbsmi{上} auch für
  \emph{von unten nach oben, aufwärts} bzw. \emph{vorwärts gehen}
  steht\footcite[vgl.][\nopage]{babla2016a}. Eine gute Übersetzung würde auch an
  dieser Stelle -- auf der Basis dieser Primärbedeutung -- die bevorzugte
  Methode der Eingrenzung ohne direkte Spezifikation bewahren; sie würde diese
  'ZEN gemäße' Art des 'Denkens' auch hier verdeutlichen. Hier fehlt mir noch
  eine gute Idee für die Umsetzung.

  \item[125-126:] \emph{shin jitsu}(= \cnbsmi{真} \cnbsmi{實}) soll
  \emph{Realität} meinen, und \emph{fu ko} (= \cnbsmi{不} \cnbsmi{虛} ) für
  \emph{nicht/keine Unwahrheit} stehen. Ersteres übersetze ich mit
  \emph{wirklich}, letzteres müsste dann \emph{wahr} heißen. Ich belasse
  letzteres aber bei \emph{nicht unwahr}, um die Neigung des Hannya Shingyos zur
  (doppelten) Verneinung zu erhalten.
\end{description}

% insert the bibliographical data here
\bibliography{bibfiles/hsResourcesDe}

\end{document}
 }

\maketitle
%%-- end(titlepage)
\section{Der Anlass}
 
Wäre es nicht schön, das \emph{Hannya Shingyo} -- mit anderen zusammen -- auch
auswendig vortragen zu können? Immerhin hat die Retization dieses Textes im
(Zen)-Buddhismus eine große Tradition!

Der Weg zum flüssigen Mitsprechen ist holprig: Wie lernt man solch eine sperrige
Folge japanischer Silben, wie einen so erratischen Textblock? Das Lernen dürfte
leichter fallen, wenn eine Struktur erkennbar wäre, etwa in einer
mehrspaltigen, mehrsprachigen, sinnhaft gegliederten Aufbereitung.

Dazu müsste der japanische Text jedoch recht wortgetreu übersetzt sein.
Denn nur so ließe sich die Übersetzung in einer Zeile mit dem übersetzten
Satzteil arrangieren. Würden die deutschen mit den japanisch-chinesischen
Phrasen so auch optisch korrespondieren, erschlössen sich die
Sinneinheiten direkt.

Trotzdem sollte die Übersetzung auch noch elegant sein: Das \emph{Hannya
Shingyo} ist ein Lehrtext, ein Sutra. Zuerst dürfte es mündlich vorgetragen
worden sein, als Ansprache an die Schüler. Mithin wird man darin -- ganz
sprachunabhängig -- auch rhetorische Elemente finden: Einen Interesse weckenden
\emph{Einstieg} etwa. Oder eine aufrüttelnde \emph{Kernthese}, die allmähliche
\emph{Entfaltung} ihrer Feinheiten, und die sich daran anschließende Begründung
der \emph{Konsequenzen}. Und natürlich einen einprägsamen \emph{Schluss}. Wäre es
nicht schön, wenn ein \emph{Hannya-Shingyo-Lerntext} auch das noch erkennen
ließe?

Gleichwohl müsste die Übertragung immer genau bleiben, von der Bedeutung und der
syntaktischen Struktur her\footnote{Doris Wolter hat dankenswerterweise
verschiedene Übersetzungen ins Deutsche zusammengetragen.
(\cite[vgl.][\nopage]{Wolter2010a}) Vergleicht man diese Versionen, offenbaren
sich erhebliche Unterschiede. Insbesondere das letzte Drittel des Hannya
Shingyos scheint dabei zu besonders 'poetischen' Übertragungen einzuladen.
Angesichts der existentiellen philosophischen Dimension des Zen-Buddhismus und
des Anspruchs auf letztgültige Wahrheiten im \emph{Hannya Shingyo} selbst ist
das schlicht unzufriedenstellend.}. Sie sollte so wenig als möglich
interpretieren.

Es gibt wunderbare Übersetzungen: z.B. die von
Deshimaru\footcite[vgl.][]{Deshimaru1988a}, die eher ein philosophischer
Hintergrundbericht sein will, als eine pure Übersetzung. Oder die universitär
abgesicherte, elegante Übertragung von Scheid\footcite[vgl.][]{Scheid2016a}.
Oder die wortgetreue von Boeck\footcite[vgl.][]{Boeck2016a}.

Nur liefern sie alle leider keinen mehrsprachigen, sinnhaft gegliederten
Lerntext, der bei aller Worttreue auch noch die elegante Rhetorik des Originals
erahnen ließe. Wie wäre es also mit folgender Variante?

\newpage
\section{Der Text} 

\sffamily

\begin{center}
\begin{tabular}{r|rl|rl|rl}
~ & \multicolumn{6}{l}{\textsc{Der Titel:}}\\
\hline
{\tiny\texttt{001}}&
  \multicolumn{2}{l|}{\cnbsmi{摩}  \cnbsmi{訶} \cnbsmi{般} \cnbsmi{若}} &
  \multicolumn{2}{l|}{\textbf{ma kā}  \textbf{han nya}} &
  \textrm{\emph{[Die]}}& \textrm{maha prajñā \emph{= höchste Weisheit}}\\
{\tiny\texttt{002}}&
  ~ & \cnbsmi{波} \cnbsmi{羅} \cnbsmi{蜜} \cnbsmi{多} & 
  ~ & \textbf{ha ra mi tā} & 
  {\tiny \textrm{($\rightarrow$)}} & 
    \textrm{pāramitā\emph{, die über sich hinausführt,}}\\
{\tiny\texttt{003}}& 
  ~ & \cnbsmi{心} \cnbsmi{經} &
  ~ & \textbf{shin gyō} & 
  \textrm{\emph{[als das]}} & \textrm{essentielle Sutra \emph{[schlechthin]}} \\
\hline
~ & \multicolumn{6}{l}{\textsc{Das Manifest:}}\\
\hline
{\tiny\texttt{004}}& 
~ & ~  & ~ & ~ &
  \textrm{Indem} {\tiny \textrm{($\rightarrow$)}} & \textrm{\emph{[ein der]}} \\
{\tiny\texttt{005}}&
  ~ & \cnbsmi{觀} \cnbsmi{自} \cnbsmi{在} & 
  ~ & \textbf{kan ji zai} & 
  {\tiny \textrm{($\rightarrow$)}} &
    \textrm{freien Sicht \emph{[zugewandter]}} \\
{\tiny\texttt{006}}&
  ~ & \cnbsmi{菩} \cnbsmi{薩} \cnbsmi{。}& 
  ~ & \textbf{bo} \textbf{sa}.& 
  ~ & \textrm{\emph{[lebender Buddha, ein]} Bodhisattva} \\  
{\tiny\texttt{007}}& 
  ~ & \cnbsmi{行} \cnbsmi{深} &
  ~ & \textbf{gyō} \textbf{jin} & 
  ~ & \textrm{tief \emph{[und gründlich]} praktizierend} \\  
{\tiny\texttt{008}}& 
  ~ & \cnbsmi{般} \cnbsmi{若} & 
  ~ & \textbf{han nya} & 
  ~ & \textrm{\emph{[die]} Prajñā \emph{, Weisheit}} \\  
{\tiny\texttt{009}} &
  ~ & \cnbsmi{波} \cnbsmi{羅} \cnbsmi{蜜} \cnbsmi{多}& 
  ~ & \textbf{ha ra mi ta} & 
  ~ & \textrm{Pāramitā \emph{, die über sich hinausführt,}} \\  
{\tiny\texttt{010}}&
  ~ & ~  & ~ & ~ &  ~ & \textrm{\emph{[lebt]}} \\
{\tiny\texttt{011}}&
  \cnbsmi{時}&\cnbsmi{。} &
  \textbf{ji}. & ~ &
  {\tiny \textrm{($\rightarrow$)}} & ~ \\
{\tiny\texttt{012}}& 
  ~ & ~ & ~ & ~ & ~ & \textrm{\emph{[kommt es bei ihm zum]}} \\
{\tiny\texttt{013}}& 
  ~ & \cnbsmi{照} \cnbsmi{見} &
  ~ & \textbf{shō ken} &
  ~ & \textrm{erleuchteten Sehen \emph{[, dass die]}} \\  
{\tiny\texttt{014}}& 
  ~ & \cnbsmi{五} \cnbsmi{蘊} & 
  ~ & \textbf{go on} & 
  {\tiny \textrm{($\rightarrow$)}} & \textrm{5 Skandas}  \\
{\tiny\texttt{015}}&
  ~ & \cnbsmi{皆} \cnbsmi {空} \cnbsmi{。} &
  ~ & \textbf{kai kū}. & 
  ~ & \textrm{alle leer \emph{[sind]}} \\
{\tiny\texttt{016}}&
  \cnbsmi{度} & ~ &
  \textbf{do} & ~ &
  \textrm{\emph{[und]} so} & ~ \\
{\tiny\texttt{017}}&
  ~ & \cnbsmi{一} \cnbsmi{切} &
  ~ & \textbf{is sai} &
  ~ & \textrm{entfernt \emph{[er]}} \\
{\tiny\texttt{018}}&
  ~ & \cnbsmi{苦} \cnbsmi{厄} \cnbsmi{。} & ~ &
  \textbf{ku yaku}. & 
  ~ & \textrm{Leiden \emph{[und]} Unheil.} \\
\hline
  ~ & \multicolumn{6}{l}{\textsc{Die Kernthese:}}\\
\hline
{\tiny\texttt{019}}&
  \multicolumn{2}{l|}{\cnbsmi{舍} \cnbsmi{利} \cnbsmi{子}\cnbsmi{。}}  &
  \multicolumn{2}{l|}{\textbf{sha ri shi}.} & ~ &
  \textrm{Shariputra!}\\
\hline  
{\tiny\texttt{020}}&
  ~ & ~ & ~ & ~ & 
  \multicolumn{2}{l}{\textrm{\emph{[Die 1. der 5 Skandas, nämlich die]}}} \\
{\tiny\texttt{021}}&
  ~ & \cnbsmi{色} & 
  {\tiny \textrm{($\rightarrow$)}} & \textbf{shiki} &
  ~ & \textrm{Erscheinung} \\  
{\tiny\texttt{022}}&
  \cnbsmi{不} & \cnbsmi{異} & 
  \textbf{fu} & \textbf{i} &
  \textrm{\emph{[ist]} nicht} & \textrm{getrennt \emph{[von]}} \\  
{\tiny\texttt{023}}&
  ~ & \cnbsmi{空} \cnbsmi{。} &
  {\tiny \textrm{($\rightarrow$)}} & \textbf{kū}.  &
  {\tiny \textrm{($\rightarrow$)}} & \textrm{kū, \emph{[der Leere]}} \\
\hdashline
{\tiny\texttt{024}}&
  ~ & \cnbsmi{空} &
  ~ & \textbf{kū} & 
  \textrm{\emph{[und]}} & \textrm{kū, \emph{[die Leere]}} \\
{\tiny\texttt{025}}&
  \cnbsmi{不} & \cnbsmi{異} &
  \textbf{fu} & \textbf{i} &
  \textrm{\emph{[ist]} nicht} & \textrm{getrennt \emph{[von]}} \\  
{\tiny\texttt{026}}&
  ~ & \cnbsmi{色} \cnbsmi{。} &
  ~ & \textbf{shiki}. &
  ~ & \textrm{\emph{[der]} Erscheinung.} \\
\hline
{\tiny\texttt{027}}&
  ~ & ~ & ~ & ~ & \multicolumn{2}{l}{\textrm{~\emph{Ja, mehr noch:}}}  \\  
{\tiny\texttt{028}}&
  ~ & \cnbsmi{色} &
  ~ & \textbf{shiki} & 
  ~ & \textrm{\emph{[Die]} Erscheinung} \\  
{\tiny\texttt{029}}&
  \cnbsmi{即} & \cnbsmi{是} & 
  \textbf{soku} & \textbf{ze} &
  \textrm{ist} & \textrm{eigentlich} \\  
{\tiny\texttt{030}}&
  ~ & \cnbsmi{空} \cnbsmi{。} &
  ~ & \textbf{kū}. &
  ~ & \textrm{kū, \emph{[die Leere]}} \\
 \hdashline
 {\tiny\texttt{031}}&
  ~ & \cnbsmi{空} &
  ~ & \textbf{kū} & 
  \textrm{\emph{[und]}} & \textrm{kū, \emph{[die Leere]}} \\
{\tiny\texttt{032}}&
  \cnbsmi{即} & \cnbsmi{是} & 
  \textbf{soku} & \textbf{ze} &
  \textrm{ist} & \textrm{eigentlich} \\  
{\tiny\texttt{033}}&
 ~ & \cnbsmi{色} \cnbsmi{。} &
 ~ & \textbf{shiki}. &~
 ~ & \textrm{\emph{[die]} Erscheinung.} \\
 \hline
 {\tiny\texttt{034}}&
    ~ & ~ & ~ & ~ & \multicolumn{2}{l}{
    \textrm{\emph{[Und bei den anderen 4 Skandas, also beim]}}}\\
 {\tiny\texttt{035}}&
  ~ & \cnbsmi{受} &
  ~ & \textbf{ju} &
  ~ & \textrm{Empfinden,} \\
{\tiny\texttt{036}}&
  ~ & \cnbsmi{想} &
  ~ & \textbf{sō} &
  ~ & \textrm{Wahrnehmen,} \\
 {\tiny\texttt{037}}&
  ~ & \cnbsmi{行} &
  ~ & \textbf{gyō} &
  ~ & \textrm{Wollen \emph{[und]}} \\
 {\tiny\texttt{038}}&
  ~ & \cnbsmi{識} &
  ~ & \textbf{shiki}. &
  ~ & \textrm{Unterscheiden}, \\
{\tiny\texttt{039}}&
  \cnbsmi{亦} & \cnbsmi{復} \cnbsmi{如} \cnbsmi{是} \cnbsmi{。} &
  \textbf{yaku} & \textbf{bu nyo ze}. &
  \textrm{auch \emph{[da]}} & \textrm{ist \emph{[es]} wieder gleich}. \\
\hline
  ~ & \multicolumn{6}{l}{\textsc{Die ex negativo Definition von \textrm{kū}:}}\\
\hline
{\tiny\texttt{040}} &
  \multicolumn{2}{l|}{\cnbsmi{舍} \cnbsmi{利} \cnbsmi{子} \cnbsmi{。}} &
  \multicolumn{2}{l|}{\textbf{sha ri shi}.} &
  ~ & \textrm{Shariputra!}\\
\hline
{\tiny\texttt{041}}&
  \cnbsmi{是} & \cnbsmi{諸} &
  \textbf{ze} & \textbf{sho} &
  \textrm{\emph{[Es]} ist} & \textrm{alles} \\
{\tiny\texttt{042}}&
  ~ & \cnbsmi{法} &
  ~ & \textbf{hō} &
  ~ & \textrm{Seiende} \\
{\tiny\texttt{043}}&
  ~ & \cnbsmi{空} \cnbsmi{相} \cnbsmi{。} &
  ~ & \textbf{kū sō}. &
  ~ & \textrm{\emph{[ein]} Aspekt \emph{[von]} kū}: \\
\hdashline
{\tiny\texttt{044}}&
  \cnbsmi{不} & \cnbsmi{生} &
  \textbf{fu} & \textbf{shō} & 
  \textrm{nicht} & \textrm{geboren \emph{[bzw.]} geschaffen} \\
{\tiny\texttt{045}}&
  \cnbsmi{不} & \cnbsmi{滅} \cnbsmi{。} &
  \textbf{fu} & \textbf{metsu}. &
  \textrm{nicht} & \textrm{gestorben \emph{[bzw.]} ausgelöscht},\\
\hdashline
{\tiny\texttt{046}}&
  \cnbsmi{不} & \cnbsmi{垢} &
  \textbf{fu} & \textbf{ku} &
  \textrm{nicht} & \textrm{befleckt} \\
{\tiny\texttt{047}}&
  \cnbsmi{不} & \cnbsmi{淨} &
  \textbf{fu} & \textbf{jō}. &
  \textrm{nicht} & \textrm{rein}, \\
\hdashline
\end{tabular}

\begin{tabular}{r|rl|rl|rl}
\hdashline
{\tiny\texttt{048}}&
  \cnbsmi{不} & \cnbsmi{增} &
  \textbf{fu} & \textbf{zō} &
  \textrm{nicht} & \textrm{zunehmend} \\
{\tiny\texttt{049}}&
  \cnbsmi{不} & \cnbsmi{減} &
  \textbf{fu} & \textbf{gen}. &
  \textrm{nicht} & \textrm{abnehmend}. \\
\hline
{\tiny\texttt{050}}&
  \cnbsmi{是} & \cnbsmi{故} &
  \textbf{ze} & \textbf{ko} &
  \textrm{Mithin} & {\tiny ($\rightarrow$)} \textrm{\emph{[gibt es]}}\\
{\tiny\texttt{051}}&
  ~ & \cnbsmi{空} \cnbsmi{中} \cnbsmi{。} &
  ~ & \textbf{kū chū}. &
  ~ & \textrm{in kū} \\
{\tiny\texttt{052}}&
  ~ & ~  & ~ & ~ &  ~ & \textrm{\emph{[keines der 5 Skandhas, also]}} \\
{\tiny\texttt{053}}&
  \cnbsmi{無} & \cnbsmi{色} \cnbsmi{。} &
  \textbf{mu} & \textbf{shiki} &
  \textrm{kein} & \textrm{Erscheinen}, \\
{\tiny\texttt{054}}&
  \cnbsmi{無} & \cnbsmi{受} &
  \textbf{mu} & \textbf{ju} & 
  \textrm{kein} & \textrm{Empfinden,} \\
{\tiny\texttt{055}}&
  ~ & \cnbsmi{想} &
  ~ & \textbf{sō} &
  ~ & \textrm{Wahrnehmen,} \\
{\tiny\texttt{056}}&
  ~ & \cnbsmi{行} &
  ~ & \textbf{gyō} &
  ~ & \textrm{Wollen \emph{[oder]}} \\
{\tiny\texttt{057}}&
  ~ & \cnbsmi{識} \cnbsmi{。} &
  ~ & \textbf{shiki}. & 
  ~ & \textrm{Unterscheiden}, \\
\hdashline
 {\tiny\texttt{058}}&
  \cnbsmi{無} & \cnbsmi{眼} &
  \textbf{mu} & \textbf{gen} &
  \textrm{keine} & \textrm{Augen,} \\
{\tiny\texttt{059}}&
  ~ & \cnbsmi{耳} &
  ~ & \textbf{ni} &
  ~ & \textrm{Ohren,} \\
{\tiny\texttt{060}}&
  ~ & \cnbsmi{鼻} &
  ~ & \textbf{bi} &
  ~ & \textrm{Nase,} \\
{\tiny\texttt{061}}&
  ~ & \cnbsmi{舌} &
  ~ & \textbf{ze} &
  ~ & \textrm{Zunge,} \\
{\tiny\texttt{062}}&
  ~ & \cnbsmi{身} &
  ~ & \textbf{shin} &
  \textrm{\emph{[keinen]}} & \textrm{Tastsinn \emph{[und]}} \\
{\tiny\texttt{063}}&
  ~ & \cnbsmi{意} \cnbsmi{。} &
  ~ & \textbf{i}. & 
  \textrm{\emph{[kein]}} & \textrm{Denkvermögen}. \\
\hdashline
{\tiny\texttt{064}}&
   \cnbsmi{無} & \cnbsmi{色} &
   \textbf{mu} & \textbf{shiki} &
   \textrm{keine} & \textrm{Farbe,} \\
{\tiny\texttt{065}}&
   ~ & \cnbsmi{聲} &
   ~ & \textbf{shō} &
   \textrm{\emph{[keinen]}} & \textrm{Klang,} \\
{\tiny\texttt{066}}&
   ~ & \cnbsmi{香} &
   ~ & \textbf{kō} &
   ~ & \textrm{Geruch,} \\
{\tiny\texttt{067}}&
  ~ & \cnbsmi{味} &
  ~ & \textbf{mi} &
  ~ & \textrm{Geschmack,} \\
{\tiny\texttt{068}}&
  ~ & \cnbsmi{觸} &
  ~ & \textbf{soku} & 
  \textrm{\emph{[keine]}} & \textrm{Berührung \emph{[und]}} \\
{\tiny\texttt{069}}&
  ~ & \cnbsmi{法} \cnbsmi{。} &
  ~ & \textbf{hō}.&
  \textrm{\emph{[keinen]}} & \textrm{Gedanken}; \\
\hline
{\tiny\texttt{070}}&
  ~ & ~ & ~ & ~ & ~ & \textrm{\emph{[Also gibt es in kū]}} \\
\hdashline
{\tiny\texttt{071}}&
  \cnbsmi{無} & \cnbsmi{眼} \cnbsmi{界} \cnbsmi{。} &
  \textbf{mu} & \textbf{gen kai} &
  \textrm{nicht} & \textrm{die sichtbare Welt} {\tiny ($\rightarrow$)} \\
{\tiny\texttt{072}}&
  \cnbsmi{乃}\cnbsmi{至} & ~ & 
  \textbf{nai shi} & ~ &
  \textrm{\emph{[und]} {\tiny ($\rightarrow$)}} & 
   \textrm{darum insbesondere [auch]}\\
{\tiny\texttt{073}}&
  \cnbsmi{無} & \cnbsmi{意} \cnbsmi{識} \cnbsmi{界} \cnbsmi{。}&
  \textbf{mu} & \textbf{i shiki kai}.&
  \textrm{nicht} & 
    \textrm{die Welt der Vorstellungen {\tiny ($\rightarrow$)}} \\
\hdashline
{\tiny\texttt{074}}&
  \cnbsmi{無} & \cnbsmi{無} \cnbsmi{明} \cnbsmi{。} &
  \textbf{mu} & \textbf{mu myō} &
  \textrm{kein} & \textrm{Nicht-Wissen \emph{[und]}} \\
{\tiny\texttt{075}}&
  \cnbsmi{亦} & ~ & 
  \textbf{yaku} & ~ &
  \textrm{auch} & ~ \\  
{\tiny\texttt{076}}&
  \cnbsmi{無} & \cnbsmi{無} \cnbsmi{明} \cnbsmi{盡} \cnbsmi{。} &
  \textbf{mu} & \textbf{mu myō jin.} &
  \textrm{kein} & \textrm{Ende vom Nicht-Wissen} {\tiny ($\rightarrow$)} \\
\hdashline
{\tiny\texttt{077}}&
  \cnbsmi{乃}\cnbsmi{至} & ~ & 
  \textbf{nai shi} & ~ &
  \textrm{\emph{[und]} {\tiny ($\rightarrow$)}} & 
   \textrm{darum insbesondere \emph{[auch]}}\\
\hdashline
{\tiny\texttt{078}}&
  \cnbsmi{無} & \cnbsmi{老} \cnbsmi{死} \cnbsmi{。} &
  \textbf{mu} & \textbf{rō shi} &
  \textrm{kein} & \textrm{Altern und Tod \emph{[und]}} \\
{\tiny\texttt{079}}&
  \cnbsmi{亦} & ~ &
  \textbf{yaku} & ~ &
  \textrm{auch} & ~ \\
{\tiny\texttt{080}}&
  \cnbsmi{無} & \cnbsmi{老} \cnbsmi{死} \cnbsmi{盡} \cnbsmi{。} &
  \textbf{mu} &
  \textbf{rō shi jin}. &
  \textrm{kein} & \textrm{Ende von Altern und Tod {\tiny ($\rightarrow$)}}\\
\hline
{\tiny\texttt{081}}&
  \cnbsmi{無} & \cnbsmi{苦} & 
  \textbf{mu} & \textbf{ku} &
  \textrm{kein} & \textrm{Leiden,} \\  
{\tiny\texttt{082}}&
  ~ & \cnbsmi{集} &
  ~ & \textbf{shū} & 
  ~ & \textrm{Anhäufen,} \\  
{\tiny\texttt{083}}&
  ~ & \cnbsmi{滅} &
  ~ & \textbf{metsu} & 
  ~ & \textrm{Verlöschen \emph{[und]}} \\  
{\tiny\texttt{084}}&
  ~ & \cnbsmi{道} \cnbsmi{。} & 
  ~ & \textbf{dō}. &
  \textrm{\emph{[keinen]}} & \textrm{Weg,} \\
\hdashline 
{\tiny\texttt{085}}&
  \cnbsmi{無} & \cnbsmi{智} &
  \textbf{mu} & \textbf{chi} &
  \textrm{keine} & \textrm{Erkenntnis \emph{[und]}} \\  
{\tiny\texttt{086}}&
  \cnbsmi{亦} & ~ &
  \textbf{yaku} & ~ &
  \textrm{auch} & ~ \\
{\tiny\texttt{087}}&
  \cnbsmi{無} & \cnbsmi{得} \cnbsmi{。} &
  \textbf{mu} & \textbf{toku}. &
  \textrm{keinen} & \textrm{Gewinn,} \\  
{\tiny\texttt{088}}&
  \cnbsmi{以} & ~ &
  \textbf{I} & ~ &
  \textrm{weil} &  \textrm{\emph{[kū]}} \\  
{\tiny\texttt{089}}&
  \cnbsmi{無} & \cnbsmi{所} \cnbsmi{得}  &
  \textbf{mu} & \textbf{sho tok}u  &
  \textrm{kein} & \textrm{Ort \emph{[des]} Gewinnens \emph{[ist]}.} \\
\hline
  ~ & \multicolumn{6}{l}{\textsc{Die praktische Konsequenz:}}\\
\hline
{\tiny\texttt{090}}&
  \cnbsmi{故}\cnbsmi{。} & 
    \cnbsmi{菩} \cnbsmi{提} \cnbsmi{薩} \cnbsmi{捶}\cnbsmi{。}  &
  \textbf{ko}. & \textbf{bo dai sat ta.} &
  \textrm{Darum} & 
    \textrm{\emph{[gilt:] [Ein]} Bodhisattva \emph{[zu sein,]}}\\
{\tiny\texttt{091}}&
  ~ & \cnbsmi{依} &
  ~ & \textbf{e} & 
  ~ & \textrm{bedingt \emph{[die]}} \\
\newline
{\tiny\texttt{092}}&
  ~ & \cnbsmi{般} \cnbsmi{若}  &
  ~ & \textbf{han nya} &
  ~ & \textrm{Prajñā \emph{Weisheit}} \\  
{\tiny\texttt{093}}&
  ~ & \cnbsmi{波} \cnbsmi{羅} \cnbsmi{蜜} \cnbsmi{多} &
  ~ & \textbf{ha ra mi ta} &
  ~ & \textrm{Pāramitā\emph{, die über sich hinausführt.}} \\
\hdashline
{\tiny\texttt{095}}&
  \cnbsmi{故}\cnbsmi{。} & 
    \cnbsmi{心} \cnbsmi{無} \cnbsmi{罫} \cnbsmi{礙} \cnbsmi{。} &
  \textbf{ko.} & \textbf{shin mu kei ge} &
  \textrm{Darum} & 
    \textrm{\emph{[wird sein]} Geist nicht behindert.} \\
\hdashline
{\tiny\texttt{096}}&
  ~ & \cnbsmi{無} \cnbsmi{罫} \cnbsmi{礙} & 
  ~ & \textbf{mu kei ge} & 
 \multicolumn{2}{l}
  {\textrm{\emph{[Und da der]} nicht behindert \emph{[wird]},}}\\
{\tiny\texttt{097}}&
  \cnbsmi{故}\cnbsmi{。} & \cnbsmi{無} \cnbsmi{有} & 
  \textbf{ko.} & \textbf{mu u }& 
  \textrm{darum} & \textrm{hat \emph{[der Bodhisattva]} keine} \\
{\tiny\texttt{098}}&
  ~ & \cnbsmi{恐} \cnbsmi{怖}\cnbsmi{。} &
  ~ & \textbf{ku fu} &
  ~ & \textrm{Furcht}. \\ 
\hline
\end{tabular}

\begin{tabular}{r|rl|rl|rl}
\hdashline
{\tiny\texttt{099}}&
  ~ & \cnbsmi{遠} \cnbsmi{離} &
  ~ & \textbf{on ri} &
  \textrm{\emph{[Das]}} & \textrm{übersteigend\emph{[, was er sich]}} \\      
{\tiny\texttt{100}}&
  ~ & \cnbsmi{一} \cnbsmi{切} &
  ~ & \textbf{is sai} & 
  ~ & \textrm{entfernt \emph{[hat -- nämlich]} }\\
{\tiny\texttt{101}}&
  ~ & \cnbsmi{顛} \cnbsmi{倒} &
  ~ & \textbf{ten dō} &
  ~ & \textrm{Täuschungen \emph{[und]}} \\      
{\tiny\texttt{102}}&
  ~ & \cnbsmi{夢} \cnbsmi{想} \cnbsmi{。} &
  ~ & \textbf{mu sō}. &
  ~  & \textrm{Illusionen \emph{[--]}} \\      
{\tiny\texttt{103}}&
  ~ & \cnbsmi{究} \cnbsmi{竟} &
  ~ & \textbf{ku gyō} &
  ~ & \textrm{erreicht \emph{[er]} schließlich } \\      
{\tiny\texttt{104}}&
  ~ & \cnbsmi{涅} \cnbsmi{槃} \cnbsmi{。}&
  ~ & \textbf{ne han}. &
  ~  & \textrm{\emph{[das]} Nirvana.} \\      
\hline   
{\tiny\texttt{105}}&
  ~ & \cnbsmi{三} \cnbsmi{世} &
  ~ & \textbf{san ze} &
  \textrm{\emph{[Zudem]}} & \textrm{\emph{[gilt seit]} drei Zeitaltern} \\      
{\tiny\texttt{106}}&
  ~ & \cnbsmi{諸} \cnbsmi{佛} \cnbsmi{。} &
  ~ & \textbf{sho butsu} &
  \textrm{\emph{für}} & 
    \textrm{alle Buddhas: \emph{ihre Buddhaschaft}} \\     
 {\tiny\texttt{107}}&
  ~ & \cnbsmi{依} &
  ~ & \textbf{e} &
  ~ & \textrm{bedingt \emph{[die]}} \\  
{\tiny\texttt{108}}&
  ~ & \cnbsmi{般} \cnbsmi{若}  &
  ~ & \textbf{han nya} &
  ~ & \textrm{Prajñā \emph{(Weisheit)}} \\  
{\tiny\texttt{109}}&
  ~ & \cnbsmi{波} \cnbsmi{羅} \cnbsmi{蜜} \cnbsmi{多} &
  ~ & \textbf{ha ra mi ta} &
  ~ & \textrm{Pāramitā\emph{, die über sich hinausführt.}} \\
\hline  
{\tiny\texttt{110}}&
  \cnbsmi{故}\cnbsmi{。} & \cnbsmi{得} &
  \textbf{ko}. & \textbf{toku} &
  \textrm{Darum} & \textrm{gewinnen sie die} \\  
{\tiny\texttt{111}}&
  ~ & \cnbsmi{阿} \cnbsmi{耨} \cnbsmi{多} \cnbsmi{羅} & 
  ~ & \textbf{a noku ta ra} &
  {\tiny ($\rightarrow$)} & \textrm{anuttara} \textrm{\emph{höchste}} \\
{\tiny\texttt{112}}&
  ~ & \cnbsmi{三} \cnbsmi{藐} &
  ~ & \textbf{san myaku} &
  {\tiny ($\rightarrow$)} & \textrm{samyak} \textrm{\emph{vollkommene}} \\      
{\tiny\texttt{113}}&
  ~ & \cnbsmi{三} \cnbsmi{菩} \cnbsmi{提} \cnbsmi{。} &
  ~ & \textbf{san bo dai}. &
  {\tiny ($\rightarrow$)} & \textrm{sambodhi} \textrm{\emph{Erleuchtung}} \\ 
 \hline 
{\tiny\texttt{114}}&
  \cnbsmi{故} & \cnbsmi{知} &
  \textbf{ko} & \textbf{chi} &
  \textrm{Darum} &  \textrm{wisse \emph{[nun Du Deinerseits:]}} \\  
{\tiny\texttt{115}}&
  ~ & \cnbsmi{般} \cnbsmi{若} &
  ~ & \textbf{han nya} &
  \textrm{\emph{[Das]}} & \textrm{Prajñā} {\tiny ($\rightarrow$)} \\  
{\tiny\texttt{116}}&
  ~ & \cnbsmi{波} \cnbsmi{羅} \cnbsmi{蜜} \cnbsmi{多}&
  ~ & \textbf{ha ra mi ta}.&
  ~ & \textrm{Pāramitā} {\tiny ($\rightarrow$)} \\ 
{\tiny\texttt{117}}&
  \cnbsmi{是} & \cnbsmi{大} \cnbsmi{神} \cnbsmi{咒} \cnbsmi{。} &
  \textbf{ze} & \textbf{dai jin shu}. &
  \textrm{ist} & \textrm{\emph{[ein]} großes wunderbares Mantra}; \\  
{\tiny\texttt{118}}&
  \cnbsmi{是} & \cnbsmi{大} \cnbsmi{明} \cnbsmi{咒} &
  \textbf{ze} & \textbf{dai myō shu}. &
  \textrm{\emph{[es]} ist} & \textrm{\emph{[ein]} großes leuchtendes Mantra}, \\ 
{\tiny\texttt{119}}&
  \cnbsmi{是} & \cnbsmi{無} \cnbsmi{上} \cnbsmi{咒} \cnbsmi{。} &
  \textbf{ze} & \textbf{mu jō shu}. &
  \textrm{\emph{[es]} ist} & 
    \textrm{\emph{[das]} {\tiny ($\rightarrow$)} höchste Mantra} \\
{\tiny\texttt{120}}&
  \cnbsmi{是} & \cnbsmi{無} \cnbsmi{等} \cnbsmi{等} \cnbsmi{咒} \cnbsmi{。}  &
  \textbf{ze} &  \textbf{mu tō dō shu}.  &
  \textrm{\emph{[es]} ist} & \textrm{\emph{[das]} nicht übersteigbare Mantra} \\  
{\tiny\texttt{121}}&
  ~ & \cnbsmi{能} & 
  ~ & \textbf{nō} &  
  ~ \textrm{\emph{[es]}} & \textrm{dient \emph{[dem]}} \\
{\tiny\texttt{122}}&
  ~ & \cnbsmi{除} \cnbsmi{一} \cnbsmi{切} & 
  ~ & \textbf{jo is sai} & 
  ~ & \textrm{Beseitigen \emph{[und]} Abschneiden} \\
{\tiny\texttt{123}}&
  ~ & \cnbsmi{苦} \cnbsmi{。} & 
  ~ & \textbf{ku}. & 
 ~ & \textrm{\emph{[von]} Leiden.} \\
\hline 
  ~ & \multicolumn{6}{l}{\textsc{Das Fazit \ldots}}\\
\hline
{\tiny\texttt{124}}&
  ~ & ~ & ~ & ~ &  & \textrm{\emph{[Und weil dies]}} \\      
{\tiny\texttt{125}}&
  ~ & \cnbsmi{真} \cnbsmi{實} &
  ~ & \textbf{shin jitsu}  &
  ~ & \textrm{wirklich \emph{[und]} {\tiny ($\rightarrow$)}} \\  
{\tiny\texttt{126}}&
  ~ & \cnbsmi{不} \cnbsmi{虛} \cnbsmi{。}&
  ~ & \textbf{fu ko} &
  ~ & \textrm{nicht unwahr \emph{[ist,]}} \\
{\tiny\texttt{127}}&
  \cnbsmi{故} & ~ &
  \textbf{ko} & ~ &
  \textrm{darum} & \textrm{\emph{[wird die]} }\\  
{\tiny\texttt{128}}&
  ~ & \cnbsmi{說} & 
  ~ & \textbf{setsu} &
  ~ & \textrm{Bedeutung \emph{[der]}} \\  
 {\tiny\texttt{129}}&
  ~ & \cnbsmi{般} \cnbsmi{若} &
  ~ & \textbf{han nya} &
  ~ & \textrm{Prajñā} \\  
{\tiny\texttt{130}}&
  ~ & \cnbsmi{波} \cnbsmi{羅} \cnbsmi{蜜} \cnbsmi{多}&
  ~ & \textbf{ha ra mi ta} &
  ~ & \textrm{Pāramitā} \\ 
{\tiny\texttt{131}}&
  ~ & \cnbsmi{咒} &
  ~ & \textbf{shu}. & 
  \textrm{\emph{[als]}} & \textrm{Mantra} \\
{\tiny\texttt{132}}&
  \cnbsmi{即} & ~ & 
  \textbf{soku} & ~ &
  \multicolumn{2}{l}{\textrm{eigentlich \emph{[auch durch die]}}} \\  
{\tiny\texttt{133}}&
  ~ & \cnbsmi{說} &
  ~ & \textbf{setsu} & 
  ~ & \textrm{Bedeutung \emph{[des nun}}\\
{\tiny\texttt{134}}&
  ~ & \cnbsmi{咒} &
  ~ & \textbf{shu}  & 
  ~ & \textrm{\emph{folgenden]} Mantras} \\
{\tiny\texttt{135}}&
  ~ & \cnbsmi{曰} &
  ~ & \textbf{watsu} & 
  ~ & \textrm{ausgesagt:} \\ 
\hline
\end{tabular}

\begin{tabular}{r|rl|rl|rl}
\hline
  ~ & \multicolumn{6}{l}{\textsc{\ldots in Form eines Mantras:}}\\
\hline
{\tiny\texttt{136}}&
  ~ & ~ & ~ & ~ & ~ & \textrm{\emph{Lasst uns}} \\
{\tiny\texttt{137}}&
  ~ & \cnbsmi{羯} \cnbsmi{諦}&
  ~ & \textbf{gya tei} &
  ~ & \textrm{hinübergehen,} \\  
{\tiny\texttt{138}}&
  ~ & \cnbsmi{羯} \cnbsmi{諦}&
  ~ & \textbf{gya tei} &
  ~ & \textrm{hinübergehen,} \\  
{\tiny\texttt{139}}&
  \cnbsmi{波} \cnbsmi{羅} & \cnbsmi{羯} \cnbsmi{諦}&
  \textbf{ha ra} & \textbf{gya tei} &
  \multicolumn{2}{l}{\textrm{mit anderen hinübergehen,}} \\
{\tiny\texttt{140}}&
  \cnbsmi{波} \cnbsmi{羅} \cnbsmi{僧} & \cnbsmi{羯} \cnbsmi{諦}&
  \textbf{ha ra sō} & \textbf{gya tei} &
  \multicolumn{2}{l}{\textrm{mit anderen vollständig hinübergehen,}} \\
 \hdashline
 {\tiny\texttt{141}}&
  \cnbsmi{菩} \cnbsmi{提} \cnbsmi{薩} & \cnbsmi{婆} \cnbsmi{訶}&
  \textbf{bo ji} & \textbf{so wa ka} &
  \textrm{\emph{auf dem}} & \textrm{Weg \emph{zur} Vollendung.} \\
 \hline
   ~ & \multicolumn{6}{l}{\textsc{Punkt}}\\
 \hline
 {\tiny\texttt{142}}& 
  ~ & \cnbsmi{般} \cnbsmi{若} &
  ~ & \textbf{han nya} &
  \textrm{\emph{[So die]}}& \textrm{prajñā \emph{, Weisheit}}\\
{\tiny\texttt{143}}& 
  ~ & \cnbsmi{心} \cnbsmi{經} &
  ~ & \textbf{shin gyō}. & 
  \textrm{\emph{[als]}} & \textrm{essentielles Sutra} \\
 \hline 
%\end{longtable}
\end{tabular}
\end{center}
\rmfamily

\section{Die Gestaltung} 

In der linken Spalte meiner Lernversion des Hannya Shingyos steht der
chinesische Text. Er folgt dem universitär abgesicherten Text von
Scheid\footcite[vgl.][\nopage]{Scheid2016a} und ist -- entsprechend der
europäischen Tradition -- von links nach rechts und von oben nach unten zu
lesen. Er unterscheidet sich von den chinesischen Versionen, die die anderen
hier zitierten Autoren präsentieren, höchstens in der Punktion.

Die mittlere Spalte meiner Lernversion präsentiert den japanischen Text in
europäischer Umschrift. Sie folgt -- mit drei Ausnahmen -- dem Text von
Deshimaru\footcite[vgl.][30]{Deshimaru1988a} und ist ebenfalls von links nach
rechts und von oben nach unten zu lesen. Die erste Ausnahme betrifft das Wort
\emph{bo sa} in Zeile [006]. Hier steht bei Deshimaru \emph{bo satsu}. Die
zweite Ausnahme betrifft das Wort \emph{ze} in Zeile [061]. Hier steht bei
Deshimaru \emph{ze(tsu)}. Da in der Sangha, zu der ich mich hingezogen
fühle\footcite[vgl.][\nopage]{DaiShinZen2016a}, die Silbe \emph{tsu} nicht
gesprochen wird, habe ich mir erlaubt, es in meiner Lernversion zu unterdrücken.
Inhaltlich entsteht dadurch keine Veränderung, phonetisch nur eine geringe: das
auslautend \emph{u} wird im Japanischen fast nicht gesprochen, jedenfalls noch
weniger als das deutsche Auslaut-e in \emph{Stange} oder \emph{Karte}. Die
dritte Ausnahme betrifft die Groß- und Kleinschreibung: ich habe die konsequente
Kleinschreibung der Version von Scheid übernommen. Die Großschreibung nach einem
Punkt signalisiert harte syntaktische Abschlüsse, die semantisch so nicht
stimmen.

Meine Übersetzung ins Deutsche folgt in der Regel der anregenden, wortweisen
Übersetzung von Boeck\footcite[vgl.][\nopage]{Boeck2016a}, allerdings im
Abgleich mit den Erläuterungen von Deshimaru und Scheid. Mein eigenes Zutun
wollte von Anfang an nicht mehr bieten als eine geschickte Anordnung, bei der
eine möglichst wortgetreue Übersetzung zeilenmäßig in der Nähe der zu
übersetzenden Phrase bleibt. Das Hannya Shingyo sollte in sinnhaften Einheiten
lernbar gemacht werden. Um das zu erreichen, habe ich die großen syntaktischen
Freiheiten der deutschen Sprache genutzt: im Zweifel habe ich die etwas
geschrobenere Formulierung mit genauer Zuordnung der eleganteren, aber
entfernenden vorgezogen.

Um meine eher syntaktisch motivierten Zutaten als solche zu kennzeichnen, habe
ich sie in eckige Klammern eingeschlossen und kursiv gesetzt. Der deutsche Text
sollte sich mit diesen Zutaten schlüssig von links nach rechts und oben nach
unten lesen lassen. Unmarkierte deutsche Wörter sollten in der Zeile stehen, in
denen auch die chinesischen und japanischen Korrelate stehen - jedoch nicht
immer in derselben Reihenfolge, wie die Originale.

Und noch zwei letzte typographische Aufschlüsselung: 

\begin{enumerate}
  \item Die chinesische Schrift ist eine Begriffsschrift. Trotzdem enthält sie
  auch syntaktische Konnektoren, etwa die Negationen \emph{mu} (= \cnbsmi{無})
  und  \emph{fu} (= \cnbsmi{不}), die additive Konjunktion \emph{yaku} (=
  \cnbsmi{亦} = auch), die einfache Schlussfolgerung \emph{ko} (= \cnbsmi{故} =
  darum) oder die betonte Schlussfolgerung \emph{nai shi} (= \cnbsmi{乃}
  \cnbsmi{至} = darum insbesondere)\footnote{Boeck übersetzt \emph{fu} mit der
  deutschen Vorsilbe \emph{un-} und \emph{mu} mit der expliziten Negation
  \emph{nicht}. \emph{yaku} übersetzt er ebenfalls als auch. \emph{ko} übersetzt
  er wörtlich als \emph{Ursache}. Und \emph{nai shi} übersetzt er als \emph{dann
  extrem}, was ich als \emph{darum inbesondere}
  übernehme.\cite[vgl][\nopage]{Boeck2016a}}. Diese Patikel strukturieren den
  Text logisch. Deshalb habe ich sie in der linearen Anordnung jeweils nach
  links herausgezogen. Im selben Sinne habe ich auch einige andere, gliedernde
  Partikel optisch arrangiert.
  \item Im Text erscheint gelegentlich ein verweisender Pfeil
  \emph{$\rightarrow$}. Zu diesen Zeilen gibt es eine Erläuterung der
  Übersetzung. Die Zeilennummern werden im Kapitel mit den Übersetzungshinweisen
  als Referenz benutzt.
\end{enumerate}

\section{Die Übersetzung} 

Einige Entscheidungen habe ich im folgenden erläutert. Mit ist natürlich klar,
dass eine wirklich wissenschaftliche Aufbereitung viele Aspekte und Behauptungen
nachweisen müsste, auf die ich hier ohne Nachweis zurückgreife. Sie sind das
Ergebnis der Arbeit der anderen Autoren. Ihnen gebührt dafür Respekt,
Anerkennung und Dank, nicht mir. In einer späteren Version werde ich die
Nachweise sicher nachholen. Bis dahin möge man mir nachsehen, dass ich einfach
nur eine besser zu lernende Version erstellen wollte.

\begin{description}

  \item[001-003:] Das Hannya Shingyo ist ursprünglich in Sanskrit geschrieben,
  von dort ins Chinesische übertragen und von da aus ist es dann noch einmal ins
  Japanische übersetzt worden. Das Chinesische selbst ist eine Begriffsschrift,
  sodass sich die Übersetzung ins Japanische auf die Definition einer 'anderen'
  Aussprache konzentrieren konnte. Allerdings hatte die chinesische Version
  einige ursprüngliche Formulierung als 'wörtliche Zitate' bewahrt. Dabei ist
  die Aussprache des Sanskrit mit chinesischen Silben lautlich nachgebildet
  worden. Die Übertragung ins Japanische hat diese Idee übernommen.
  Damit entsteht jedoch eine 'Doppeldeutigkeit'. Denn die das Sanskrit mehr oder
  minder gut nachbildenden japanischen Wörter und Silben haben natürlich eine
  eigene unabhängige Bedeutung. Dem entsprechend wird gelegentlich gesagt, die
  Übertragungen hätten die Bedeutung des Hannya Shingyos
  \enquote{vertieft}\footcite[vgl.][56]{Deshimaru1988a}. Das \emph{Hannya
  Shingyo} als Name des Textes ist jedenfalls das erste Zitat aus dem Sanskrit.

  \item[005-006:] Der Ausdruck \emph{kan ji zai bo sa} bildet auch ein solches
  lautliches Zitat, allerdings in etwas \emph{verschleierter Form}: er soll den
  Ausdruck \emph{Boddhisattva Avalokiteshvara} wiedergeben. Dabei beziehen sich
  die Silben \emph{bo sa} direkt auf auf den Titel \emph{Boddhisattva}.
  Titelträger ist im Original \emph{Avalokitesvara}, ein Schüler von Buddha.
  Dieser hat einen Beinamen gehabt, auf den sich die Silben \emph{kan} (=
  \emph{beobachten}) und \emph{ji zai} (= \emph{Freiheit})  beziehen . Darum
  kann man den Namen nicht unübersetzt in einen deutschen Text übernehmen: es
  wird hier eben nicht über eine konkrete Einzelperson gesprochen. Vielmehr
  fungiert diese konkrete Person als Typus. Die so verallgemeinerte Aussage
  erlaubt es dem Hörer, sich einbezogen zu fühlen. Um das im Deutschen
  nachzubilden, nutze ich den unbestimmten Artikel und folge ansonsten der
  Deutung von Deshimaru\footcite[vgl.][57 et passim]{Deshimaru1988a}.

  \item[004,011:] \emph{ji} (= \cnbsmi{時}) soll \emph{Zeit} bedeuten und wird
  als Konjunktion zumeist mit \emph{als} oder \emph{während} übersetzt. Im
  deutschen kennen wir zwei Arten der 'zeitlichen' Verbindung zweier Fakten. Die
  eine betont eher die Zufälligkeit, die andere die Ursächlichkeit:
  \emph{\underline{als} ich Zucker aß, bekam ich Kopfschmerzen} meint etwas
  anderes als, \emph{\underline{indem} ich Zucker aß, bekam ich Kopfschmerzen}.
  Im Hannya Shingyo ist eine ursächliche Verknüpfung gemeint: \emph{Das
  Praktizieren der Höchsten Wahrheit führt zu der Erkenntnis, dass \ldots}. Das
  Wort \emph{indem} markiert diese ursächliche Beziehung gut.

  \item[014:] Die 5 Skandhas -- nämlich \emph{Empfindung, Wahrnehmung, Gedanken,
  Handeln und Bewusstsein} -- bilden eine zentrale Achse des Textes:
  zuerst wird ihr Oberbegriff \emph{go on} (= \cnbsmi{五} \cnbsmi{蘊}) genannt.
  Danach wird von jeder einzelnen gesagt, sie sei nicht nur nicht getrennt von
  \emph{kū}, sondern sie sei \emph{kū} (020-039). Schließlich wird auch gesagt,
  dass es sie in \emph{kū} ansich nicht gäbe (050-054), genauso wenig, wie
  entsprechenden Organe (055-060) oder deren Resulte (061-66). Dem liegt ein
  Weltbild zugrunde, das sicher nicht mehr unseres ist. Deshalb ist es
  angemessen, den fremden Begriff 'Skandha' als \emph{Fremdwort} in die
  Übersetzung zu übernehmen. Allerdings: die Pointe des Hannya Shingyos, dass es
  das, was dieses fremde Weltbild beschreibt, in \emph{kū} nicht gäbe, ließe
  sich umstandlos auch mit unserem heutigen physisch / psychischen Weltbild
  formulieren. Man muss sich also die 'veraltete' Sichtweise nicht zu eigen
  machen, um das Hannya Shingyo zu verstehen und seine Aussage zu bejahen. Das
  Hannya Shingyo ist -- so gesehen -- sehr modern.
  
  \item[021:] Es ist üblich, \emph{shiki} mit \emph{Form} zu übersetzen. Das
  wird der rhetorischen Form des Textes aber nicht gerecht: \emph{shiki} ist die
  erste der 5 Skandas. Die anderen 4 werden in den Zeilen [035-038] aufglistet.
  Die Übersetzung von \emph{shiki} muss auch das 1. Skandha schon als Teil einer
  Reihe erscheinen lassen. Dazu eignet sich das Wort \emph{Form} nicht.

  \item[021-033:] Außerdem wird diese ganze Sentenz gelegentlich zu der Aussage
  verknappt, \emph{Form sei Leere und Leere sei Form}. Damit geht eine -- auch
  rhetorisch entscheidende -- Pointe des Originals verloren: Zuerst sagt das
  Hannya Shingyo, \emph{shiki}, die \emph{Erscheinung} sei nicht getrennt von
  \emph{kū}. Dies muss den Hörer verwirren. Denn das normale Verständnis besagt
  doch wohl eher, dass es sich dabei um verschiedene Dinge handelt. Und mit
  diesem Erwartungshorizont spielt der Text. Denn er setzt danach -- sozusagen
  -- 'noch eins drauf': Er verschäft die Situation, in dem er sagt, dass die
  \emph{Erscheinung} und \emph{kū} nicht nur nicht getrennt seien, sondern dass
  das eine realiter auch das andere \emph{sei}. Rhethorisch gesehen präsentiert
  das Hannya Shingyo also zuerst eine 'steile' These, die es im folgenden wird
  erläutern und begründen müssen. Auf jeden Fall -- und das ist der rednerische
  Zweck dieses Vorgehens -- hat es mit dieser Konstruktion die Aufmerksamkeit
  seiner Hörer geweckt. Darum ist es notwendig, diese rhethorische Verschärfung
  auch in der Übersetzung zu erhalten.
  
  \item[022:] Oft wird \emph{i} mit \emph{verschieden} übersetzt. Das wird dem
  Original nicht gerecht. Denn tatsächlich geht es im folgenden Text [044-087],
  in dem \emph{kū} ex negativo definiert wird, um nichts anderes, als die
  Feststellung von Unterschieden.  Die Pointe des Hannya Shingyos ist aber, dass
  \emph{kū} trotz aller Verschiedenartigkeit dennoch -- irgendwie -- mit den 5
  Skandhas zusammenfällt, also trotz aller Verschiedenartigkeit nicht getrennt
  ist von \emph{shiki}. Darum habe ich mich für das Übersetzung \emph{getrennt}
  entschieden; es unterstreicht die intellektuelle Brisanz des Hannya Shingyos.

  \item[023ff:] Es ist üblich, \emph{kū} mit dem Wort \emph{Leere} zu
  übersetzen. Allerdings bringt das Wort \emph{Leere} eigene Konnotationen mit,
  die dem eigentlich Gemeinten entgegenstehen. Das Problem schillernder Begriffe
  kennt pikanterweise sogar das Hannya Shingyo selbst, mehr noch: es spielt
  sogar mit dem Phänomen: Es nimmt nämlich einen dem Gemeinten nahestehenden,
  vermeintlich klaren Begriff \emph{kū} und schärft diesen mittels Aussagen
  darüber, was das Gemeinte alles \emph{nicht} ist. Solch ein Verfahren nennt
  man eine \emph{Ex-Negativo-Definition}. Tatsächlich besteht das Hannya Shingyo
  im Kern aus einer Liste von negierenden Abgrenzungen [044-087]. Aus diesem
  Grund ist es besser, nicht das auch durch die europäische Philosophie
  aufgeheizte Wort \emph{Leere} durch vielfache Wiederholgungen zum Kern zu
  machen, sondern das Original -- also \emph{kū} --  zu verwenden und dessen
  Bedeutung gerade über Negationen klarwerden zu lassen.

  \item[050:] \emph{ze ko} (= \cnbsmi{是} \cnbsmi{故}) steht für \emph{sein
  Ursache}. Während ich später in Zeile [095ff] \emph{ko} konsequent als
  \emph{darum} übersetze, um den repitiven Charakter zu erhalten, wähle ich hier
  - zu Beginn der Deduktion - das stärkere und elegantere \emph{mithin} als
  Übersetzung.
  \item[072:] \emph{nai shi} besagt für sich genommen \emph{dann extrem}. Es
  geht also um eine besonders wichtige Schlussfolgerung. Solch ein sprachliches
  Konstrukt kennen wir auch im Deutschen, nämlich die einleitende Formel:
  \emph{Darum ist/wird/\ldots inbesondere \ldots}.

  \item[071-073:] Die Kombination \emph{gen kai} (= \cnbsmi{眼} \cnbsmi{界}) steht
  wörtlich für \emph{[Auge Welt]}, die Sequenz \emph{i shiki kai} (= \cnbsmi{意}
  \cnbsmi{識} \cnbsmi{界}) hingegen für \emph{Denkvermögen Unterscheiden Welt}.
  Erstere meint also die sichtbare, die erscheinende Welt, letztere die Welt der
  trennenden Vorstellungen und Konzepte. Auch in dieser Gegenüberstellung trifft
  man indirekt die fünf Skandas wieder: Zeile [058] hat schon \emph{gen} (= das
  \emph{Auge}) dem ersten Skandha \emph{shiki} (= \cnbsmi{色} =
  \emph{Erscheinen}) aus Zeile [053] als Organ zugeordnet. Für das fünfte
  Skandha, das Unterscheiden als intellektuelles Tun -- japanisch ebenfalls
  \emph{shiki} genannt -- wird ein anderes Zeichen benutzt als für das erste
  Skandha, nämlich \cnbsmi{識} (Zeile [057]. Und eben dieses zweite \emph{shiki}
  erscheint auch in Zeile [073]. Die rhetorische Konstruktion 'von \emph{gen
  kai} bis \emph{shiki kai} spannt also indirekt erneut den ganzen Bogen über
  alle fünf Skandhas auf.

  \item[074-080:] Die rhetorische Konstruktion \emph{Es gibt in kū nicht XYZ}
  und \emph{Es gibt in kū kein Ende von XYZ} ist besonders aufreizend für
  (europäische) Logiker: Ersteres negiert die Existenz von XYZ; letzteres setzt
  seine Existenz voraus und betont diese durch den impliziten Hinweis auf seine
  Ewigkeit, ausgedrückt durch eine doppelte Verneinung. Damit widersetzt sich
  das Hannya Shingyo der formalen Logik, in dem es dem europäischen Verständnis
  sein \emph{tertium datur} entgegenstellt, nicht ohne diese Logik allerdings
  selbst souverän zu benutzen. Dem Zen entsprechend ist das kein Widerspruch,
  sondern geradezu der Sinn allen Tuns: alle gedanklichen Konstrukte müssen
  aufgehoben werden, wenn \emph{kū} selbst im  Akt der Erleuchtung erfahrbar
  werden soll.

  \item[111-113:] Auch die Sentenz \emph{anokutara sanmyaku sanbodai} ist eine
  zitierende Sanskritnachahmung und meint \emph{höchste, vollkommene
  Erleuchtung}\footcite[vgl.][\nopage Anm. 10]{Scheid2016a}. Welches der Worte
  was bedeutet, habe ich den Quellen bisher nicht entnehmen können. Meine
  Zuordnung ist also willkürlich, folgt aber der Tradition.

  \item[115-121:] Hier findet eine rhetorisch geniale Umdeutung statt, die eine
  große Auswirkung auf den Buddhismus hat: Bisher war der Begriff \emph{han nya
  ha ra mi ta} beschreibend. Er stand für die \emph{die höchste Weisheit, die
  über sich hinausführt}. Jetzt wird der Ausdruck zum Namen des Textes selbst:
  indem er mehrfach als herausgehobenes \emph{Mantra} bezeichnet wird,
  verschiebt sich seine Bedeutung: der Terminus \emph{han nya ha ra mi ta} wird
  zum Namen des Textes. Und in dem diesem dann auch noch eine Wirkung
  zugesprochen wird, wird seine Rezitation zu einem Mittel. Kein Wunder also,
  dass alle Buddhisten diesen Text rezitieren: es steckt in ihm selbst.
 
  \item[119:] Die Phrase \emph{mu jō shu} verwendet wieder einmal eine der im
  \emph{Hannya Shingyo} so gern genutzte 'negative Zuschreibungen':
  \emph{mu} (= \cnbsmi{無}) ist die bekannte Verneinigung; und \emph{shu} (=
  \cnbsmi{咒}) steht für das \emph{Mantra}. Also wird \emph{jō} (= \cnbsmi{上})
  ein Attribut sein, das negiert dem Objekt \emph{Mantra} zugsprochen wird:
  Ein chinesisch-deutsches Internetlexikon sagt, das \cnbsmi{上} auch für
  \emph{von unten nach oben, aufwärts} bzw. \emph{vorwärts gehen}
  steht\footcite[vgl.][\nopage]{babla2016a}. Eine gute Übersetzung würde auch an
  dieser Stelle -- auf der Basis dieser Primärbedeutung -- die bevorzugte
  Methode der Eingrenzung ohne direkte Spezifikation bewahren; sie würde diese
  'ZEN gemäße' Art des 'Denkens' auch hier verdeutlichen. Hier fehlt mir noch
  eine gute Idee für die Umsetzung.

  \item[125-126:] \emph{shin jitsu}(= \cnbsmi{真} \cnbsmi{實}) soll
  \emph{Realität} meinen, und \emph{fu ko} (= \cnbsmi{不} \cnbsmi{虛} ) für
  \emph{nicht/keine Unwahrheit} stehen. Ersteres übersetze ich mit
  \emph{wirklich}, letzteres müsste dann \emph{wahr} heißen. Ich belasse
  letzteres aber bei \emph{nicht unwahr}, um die Neigung des Hannya Shingyos zur
  (doppelten) Verneinung zu erhalten.
\end{description}

% insert the bibliographical data here
\bibliography{bibfiles/hsResourcesDe}

\end{document}
 }

\maketitle
%%-- end(titlepage)
\section{Der Anlass}
 
Wäre es nicht schön, das \emph{Hannya Shingyo} -- mit anderen zusammen -- auch
auswendig vortragen zu können? Immerhin hat die Retization dieses Textes im
(Zen)-Buddhismus eine große Tradition!

Der Weg zum flüssigen Mitsprechen ist holprig: Wie lernt man solch eine sperrige
Folge japanischer Silben, wie einen so erratischen Textblock? Das Lernen dürfte
leichter fallen, wenn eine Struktur erkennbar wäre, etwa in einer
mehrspaltigen, mehrsprachigen, sinnhaft gegliederten Aufbereitung.

Dazu müsste der japanische Text jedoch recht wortgetreu übersetzt sein.
Denn nur so ließe sich die Übersetzung in einer Zeile mit dem übersetzten
Satzteil arrangieren. Würden die deutschen mit den japanisch-chinesischen
Phrasen so auch optisch korrespondieren, erschlössen sich die
Sinneinheiten direkt.

Trotzdem sollte die Übersetzung auch noch elegant sein: Das \emph{Hannya
Shingyo} ist ein Lehrtext, ein Sutra. Zuerst dürfte es mündlich vorgetragen
worden sein, als Ansprache an die Schüler. Mithin wird man darin -- ganz
sprachunabhängig -- auch rhetorische Elemente finden: Einen Interesse weckenden
\emph{Einstieg} etwa. Oder eine aufrüttelnde \emph{Kernthese}, die allmähliche
\emph{Entfaltung} ihrer Feinheiten, und die sich daran anschließende Begründung
der \emph{Konsequenzen}. Und natürlich einen einprägsamen \emph{Schluss}. Wäre es
nicht schön, wenn ein \emph{Hannya-Shingyo-Lerntext} auch das noch erkennen
ließe?

Gleichwohl müsste die Übertragung immer genau bleiben, von der Bedeutung und der
syntaktischen Struktur her\footnote{Doris Wolter hat dankenswerterweise
verschiedene Übersetzungen ins Deutsche zusammengetragen.
(\cite[vgl.][\nopage]{Wolter2010a}) Vergleicht man diese Versionen, offenbaren
sich erhebliche Unterschiede. Insbesondere das letzte Drittel des Hannya
Shingyos scheint dabei zu besonders 'poetischen' Übertragungen einzuladen.
Angesichts der existentiellen philosophischen Dimension des Zen-Buddhismus und
des Anspruchs auf letztgültige Wahrheiten im \emph{Hannya Shingyo} selbst ist
das schlicht unzufriedenstellend.}. Sie sollte so wenig als möglich
interpretieren.

Es gibt wunderbare Übersetzungen: z.B. die von
Deshimaru\footcite[vgl.][]{Deshimaru1988a}, die eher ein philosophischer
Hintergrundbericht sein will, als eine pure Übersetzung. Oder die universitär
abgesicherte, elegante Übertragung von Scheid\footcite[vgl.][]{Scheid2016a}.
Oder die wortgetreue von Boeck\footcite[vgl.][]{Boeck2016a}.

Nur liefern sie alle leider keinen mehrsprachigen, sinnhaft gegliederten
Lerntext, der bei aller Worttreue auch noch die elegante Rhetorik des Originals
erahnen ließe. Wie wäre es also mit folgender Variante?

\newpage
\section{Der Text} 

\sffamily

\begin{center}
\begin{tabular}{r|rl|rl|rl}
~ & \multicolumn{6}{l}{\textsc{Der Titel:}}\\
\hline
{\tiny\texttt{001}}&
  \multicolumn{2}{l|}{\cnbsmi{摩}  \cnbsmi{訶} \cnbsmi{般} \cnbsmi{若}} &
  \multicolumn{2}{l|}{\textbf{ma kā}  \textbf{han nya}} &
  \textrm{\emph{[Die]}}& \textrm{maha prajñā \emph{= höchste Weisheit}}\\
{\tiny\texttt{002}}&
  ~ & \cnbsmi{波} \cnbsmi{羅} \cnbsmi{蜜} \cnbsmi{多} & 
  ~ & \textbf{ha ra mi tā} & 
  {\tiny \textrm{($\rightarrow$)}} & 
    \textrm{pāramitā\emph{, die über sich hinausführt,}}\\
{\tiny\texttt{003}}& 
  ~ & \cnbsmi{心} \cnbsmi{經} &
  ~ & \textbf{shin gyō} & 
  \textrm{\emph{[als das]}} & \textrm{essentielle Sutra \emph{[schlechthin]}} \\
\hline
~ & \multicolumn{6}{l}{\textsc{Das Manifest:}}\\
\hline
{\tiny\texttt{004}}& 
~ & ~  & ~ & ~ &
  \textrm{Indem} {\tiny \textrm{($\rightarrow$)}} & \textrm{\emph{[ein der]}} \\
{\tiny\texttt{005}}&
  ~ & \cnbsmi{觀} \cnbsmi{自} \cnbsmi{在} & 
  ~ & \textbf{kan ji zai} & 
  {\tiny \textrm{($\rightarrow$)}} &
    \textrm{freien Sicht \emph{[zugewandter]}} \\
{\tiny\texttt{006}}&
  ~ & \cnbsmi{菩} \cnbsmi{薩} \cnbsmi{。}& 
  ~ & \textbf{bo} \textbf{sa}.& 
  ~ & \textrm{\emph{[lebender Buddha, ein]} Bodhisattva} \\  
{\tiny\texttt{007}}& 
  ~ & \cnbsmi{行} \cnbsmi{深} &
  ~ & \textbf{gyō} \textbf{jin} & 
  ~ & \textrm{tief \emph{[und gründlich]} praktizierend} \\  
{\tiny\texttt{008}}& 
  ~ & \cnbsmi{般} \cnbsmi{若} & 
  ~ & \textbf{han nya} & 
  ~ & \textrm{\emph{[die]} Prajñā \emph{, Weisheit}} \\  
{\tiny\texttt{009}} &
  ~ & \cnbsmi{波} \cnbsmi{羅} \cnbsmi{蜜} \cnbsmi{多}& 
  ~ & \textbf{ha ra mi ta} & 
  ~ & \textrm{Pāramitā \emph{, die über sich hinausführt,}} \\  
{\tiny\texttt{010}}&
  ~ & ~  & ~ & ~ &  ~ & \textrm{\emph{[lebt]}} \\
{\tiny\texttt{011}}&
  \cnbsmi{時}&\cnbsmi{。} &
  \textbf{ji}. & ~ &
  {\tiny \textrm{($\rightarrow$)}} & ~ \\
{\tiny\texttt{012}}& 
  ~ & ~ & ~ & ~ & ~ & \textrm{\emph{[kommt es bei ihm zum]}} \\
{\tiny\texttt{013}}& 
  ~ & \cnbsmi{照} \cnbsmi{見} &
  ~ & \textbf{shō ken} &
  ~ & \textrm{erleuchteten Sehen \emph{[, dass die]}} \\  
{\tiny\texttt{014}}& 
  ~ & \cnbsmi{五} \cnbsmi{蘊} & 
  ~ & \textbf{go on} & 
  {\tiny \textrm{($\rightarrow$)}} & \textrm{5 Skandas}  \\
{\tiny\texttt{015}}&
  ~ & \cnbsmi{皆} \cnbsmi {空} \cnbsmi{。} &
  ~ & \textbf{kai kū}. & 
  ~ & \textrm{alle leer \emph{[sind]}} \\
{\tiny\texttt{016}}&
  \cnbsmi{度} & ~ &
  \textbf{do} & ~ &
  \textrm{\emph{[und]} so} & ~ \\
{\tiny\texttt{017}}&
  ~ & \cnbsmi{一} \cnbsmi{切} &
  ~ & \textbf{is sai} &
  ~ & \textrm{entfernt \emph{[er]}} \\
{\tiny\texttt{018}}&
  ~ & \cnbsmi{苦} \cnbsmi{厄} \cnbsmi{。} & ~ &
  \textbf{ku yaku}. & 
  ~ & \textrm{Leiden \emph{[und]} Unheil.} \\
\hline
  ~ & \multicolumn{6}{l}{\textsc{Die Kernthese:}}\\
\hline
{\tiny\texttt{019}}&
  \multicolumn{2}{l|}{\cnbsmi{舍} \cnbsmi{利} \cnbsmi{子}\cnbsmi{。}}  &
  \multicolumn{2}{l|}{\textbf{sha ri shi}.} & ~ &
  \textrm{Shariputra!}\\
\hline  
{\tiny\texttt{020}}&
  ~ & ~ & ~ & ~ & 
  \multicolumn{2}{l}{\textrm{\emph{[Die 1. der 5 Skandas, nämlich die]}}} \\
{\tiny\texttt{021}}&
  ~ & \cnbsmi{色} & 
  {\tiny \textrm{($\rightarrow$)}} & \textbf{shiki} &
  ~ & \textrm{Erscheinung} \\  
{\tiny\texttt{022}}&
  \cnbsmi{不} & \cnbsmi{異} & 
  \textbf{fu} & \textbf{i} &
  \textrm{\emph{[ist]} nicht} & \textrm{getrennt \emph{[von]}} \\  
{\tiny\texttt{023}}&
  ~ & \cnbsmi{空} \cnbsmi{。} &
  {\tiny \textrm{($\rightarrow$)}} & \textbf{kū}.  &
  {\tiny \textrm{($\rightarrow$)}} & \textrm{kū, \emph{[der Leere]}} \\
\hdashline
{\tiny\texttt{024}}&
  ~ & \cnbsmi{空} &
  ~ & \textbf{kū} & 
  \textrm{\emph{[und]}} & \textrm{kū, \emph{[die Leere]}} \\
{\tiny\texttt{025}}&
  \cnbsmi{不} & \cnbsmi{異} &
  \textbf{fu} & \textbf{i} &
  \textrm{\emph{[ist]} nicht} & \textrm{getrennt \emph{[von]}} \\  
{\tiny\texttt{026}}&
  ~ & \cnbsmi{色} \cnbsmi{。} &
  ~ & \textbf{shiki}. &
  ~ & \textrm{\emph{[der]} Erscheinung.} \\
\hline
{\tiny\texttt{027}}&
  ~ & ~ & ~ & ~ & \multicolumn{2}{l}{\textrm{~\emph{Ja, mehr noch:}}}  \\  
{\tiny\texttt{028}}&
  ~ & \cnbsmi{色} &
  ~ & \textbf{shiki} & 
  ~ & \textrm{\emph{[Die]} Erscheinung} \\  
{\tiny\texttt{029}}&
  \cnbsmi{即} & \cnbsmi{是} & 
  \textbf{soku} & \textbf{ze} &
  \textrm{ist} & \textrm{eigentlich} \\  
{\tiny\texttt{030}}&
  ~ & \cnbsmi{空} \cnbsmi{。} &
  ~ & \textbf{kū}. &
  ~ & \textrm{kū, \emph{[die Leere]}} \\
 \hdashline
 {\tiny\texttt{031}}&
  ~ & \cnbsmi{空} &
  ~ & \textbf{kū} & 
  \textrm{\emph{[und]}} & \textrm{kū, \emph{[die Leere]}} \\
{\tiny\texttt{032}}&
  \cnbsmi{即} & \cnbsmi{是} & 
  \textbf{soku} & \textbf{ze} &
  \textrm{ist} & \textrm{eigentlich} \\  
{\tiny\texttt{033}}&
 ~ & \cnbsmi{色} \cnbsmi{。} &
 ~ & \textbf{shiki}. &~
 ~ & \textrm{\emph{[die]} Erscheinung.} \\
 \hline
 {\tiny\texttt{034}}&
    ~ & ~ & ~ & ~ & \multicolumn{2}{l}{
    \textrm{\emph{[Und bei den anderen 4 Skandas, also beim]}}}\\
 {\tiny\texttt{035}}&
  ~ & \cnbsmi{受} &
  ~ & \textbf{ju} &
  ~ & \textrm{Empfinden,} \\
{\tiny\texttt{036}}&
  ~ & \cnbsmi{想} &
  ~ & \textbf{sō} &
  ~ & \textrm{Wahrnehmen,} \\
 {\tiny\texttt{037}}&
  ~ & \cnbsmi{行} &
  ~ & \textbf{gyō} &
  ~ & \textrm{Wollen \emph{[und]}} \\
 {\tiny\texttt{038}}&
  ~ & \cnbsmi{識} &
  ~ & \textbf{shiki}. &
  ~ & \textrm{Unterscheiden}, \\
{\tiny\texttt{039}}&
  \cnbsmi{亦} & \cnbsmi{復} \cnbsmi{如} \cnbsmi{是} \cnbsmi{。} &
  \textbf{yaku} & \textbf{bu nyo ze}. &
  \textrm{auch \emph{[da]}} & \textrm{ist \emph{[es]} wieder gleich}. \\
\hline
  ~ & \multicolumn{6}{l}{\textsc{Die ex negativo Definition von \textrm{kū}:}}\\
\hline
{\tiny\texttt{040}} &
  \multicolumn{2}{l|}{\cnbsmi{舍} \cnbsmi{利} \cnbsmi{子} \cnbsmi{。}} &
  \multicolumn{2}{l|}{\textbf{sha ri shi}.} &
  ~ & \textrm{Shariputra!}\\
\hline
{\tiny\texttt{041}}&
  \cnbsmi{是} & \cnbsmi{諸} &
  \textbf{ze} & \textbf{sho} &
  \textrm{\emph{[Es]} ist} & \textrm{alles} \\
{\tiny\texttt{042}}&
  ~ & \cnbsmi{法} &
  ~ & \textbf{hō} &
  ~ & \textrm{Seiende} \\
{\tiny\texttt{043}}&
  ~ & \cnbsmi{空} \cnbsmi{相} \cnbsmi{。} &
  ~ & \textbf{kū sō}. &
  ~ & \textrm{\emph{[ein]} Aspekt \emph{[von]} kū}: \\
\hdashline
{\tiny\texttt{044}}&
  \cnbsmi{不} & \cnbsmi{生} &
  \textbf{fu} & \textbf{shō} & 
  \textrm{nicht} & \textrm{geboren \emph{[bzw.]} geschaffen} \\
{\tiny\texttt{045}}&
  \cnbsmi{不} & \cnbsmi{滅} \cnbsmi{。} &
  \textbf{fu} & \textbf{metsu}. &
  \textrm{nicht} & \textrm{gestorben \emph{[bzw.]} ausgelöscht},\\
\hdashline
{\tiny\texttt{046}}&
  \cnbsmi{不} & \cnbsmi{垢} &
  \textbf{fu} & \textbf{ku} &
  \textrm{nicht} & \textrm{befleckt} \\
{\tiny\texttt{047}}&
  \cnbsmi{不} & \cnbsmi{淨} &
  \textbf{fu} & \textbf{jō}. &
  \textrm{nicht} & \textrm{rein}, \\
\hdashline
\end{tabular}

\begin{tabular}{r|rl|rl|rl}
\hdashline
{\tiny\texttt{048}}&
  \cnbsmi{不} & \cnbsmi{增} &
  \textbf{fu} & \textbf{zō} &
  \textrm{nicht} & \textrm{zunehmend} \\
{\tiny\texttt{049}}&
  \cnbsmi{不} & \cnbsmi{減} &
  \textbf{fu} & \textbf{gen}. &
  \textrm{nicht} & \textrm{abnehmend}. \\
\hline
{\tiny\texttt{050}}&
  \cnbsmi{是} & \cnbsmi{故} &
  \textbf{ze} & \textbf{ko} &
  \textrm{Mithin} & {\tiny ($\rightarrow$)} \textrm{\emph{[gibt es]}}\\
{\tiny\texttt{051}}&
  ~ & \cnbsmi{空} \cnbsmi{中} \cnbsmi{。} &
  ~ & \textbf{kū chū}. &
  ~ & \textrm{in kū} \\
{\tiny\texttt{052}}&
  ~ & ~  & ~ & ~ &  ~ & \textrm{\emph{[keines der 5 Skandhas, also]}} \\
{\tiny\texttt{053}}&
  \cnbsmi{無} & \cnbsmi{色} \cnbsmi{。} &
  \textbf{mu} & \textbf{shiki} &
  \textrm{kein} & \textrm{Erscheinen}, \\
{\tiny\texttt{054}}&
  \cnbsmi{無} & \cnbsmi{受} &
  \textbf{mu} & \textbf{ju} & 
  \textrm{kein} & \textrm{Empfinden,} \\
{\tiny\texttt{055}}&
  ~ & \cnbsmi{想} &
  ~ & \textbf{sō} &
  ~ & \textrm{Wahrnehmen,} \\
{\tiny\texttt{056}}&
  ~ & \cnbsmi{行} &
  ~ & \textbf{gyō} &
  ~ & \textrm{Wollen \emph{[oder]}} \\
{\tiny\texttt{057}}&
  ~ & \cnbsmi{識} \cnbsmi{。} &
  ~ & \textbf{shiki}. & 
  ~ & \textrm{Unterscheiden}, \\
\hdashline
 {\tiny\texttt{058}}&
  \cnbsmi{無} & \cnbsmi{眼} &
  \textbf{mu} & \textbf{gen} &
  \textrm{keine} & \textrm{Augen,} \\
{\tiny\texttt{059}}&
  ~ & \cnbsmi{耳} &
  ~ & \textbf{ni} &
  ~ & \textrm{Ohren,} \\
{\tiny\texttt{060}}&
  ~ & \cnbsmi{鼻} &
  ~ & \textbf{bi} &
  ~ & \textrm{Nase,} \\
{\tiny\texttt{061}}&
  ~ & \cnbsmi{舌} &
  ~ & \textbf{ze} &
  ~ & \textrm{Zunge,} \\
{\tiny\texttt{062}}&
  ~ & \cnbsmi{身} &
  ~ & \textbf{shin} &
  \textrm{\emph{[keinen]}} & \textrm{Tastsinn \emph{[und]}} \\
{\tiny\texttt{063}}&
  ~ & \cnbsmi{意} \cnbsmi{。} &
  ~ & \textbf{i}. & 
  \textrm{\emph{[kein]}} & \textrm{Denkvermögen}. \\
\hdashline
{\tiny\texttt{064}}&
   \cnbsmi{無} & \cnbsmi{色} &
   \textbf{mu} & \textbf{shiki} &
   \textrm{keine} & \textrm{Farbe,} \\
{\tiny\texttt{065}}&
   ~ & \cnbsmi{聲} &
   ~ & \textbf{shō} &
   \textrm{\emph{[keinen]}} & \textrm{Klang,} \\
{\tiny\texttt{066}}&
   ~ & \cnbsmi{香} &
   ~ & \textbf{kō} &
   ~ & \textrm{Geruch,} \\
{\tiny\texttt{067}}&
  ~ & \cnbsmi{味} &
  ~ & \textbf{mi} &
  ~ & \textrm{Geschmack,} \\
{\tiny\texttt{068}}&
  ~ & \cnbsmi{觸} &
  ~ & \textbf{soku} & 
  \textrm{\emph{[keine]}} & \textrm{Berührung \emph{[und]}} \\
{\tiny\texttt{069}}&
  ~ & \cnbsmi{法} \cnbsmi{。} &
  ~ & \textbf{hō}.&
  \textrm{\emph{[keinen]}} & \textrm{Gedanken}; \\
\hline
{\tiny\texttt{070}}&
  ~ & ~ & ~ & ~ & ~ & \textrm{\emph{[Also gibt es in kū]}} \\
\hdashline
{\tiny\texttt{071}}&
  \cnbsmi{無} & \cnbsmi{眼} \cnbsmi{界} \cnbsmi{。} &
  \textbf{mu} & \textbf{gen kai} &
  \textrm{nicht} & \textrm{die sichtbare Welt} {\tiny ($\rightarrow$)} \\
{\tiny\texttt{072}}&
  \cnbsmi{乃}\cnbsmi{至} & ~ & 
  \textbf{nai shi} & ~ &
  \textrm{\emph{[und]} {\tiny ($\rightarrow$)}} & 
   \textrm{darum insbesondere [auch]}\\
{\tiny\texttt{073}}&
  \cnbsmi{無} & \cnbsmi{意} \cnbsmi{識} \cnbsmi{界} \cnbsmi{。}&
  \textbf{mu} & \textbf{i shiki kai}.&
  \textrm{nicht} & 
    \textrm{die Welt der Vorstellungen {\tiny ($\rightarrow$)}} \\
\hdashline
{\tiny\texttt{074}}&
  \cnbsmi{無} & \cnbsmi{無} \cnbsmi{明} \cnbsmi{。} &
  \textbf{mu} & \textbf{mu myō} &
  \textrm{kein} & \textrm{Nicht-Wissen \emph{[und]}} \\
{\tiny\texttt{075}}&
  \cnbsmi{亦} & ~ & 
  \textbf{yaku} & ~ &
  \textrm{auch} & ~ \\  
{\tiny\texttt{076}}&
  \cnbsmi{無} & \cnbsmi{無} \cnbsmi{明} \cnbsmi{盡} \cnbsmi{。} &
  \textbf{mu} & \textbf{mu myō jin.} &
  \textrm{kein} & \textrm{Ende vom Nicht-Wissen} {\tiny ($\rightarrow$)} \\
\hdashline
{\tiny\texttt{077}}&
  \cnbsmi{乃}\cnbsmi{至} & ~ & 
  \textbf{nai shi} & ~ &
  \textrm{\emph{[und]} {\tiny ($\rightarrow$)}} & 
   \textrm{darum insbesondere \emph{[auch]}}\\
\hdashline
{\tiny\texttt{078}}&
  \cnbsmi{無} & \cnbsmi{老} \cnbsmi{死} \cnbsmi{。} &
  \textbf{mu} & \textbf{rō shi} &
  \textrm{kein} & \textrm{Altern und Tod \emph{[und]}} \\
{\tiny\texttt{079}}&
  \cnbsmi{亦} & ~ &
  \textbf{yaku} & ~ &
  \textrm{auch} & ~ \\
{\tiny\texttt{080}}&
  \cnbsmi{無} & \cnbsmi{老} \cnbsmi{死} \cnbsmi{盡} \cnbsmi{。} &
  \textbf{mu} &
  \textbf{rō shi jin}. &
  \textrm{kein} & \textrm{Ende von Altern und Tod {\tiny ($\rightarrow$)}}\\
\hline
{\tiny\texttt{081}}&
  \cnbsmi{無} & \cnbsmi{苦} & 
  \textbf{mu} & \textbf{ku} &
  \textrm{kein} & \textrm{Leiden,} \\  
{\tiny\texttt{082}}&
  ~ & \cnbsmi{集} &
  ~ & \textbf{shū} & 
  ~ & \textrm{Anhäufen,} \\  
{\tiny\texttt{083}}&
  ~ & \cnbsmi{滅} &
  ~ & \textbf{metsu} & 
  ~ & \textrm{Verlöschen \emph{[und]}} \\  
{\tiny\texttt{084}}&
  ~ & \cnbsmi{道} \cnbsmi{。} & 
  ~ & \textbf{dō}. &
  \textrm{\emph{[keinen]}} & \textrm{Weg,} \\
\hdashline 
{\tiny\texttt{085}}&
  \cnbsmi{無} & \cnbsmi{智} &
  \textbf{mu} & \textbf{chi} &
  \textrm{keine} & \textrm{Erkenntnis \emph{[und]}} \\  
{\tiny\texttt{086}}&
  \cnbsmi{亦} & ~ &
  \textbf{yaku} & ~ &
  \textrm{auch} & ~ \\
{\tiny\texttt{087}}&
  \cnbsmi{無} & \cnbsmi{得} \cnbsmi{。} &
  \textbf{mu} & \textbf{toku}. &
  \textrm{keinen} & \textrm{Gewinn,} \\  
{\tiny\texttt{088}}&
  \cnbsmi{以} & ~ &
  \textbf{I} & ~ &
  \textrm{weil} &  \textrm{\emph{[kū]}} \\  
{\tiny\texttt{089}}&
  \cnbsmi{無} & \cnbsmi{所} \cnbsmi{得}  &
  \textbf{mu} & \textbf{sho tok}u  &
  \textrm{kein} & \textrm{Ort \emph{[des]} Gewinnens \emph{[ist]}.} \\
\hline
  ~ & \multicolumn{6}{l}{\textsc{Die praktische Konsequenz:}}\\
\hline
{\tiny\texttt{090}}&
  \cnbsmi{故}\cnbsmi{。} & 
    \cnbsmi{菩} \cnbsmi{提} \cnbsmi{薩} \cnbsmi{捶}\cnbsmi{。}  &
  \textbf{ko}. & \textbf{bo dai sat ta.} &
  \textrm{Darum} & 
    \textrm{\emph{[gilt:] [Ein]} Bodhisattva \emph{[zu sein,]}}\\
{\tiny\texttt{091}}&
  ~ & \cnbsmi{依} &
  ~ & \textbf{e} & 
  ~ & \textrm{bedingt \emph{[die]}} \\
\newline
{\tiny\texttt{092}}&
  ~ & \cnbsmi{般} \cnbsmi{若}  &
  ~ & \textbf{han nya} &
  ~ & \textrm{Prajñā \emph{Weisheit}} \\  
{\tiny\texttt{093}}&
  ~ & \cnbsmi{波} \cnbsmi{羅} \cnbsmi{蜜} \cnbsmi{多} &
  ~ & \textbf{ha ra mi ta} &
  ~ & \textrm{Pāramitā\emph{, die über sich hinausführt.}} \\
\hdashline
{\tiny\texttt{095}}&
  \cnbsmi{故}\cnbsmi{。} & 
    \cnbsmi{心} \cnbsmi{無} \cnbsmi{罫} \cnbsmi{礙} \cnbsmi{。} &
  \textbf{ko.} & \textbf{shin mu kei ge} &
  \textrm{Darum} & 
    \textrm{\emph{[wird sein]} Geist nicht behindert.} \\
\hdashline
{\tiny\texttt{096}}&
  ~ & \cnbsmi{無} \cnbsmi{罫} \cnbsmi{礙} & 
  ~ & \textbf{mu kei ge} & 
 \multicolumn{2}{l}
  {\textrm{\emph{[Und da der]} nicht behindert \emph{[wird]},}}\\
{\tiny\texttt{097}}&
  \cnbsmi{故}\cnbsmi{。} & \cnbsmi{無} \cnbsmi{有} & 
  \textbf{ko.} & \textbf{mu u }& 
  \textrm{darum} & \textrm{hat \emph{[der Bodhisattva]} keine} \\
{\tiny\texttt{098}}&
  ~ & \cnbsmi{恐} \cnbsmi{怖}\cnbsmi{。} &
  ~ & \textbf{ku fu} &
  ~ & \textrm{Furcht}. \\ 
\hline
\end{tabular}

\begin{tabular}{r|rl|rl|rl}
\hdashline
{\tiny\texttt{099}}&
  ~ & \cnbsmi{遠} \cnbsmi{離} &
  ~ & \textbf{on ri} &
  \textrm{\emph{[Das]}} & \textrm{übersteigend\emph{[, was er sich]}} \\      
{\tiny\texttt{100}}&
  ~ & \cnbsmi{一} \cnbsmi{切} &
  ~ & \textbf{is sai} & 
  ~ & \textrm{entfernt \emph{[hat -- nämlich]} }\\
{\tiny\texttt{101}}&
  ~ & \cnbsmi{顛} \cnbsmi{倒} &
  ~ & \textbf{ten dō} &
  ~ & \textrm{Täuschungen \emph{[und]}} \\      
{\tiny\texttt{102}}&
  ~ & \cnbsmi{夢} \cnbsmi{想} \cnbsmi{。} &
  ~ & \textbf{mu sō}. &
  ~  & \textrm{Illusionen \emph{[--]}} \\      
{\tiny\texttt{103}}&
  ~ & \cnbsmi{究} \cnbsmi{竟} &
  ~ & \textbf{ku gyō} &
  ~ & \textrm{erreicht \emph{[er]} schließlich } \\      
{\tiny\texttt{104}}&
  ~ & \cnbsmi{涅} \cnbsmi{槃} \cnbsmi{。}&
  ~ & \textbf{ne han}. &
  ~  & \textrm{\emph{[das]} Nirvana.} \\      
\hline   
{\tiny\texttt{105}}&
  ~ & \cnbsmi{三} \cnbsmi{世} &
  ~ & \textbf{san ze} &
  \textrm{\emph{[Zudem]}} & \textrm{\emph{[gilt seit]} drei Zeitaltern} \\      
{\tiny\texttt{106}}&
  ~ & \cnbsmi{諸} \cnbsmi{佛} \cnbsmi{。} &
  ~ & \textbf{sho butsu} &
  \textrm{\emph{für}} & 
    \textrm{alle Buddhas: \emph{ihre Buddhaschaft}} \\     
 {\tiny\texttt{107}}&
  ~ & \cnbsmi{依} &
  ~ & \textbf{e} &
  ~ & \textrm{bedingt \emph{[die]}} \\  
{\tiny\texttt{108}}&
  ~ & \cnbsmi{般} \cnbsmi{若}  &
  ~ & \textbf{han nya} &
  ~ & \textrm{Prajñā \emph{(Weisheit)}} \\  
{\tiny\texttt{109}}&
  ~ & \cnbsmi{波} \cnbsmi{羅} \cnbsmi{蜜} \cnbsmi{多} &
  ~ & \textbf{ha ra mi ta} &
  ~ & \textrm{Pāramitā\emph{, die über sich hinausführt.}} \\
\hline  
{\tiny\texttt{110}}&
  \cnbsmi{故}\cnbsmi{。} & \cnbsmi{得} &
  \textbf{ko}. & \textbf{toku} &
  \textrm{Darum} & \textrm{gewinnen sie die} \\  
{\tiny\texttt{111}}&
  ~ & \cnbsmi{阿} \cnbsmi{耨} \cnbsmi{多} \cnbsmi{羅} & 
  ~ & \textbf{a noku ta ra} &
  {\tiny ($\rightarrow$)} & \textrm{anuttara} \textrm{\emph{höchste}} \\
{\tiny\texttt{112}}&
  ~ & \cnbsmi{三} \cnbsmi{藐} &
  ~ & \textbf{san myaku} &
  {\tiny ($\rightarrow$)} & \textrm{samyak} \textrm{\emph{vollkommene}} \\      
{\tiny\texttt{113}}&
  ~ & \cnbsmi{三} \cnbsmi{菩} \cnbsmi{提} \cnbsmi{。} &
  ~ & \textbf{san bo dai}. &
  {\tiny ($\rightarrow$)} & \textrm{sambodhi} \textrm{\emph{Erleuchtung}} \\ 
 \hline 
{\tiny\texttt{114}}&
  \cnbsmi{故} & \cnbsmi{知} &
  \textbf{ko} & \textbf{chi} &
  \textrm{Darum} &  \textrm{wisse \emph{[nun Du Deinerseits:]}} \\  
{\tiny\texttt{115}}&
  ~ & \cnbsmi{般} \cnbsmi{若} &
  ~ & \textbf{han nya} &
  \textrm{\emph{[Das]}} & \textrm{Prajñā} {\tiny ($\rightarrow$)} \\  
{\tiny\texttt{116}}&
  ~ & \cnbsmi{波} \cnbsmi{羅} \cnbsmi{蜜} \cnbsmi{多}&
  ~ & \textbf{ha ra mi ta}.&
  ~ & \textrm{Pāramitā} {\tiny ($\rightarrow$)} \\ 
{\tiny\texttt{117}}&
  \cnbsmi{是} & \cnbsmi{大} \cnbsmi{神} \cnbsmi{咒} \cnbsmi{。} &
  \textbf{ze} & \textbf{dai jin shu}. &
  \textrm{ist} & \textrm{\emph{[ein]} großes wunderbares Mantra}; \\  
{\tiny\texttt{118}}&
  \cnbsmi{是} & \cnbsmi{大} \cnbsmi{明} \cnbsmi{咒} &
  \textbf{ze} & \textbf{dai myō shu}. &
  \textrm{\emph{[es]} ist} & \textrm{\emph{[ein]} großes leuchtendes Mantra}, \\ 
{\tiny\texttt{119}}&
  \cnbsmi{是} & \cnbsmi{無} \cnbsmi{上} \cnbsmi{咒} \cnbsmi{。} &
  \textbf{ze} & \textbf{mu jō shu}. &
  \textrm{\emph{[es]} ist} & 
    \textrm{\emph{[das]} {\tiny ($\rightarrow$)} höchste Mantra} \\
{\tiny\texttt{120}}&
  \cnbsmi{是} & \cnbsmi{無} \cnbsmi{等} \cnbsmi{等} \cnbsmi{咒} \cnbsmi{。}  &
  \textbf{ze} &  \textbf{mu tō dō shu}.  &
  \textrm{\emph{[es]} ist} & \textrm{\emph{[das]} nicht übersteigbare Mantra} \\  
{\tiny\texttt{121}}&
  ~ & \cnbsmi{能} & 
  ~ & \textbf{nō} &  
  ~ \textrm{\emph{[es]}} & \textrm{dient \emph{[dem]}} \\
{\tiny\texttt{122}}&
  ~ & \cnbsmi{除} \cnbsmi{一} \cnbsmi{切} & 
  ~ & \textbf{jo is sai} & 
  ~ & \textrm{Beseitigen \emph{[und]} Abschneiden} \\
{\tiny\texttt{123}}&
  ~ & \cnbsmi{苦} \cnbsmi{。} & 
  ~ & \textbf{ku}. & 
 ~ & \textrm{\emph{[von]} Leiden.} \\
\hline 
  ~ & \multicolumn{6}{l}{\textsc{Das Fazit \ldots}}\\
\hline
{\tiny\texttt{124}}&
  ~ & ~ & ~ & ~ &  & \textrm{\emph{[Und weil dies]}} \\      
{\tiny\texttt{125}}&
  ~ & \cnbsmi{真} \cnbsmi{實} &
  ~ & \textbf{shin jitsu}  &
  ~ & \textrm{wirklich \emph{[und]} {\tiny ($\rightarrow$)}} \\  
{\tiny\texttt{126}}&
  ~ & \cnbsmi{不} \cnbsmi{虛} \cnbsmi{。}&
  ~ & \textbf{fu ko} &
  ~ & \textrm{nicht unwahr \emph{[ist,]}} \\
{\tiny\texttt{127}}&
  \cnbsmi{故} & ~ &
  \textbf{ko} & ~ &
  \textrm{darum} & \textrm{\emph{[wird die]} }\\  
{\tiny\texttt{128}}&
  ~ & \cnbsmi{說} & 
  ~ & \textbf{setsu} &
  ~ & \textrm{Bedeutung \emph{[der]}} \\  
 {\tiny\texttt{129}}&
  ~ & \cnbsmi{般} \cnbsmi{若} &
  ~ & \textbf{han nya} &
  ~ & \textrm{Prajñā} \\  
{\tiny\texttt{130}}&
  ~ & \cnbsmi{波} \cnbsmi{羅} \cnbsmi{蜜} \cnbsmi{多}&
  ~ & \textbf{ha ra mi ta} &
  ~ & \textrm{Pāramitā} \\ 
{\tiny\texttt{131}}&
  ~ & \cnbsmi{咒} &
  ~ & \textbf{shu}. & 
  \textrm{\emph{[als]}} & \textrm{Mantra} \\
{\tiny\texttt{132}}&
  \cnbsmi{即} & ~ & 
  \textbf{soku} & ~ &
  \multicolumn{2}{l}{\textrm{eigentlich \emph{[auch durch die]}}} \\  
{\tiny\texttt{133}}&
  ~ & \cnbsmi{說} &
  ~ & \textbf{setsu} & 
  ~ & \textrm{Bedeutung \emph{[des nun}}\\
{\tiny\texttt{134}}&
  ~ & \cnbsmi{咒} &
  ~ & \textbf{shu}  & 
  ~ & \textrm{\emph{folgenden]} Mantras} \\
{\tiny\texttt{135}}&
  ~ & \cnbsmi{曰} &
  ~ & \textbf{watsu} & 
  ~ & \textrm{ausgesagt:} \\ 
\hline
\end{tabular}

\begin{tabular}{r|rl|rl|rl}
\hline
  ~ & \multicolumn{6}{l}{\textsc{\ldots in Form eines Mantras:}}\\
\hline
{\tiny\texttt{136}}&
  ~ & ~ & ~ & ~ & ~ & \textrm{\emph{Lasst uns}} \\
{\tiny\texttt{137}}&
  ~ & \cnbsmi{羯} \cnbsmi{諦}&
  ~ & \textbf{gya tei} &
  ~ & \textrm{hinübergehen,} \\  
{\tiny\texttt{138}}&
  ~ & \cnbsmi{羯} \cnbsmi{諦}&
  ~ & \textbf{gya tei} &
  ~ & \textrm{hinübergehen,} \\  
{\tiny\texttt{139}}&
  \cnbsmi{波} \cnbsmi{羅} & \cnbsmi{羯} \cnbsmi{諦}&
  \textbf{ha ra} & \textbf{gya tei} &
  \multicolumn{2}{l}{\textrm{mit anderen hinübergehen,}} \\
{\tiny\texttt{140}}&
  \cnbsmi{波} \cnbsmi{羅} \cnbsmi{僧} & \cnbsmi{羯} \cnbsmi{諦}&
  \textbf{ha ra sō} & \textbf{gya tei} &
  \multicolumn{2}{l}{\textrm{mit anderen vollständig hinübergehen,}} \\
 \hdashline
 {\tiny\texttt{141}}&
  \cnbsmi{菩} \cnbsmi{提} \cnbsmi{薩} & \cnbsmi{婆} \cnbsmi{訶}&
  \textbf{bo ji} & \textbf{so wa ka} &
  \textrm{\emph{auf dem}} & \textrm{Weg \emph{zur} Vollendung.} \\
 \hline
   ~ & \multicolumn{6}{l}{\textsc{Punkt}}\\
 \hline
 {\tiny\texttt{142}}& 
  ~ & \cnbsmi{般} \cnbsmi{若} &
  ~ & \textbf{han nya} &
  \textrm{\emph{[So die]}}& \textrm{prajñā \emph{, Weisheit}}\\
{\tiny\texttt{143}}& 
  ~ & \cnbsmi{心} \cnbsmi{經} &
  ~ & \textbf{shin gyō}. & 
  \textrm{\emph{[als]}} & \textrm{essentielles Sutra} \\
 \hline 
%\end{longtable}
\end{tabular}
\end{center}
\rmfamily

\section{Die Gestaltung} 

In der linken Spalte meiner Lernversion des Hannya Shingyos steht der
chinesische Text. Er folgt dem universitär abgesicherten Text von
Scheid\footcite[vgl.][\nopage]{Scheid2016a} und ist -- entsprechend der
europäischen Tradition -- von links nach rechts und von oben nach unten zu
lesen. Er unterscheidet sich von den chinesischen Versionen, die die anderen
hier zitierten Autoren präsentieren, höchstens in der Punktion.

Die mittlere Spalte meiner Lernversion präsentiert den japanischen Text in
europäischer Umschrift. Sie folgt -- mit drei Ausnahmen -- dem Text von
Deshimaru\footcite[vgl.][30]{Deshimaru1988a} und ist ebenfalls von links nach
rechts und von oben nach unten zu lesen. Die erste Ausnahme betrifft das Wort
\emph{bo sa} in Zeile [006]. Hier steht bei Deshimaru \emph{bo satsu}. Die
zweite Ausnahme betrifft das Wort \emph{ze} in Zeile [061]. Hier steht bei
Deshimaru \emph{ze(tsu)}. Da in der Sangha, zu der ich mich hingezogen
fühle\footcite[vgl.][\nopage]{DaiShinZen2016a}, die Silbe \emph{tsu} nicht
gesprochen wird, habe ich mir erlaubt, es in meiner Lernversion zu unterdrücken.
Inhaltlich entsteht dadurch keine Veränderung, phonetisch nur eine geringe: das
auslautend \emph{u} wird im Japanischen fast nicht gesprochen, jedenfalls noch
weniger als das deutsche Auslaut-e in \emph{Stange} oder \emph{Karte}. Die
dritte Ausnahme betrifft die Groß- und Kleinschreibung: ich habe die konsequente
Kleinschreibung der Version von Scheid übernommen. Die Großschreibung nach einem
Punkt signalisiert harte syntaktische Abschlüsse, die semantisch so nicht
stimmen.

Meine Übersetzung ins Deutsche folgt in der Regel der anregenden, wortweisen
Übersetzung von Boeck\footcite[vgl.][\nopage]{Boeck2016a}, allerdings im
Abgleich mit den Erläuterungen von Deshimaru und Scheid. Mein eigenes Zutun
wollte von Anfang an nicht mehr bieten als eine geschickte Anordnung, bei der
eine möglichst wortgetreue Übersetzung zeilenmäßig in der Nähe der zu
übersetzenden Phrase bleibt. Das Hannya Shingyo sollte in sinnhaften Einheiten
lernbar gemacht werden. Um das zu erreichen, habe ich die großen syntaktischen
Freiheiten der deutschen Sprache genutzt: im Zweifel habe ich die etwas
geschrobenere Formulierung mit genauer Zuordnung der eleganteren, aber
entfernenden vorgezogen.

Um meine eher syntaktisch motivierten Zutaten als solche zu kennzeichnen, habe
ich sie in eckige Klammern eingeschlossen und kursiv gesetzt. Der deutsche Text
sollte sich mit diesen Zutaten schlüssig von links nach rechts und oben nach
unten lesen lassen. Unmarkierte deutsche Wörter sollten in der Zeile stehen, in
denen auch die chinesischen und japanischen Korrelate stehen - jedoch nicht
immer in derselben Reihenfolge, wie die Originale.

Und noch zwei letzte typographische Aufschlüsselung: 

\begin{enumerate}
  \item Die chinesische Schrift ist eine Begriffsschrift. Trotzdem enthält sie
  auch syntaktische Konnektoren, etwa die Negationen \emph{mu} (= \cnbsmi{無})
  und  \emph{fu} (= \cnbsmi{不}), die additive Konjunktion \emph{yaku} (=
  \cnbsmi{亦} = auch), die einfache Schlussfolgerung \emph{ko} (= \cnbsmi{故} =
  darum) oder die betonte Schlussfolgerung \emph{nai shi} (= \cnbsmi{乃}
  \cnbsmi{至} = darum insbesondere)\footnote{Boeck übersetzt \emph{fu} mit der
  deutschen Vorsilbe \emph{un-} und \emph{mu} mit der expliziten Negation
  \emph{nicht}. \emph{yaku} übersetzt er ebenfalls als auch. \emph{ko} übersetzt
  er wörtlich als \emph{Ursache}. Und \emph{nai shi} übersetzt er als \emph{dann
  extrem}, was ich als \emph{darum inbesondere}
  übernehme.\cite[vgl][\nopage]{Boeck2016a}}. Diese Patikel strukturieren den
  Text logisch. Deshalb habe ich sie in der linearen Anordnung jeweils nach
  links herausgezogen. Im selben Sinne habe ich auch einige andere, gliedernde
  Partikel optisch arrangiert.
  \item Im Text erscheint gelegentlich ein verweisender Pfeil
  \emph{$\rightarrow$}. Zu diesen Zeilen gibt es eine Erläuterung der
  Übersetzung. Die Zeilennummern werden im Kapitel mit den Übersetzungshinweisen
  als Referenz benutzt.
\end{enumerate}

\section{Die Übersetzung} 

Einige Entscheidungen habe ich im folgenden erläutert. Mit ist natürlich klar,
dass eine wirklich wissenschaftliche Aufbereitung viele Aspekte und Behauptungen
nachweisen müsste, auf die ich hier ohne Nachweis zurückgreife. Sie sind das
Ergebnis der Arbeit der anderen Autoren. Ihnen gebührt dafür Respekt,
Anerkennung und Dank, nicht mir. In einer späteren Version werde ich die
Nachweise sicher nachholen. Bis dahin möge man mir nachsehen, dass ich einfach
nur eine besser zu lernende Version erstellen wollte.

\begin{description}

  \item[001-003:] Das Hannya Shingyo ist ursprünglich in Sanskrit geschrieben,
  von dort ins Chinesische übertragen und von da aus ist es dann noch einmal ins
  Japanische übersetzt worden. Das Chinesische selbst ist eine Begriffsschrift,
  sodass sich die Übersetzung ins Japanische auf die Definition einer 'anderen'
  Aussprache konzentrieren konnte. Allerdings hatte die chinesische Version
  einige ursprüngliche Formulierung als 'wörtliche Zitate' bewahrt. Dabei ist
  die Aussprache des Sanskrit mit chinesischen Silben lautlich nachgebildet
  worden. Die Übertragung ins Japanische hat diese Idee übernommen.
  Damit entsteht jedoch eine 'Doppeldeutigkeit'. Denn die das Sanskrit mehr oder
  minder gut nachbildenden japanischen Wörter und Silben haben natürlich eine
  eigene unabhängige Bedeutung. Dem entsprechend wird gelegentlich gesagt, die
  Übertragungen hätten die Bedeutung des Hannya Shingyos
  \enquote{vertieft}\footcite[vgl.][56]{Deshimaru1988a}. Das \emph{Hannya
  Shingyo} als Name des Textes ist jedenfalls das erste Zitat aus dem Sanskrit.

  \item[005-006:] Der Ausdruck \emph{kan ji zai bo sa} bildet auch ein solches
  lautliches Zitat, allerdings in etwas \emph{verschleierter Form}: er soll den
  Ausdruck \emph{Boddhisattva Avalokiteshvara} wiedergeben. Dabei beziehen sich
  die Silben \emph{bo sa} direkt auf auf den Titel \emph{Boddhisattva}.
  Titelträger ist im Original \emph{Avalokitesvara}, ein Schüler von Buddha.
  Dieser hat einen Beinamen gehabt, auf den sich die Silben \emph{kan} (=
  \emph{beobachten}) und \emph{ji zai} (= \emph{Freiheit})  beziehen . Darum
  kann man den Namen nicht unübersetzt in einen deutschen Text übernehmen: es
  wird hier eben nicht über eine konkrete Einzelperson gesprochen. Vielmehr
  fungiert diese konkrete Person als Typus. Die so verallgemeinerte Aussage
  erlaubt es dem Hörer, sich einbezogen zu fühlen. Um das im Deutschen
  nachzubilden, nutze ich den unbestimmten Artikel und folge ansonsten der
  Deutung von Deshimaru\footcite[vgl.][57 et passim]{Deshimaru1988a}.

  \item[004,011:] \emph{ji} (= \cnbsmi{時}) soll \emph{Zeit} bedeuten und wird
  als Konjunktion zumeist mit \emph{als} oder \emph{während} übersetzt. Im
  deutschen kennen wir zwei Arten der 'zeitlichen' Verbindung zweier Fakten. Die
  eine betont eher die Zufälligkeit, die andere die Ursächlichkeit:
  \emph{\underline{als} ich Zucker aß, bekam ich Kopfschmerzen} meint etwas
  anderes als, \emph{\underline{indem} ich Zucker aß, bekam ich Kopfschmerzen}.
  Im Hannya Shingyo ist eine ursächliche Verknüpfung gemeint: \emph{Das
  Praktizieren der Höchsten Wahrheit führt zu der Erkenntnis, dass \ldots}. Das
  Wort \emph{indem} markiert diese ursächliche Beziehung gut.

  \item[014:] Die 5 Skandhas -- nämlich \emph{Empfindung, Wahrnehmung, Gedanken,
  Handeln und Bewusstsein} -- bilden eine zentrale Achse des Textes:
  zuerst wird ihr Oberbegriff \emph{go on} (= \cnbsmi{五} \cnbsmi{蘊}) genannt.
  Danach wird von jeder einzelnen gesagt, sie sei nicht nur nicht getrennt von
  \emph{kū}, sondern sie sei \emph{kū} (020-039). Schließlich wird auch gesagt,
  dass es sie in \emph{kū} ansich nicht gäbe (050-054), genauso wenig, wie
  entsprechenden Organe (055-060) oder deren Resulte (061-66). Dem liegt ein
  Weltbild zugrunde, das sicher nicht mehr unseres ist. Deshalb ist es
  angemessen, den fremden Begriff 'Skandha' als \emph{Fremdwort} in die
  Übersetzung zu übernehmen. Allerdings: die Pointe des Hannya Shingyos, dass es
  das, was dieses fremde Weltbild beschreibt, in \emph{kū} nicht gäbe, ließe
  sich umstandlos auch mit unserem heutigen physisch / psychischen Weltbild
  formulieren. Man muss sich also die 'veraltete' Sichtweise nicht zu eigen
  machen, um das Hannya Shingyo zu verstehen und seine Aussage zu bejahen. Das
  Hannya Shingyo ist -- so gesehen -- sehr modern.
  
  \item[021:] Es ist üblich, \emph{shiki} mit \emph{Form} zu übersetzen. Das
  wird der rhetorischen Form des Textes aber nicht gerecht: \emph{shiki} ist die
  erste der 5 Skandas. Die anderen 4 werden in den Zeilen [035-038] aufglistet.
  Die Übersetzung von \emph{shiki} muss auch das 1. Skandha schon als Teil einer
  Reihe erscheinen lassen. Dazu eignet sich das Wort \emph{Form} nicht.

  \item[021-033:] Außerdem wird diese ganze Sentenz gelegentlich zu der Aussage
  verknappt, \emph{Form sei Leere und Leere sei Form}. Damit geht eine -- auch
  rhetorisch entscheidende -- Pointe des Originals verloren: Zuerst sagt das
  Hannya Shingyo, \emph{shiki}, die \emph{Erscheinung} sei nicht getrennt von
  \emph{kū}. Dies muss den Hörer verwirren. Denn das normale Verständnis besagt
  doch wohl eher, dass es sich dabei um verschiedene Dinge handelt. Und mit
  diesem Erwartungshorizont spielt der Text. Denn er setzt danach -- sozusagen
  -- 'noch eins drauf': Er verschäft die Situation, in dem er sagt, dass die
  \emph{Erscheinung} und \emph{kū} nicht nur nicht getrennt seien, sondern dass
  das eine realiter auch das andere \emph{sei}. Rhethorisch gesehen präsentiert
  das Hannya Shingyo also zuerst eine 'steile' These, die es im folgenden wird
  erläutern und begründen müssen. Auf jeden Fall -- und das ist der rednerische
  Zweck dieses Vorgehens -- hat es mit dieser Konstruktion die Aufmerksamkeit
  seiner Hörer geweckt. Darum ist es notwendig, diese rhethorische Verschärfung
  auch in der Übersetzung zu erhalten.
  
  \item[022:] Oft wird \emph{i} mit \emph{verschieden} übersetzt. Das wird dem
  Original nicht gerecht. Denn tatsächlich geht es im folgenden Text [044-087],
  in dem \emph{kū} ex negativo definiert wird, um nichts anderes, als die
  Feststellung von Unterschieden.  Die Pointe des Hannya Shingyos ist aber, dass
  \emph{kū} trotz aller Verschiedenartigkeit dennoch -- irgendwie -- mit den 5
  Skandhas zusammenfällt, also trotz aller Verschiedenartigkeit nicht getrennt
  ist von \emph{shiki}. Darum habe ich mich für das Übersetzung \emph{getrennt}
  entschieden; es unterstreicht die intellektuelle Brisanz des Hannya Shingyos.

  \item[023ff:] Es ist üblich, \emph{kū} mit dem Wort \emph{Leere} zu
  übersetzen. Allerdings bringt das Wort \emph{Leere} eigene Konnotationen mit,
  die dem eigentlich Gemeinten entgegenstehen. Das Problem schillernder Begriffe
  kennt pikanterweise sogar das Hannya Shingyo selbst, mehr noch: es spielt
  sogar mit dem Phänomen: Es nimmt nämlich einen dem Gemeinten nahestehenden,
  vermeintlich klaren Begriff \emph{kū} und schärft diesen mittels Aussagen
  darüber, was das Gemeinte alles \emph{nicht} ist. Solch ein Verfahren nennt
  man eine \emph{Ex-Negativo-Definition}. Tatsächlich besteht das Hannya Shingyo
  im Kern aus einer Liste von negierenden Abgrenzungen [044-087]. Aus diesem
  Grund ist es besser, nicht das auch durch die europäische Philosophie
  aufgeheizte Wort \emph{Leere} durch vielfache Wiederholgungen zum Kern zu
  machen, sondern das Original -- also \emph{kū} --  zu verwenden und dessen
  Bedeutung gerade über Negationen klarwerden zu lassen.

  \item[050:] \emph{ze ko} (= \cnbsmi{是} \cnbsmi{故}) steht für \emph{sein
  Ursache}. Während ich später in Zeile [095ff] \emph{ko} konsequent als
  \emph{darum} übersetze, um den repitiven Charakter zu erhalten, wähle ich hier
  - zu Beginn der Deduktion - das stärkere und elegantere \emph{mithin} als
  Übersetzung.
  \item[072:] \emph{nai shi} besagt für sich genommen \emph{dann extrem}. Es
  geht also um eine besonders wichtige Schlussfolgerung. Solch ein sprachliches
  Konstrukt kennen wir auch im Deutschen, nämlich die einleitende Formel:
  \emph{Darum ist/wird/\ldots inbesondere \ldots}.

  \item[071-073:] Die Kombination \emph{gen kai} (= \cnbsmi{眼} \cnbsmi{界}) steht
  wörtlich für \emph{[Auge Welt]}, die Sequenz \emph{i shiki kai} (= \cnbsmi{意}
  \cnbsmi{識} \cnbsmi{界}) hingegen für \emph{Denkvermögen Unterscheiden Welt}.
  Erstere meint also die sichtbare, die erscheinende Welt, letztere die Welt der
  trennenden Vorstellungen und Konzepte. Auch in dieser Gegenüberstellung trifft
  man indirekt die fünf Skandas wieder: Zeile [058] hat schon \emph{gen} (= das
  \emph{Auge}) dem ersten Skandha \emph{shiki} (= \cnbsmi{色} =
  \emph{Erscheinen}) aus Zeile [053] als Organ zugeordnet. Für das fünfte
  Skandha, das Unterscheiden als intellektuelles Tun -- japanisch ebenfalls
  \emph{shiki} genannt -- wird ein anderes Zeichen benutzt als für das erste
  Skandha, nämlich \cnbsmi{識} (Zeile [057]. Und eben dieses zweite \emph{shiki}
  erscheint auch in Zeile [073]. Die rhetorische Konstruktion 'von \emph{gen
  kai} bis \emph{shiki kai} spannt also indirekt erneut den ganzen Bogen über
  alle fünf Skandhas auf.

  \item[074-080:] Die rhetorische Konstruktion \emph{Es gibt in kū nicht XYZ}
  und \emph{Es gibt in kū kein Ende von XYZ} ist besonders aufreizend für
  (europäische) Logiker: Ersteres negiert die Existenz von XYZ; letzteres setzt
  seine Existenz voraus und betont diese durch den impliziten Hinweis auf seine
  Ewigkeit, ausgedrückt durch eine doppelte Verneinung. Damit widersetzt sich
  das Hannya Shingyo der formalen Logik, in dem es dem europäischen Verständnis
  sein \emph{tertium datur} entgegenstellt, nicht ohne diese Logik allerdings
  selbst souverän zu benutzen. Dem Zen entsprechend ist das kein Widerspruch,
  sondern geradezu der Sinn allen Tuns: alle gedanklichen Konstrukte müssen
  aufgehoben werden, wenn \emph{kū} selbst im  Akt der Erleuchtung erfahrbar
  werden soll.

  \item[111-113:] Auch die Sentenz \emph{anokutara sanmyaku sanbodai} ist eine
  zitierende Sanskritnachahmung und meint \emph{höchste, vollkommene
  Erleuchtung}\footcite[vgl.][\nopage Anm. 10]{Scheid2016a}. Welches der Worte
  was bedeutet, habe ich den Quellen bisher nicht entnehmen können. Meine
  Zuordnung ist also willkürlich, folgt aber der Tradition.

  \item[115-121:] Hier findet eine rhetorisch geniale Umdeutung statt, die eine
  große Auswirkung auf den Buddhismus hat: Bisher war der Begriff \emph{han nya
  ha ra mi ta} beschreibend. Er stand für die \emph{die höchste Weisheit, die
  über sich hinausführt}. Jetzt wird der Ausdruck zum Namen des Textes selbst:
  indem er mehrfach als herausgehobenes \emph{Mantra} bezeichnet wird,
  verschiebt sich seine Bedeutung: der Terminus \emph{han nya ha ra mi ta} wird
  zum Namen des Textes. Und in dem diesem dann auch noch eine Wirkung
  zugesprochen wird, wird seine Rezitation zu einem Mittel. Kein Wunder also,
  dass alle Buddhisten diesen Text rezitieren: es steckt in ihm selbst.
 
  \item[119:] Die Phrase \emph{mu jō shu} verwendet wieder einmal eine der im
  \emph{Hannya Shingyo} so gern genutzte 'negative Zuschreibungen':
  \emph{mu} (= \cnbsmi{無}) ist die bekannte Verneinigung; und \emph{shu} (=
  \cnbsmi{咒}) steht für das \emph{Mantra}. Also wird \emph{jō} (= \cnbsmi{上})
  ein Attribut sein, das negiert dem Objekt \emph{Mantra} zugsprochen wird:
  Ein chinesisch-deutsches Internetlexikon sagt, das \cnbsmi{上} auch für
  \emph{von unten nach oben, aufwärts} bzw. \emph{vorwärts gehen}
  steht\footcite[vgl.][\nopage]{babla2016a}. Eine gute Übersetzung würde auch an
  dieser Stelle -- auf der Basis dieser Primärbedeutung -- die bevorzugte
  Methode der Eingrenzung ohne direkte Spezifikation bewahren; sie würde diese
  'ZEN gemäße' Art des 'Denkens' auch hier verdeutlichen. Hier fehlt mir noch
  eine gute Idee für die Umsetzung.

  \item[125-126:] \emph{shin jitsu}(= \cnbsmi{真} \cnbsmi{實}) soll
  \emph{Realität} meinen, und \emph{fu ko} (= \cnbsmi{不} \cnbsmi{虛} ) für
  \emph{nicht/keine Unwahrheit} stehen. Ersteres übersetze ich mit
  \emph{wirklich}, letzteres müsste dann \emph{wahr} heißen. Ich belasse
  letzteres aber bei \emph{nicht unwahr}, um die Neigung des Hannya Shingyos zur
  (doppelten) Verneinung zu erhalten.
\end{description}

% insert the bibliographical data here
\bibliography{bibfiles/hsResourcesDe}

\end{document}
 }

\maketitle
%%-- end(titlepage)
\section{Der Anlass}
 
Wäre es nicht schön, das \emph{Hannya Shingyō} -- mit anderen zusammen -- auch
auswendig vortragen zu können? Immerhin hat die Retization dieses Textes im
(Zen)-Buddhismus eine große Tradition!

Der Weg zum flüssigen Mitsprechen ist holprig: Wie lernt man solch eine sperrige
Folge japanischer Silben, wie einen so erratischen Textblock? Das Lernen dürfte
leichter fallen, wenn eine Struktur erkennbar wäre, etwa in einer
mehrspaltigen, mehrsprachigen, sinnhaft gegliederten Aufbereitung.

Dazu müsste der japanische Text jedoch recht wortgetreu übersetzt sein.
Denn nur so ließe sich die Übersetzung in einer Zeile mit dem übersetzten
Satzteil arrangieren. Würden die deutschen mit den japanisch-chinesischen
Phrasen so auch optisch korrespondieren, erschlössen sich die
Sinneinheiten direkt.

Trotzdem sollte die Übersetzung auch noch elegant sein: Das \emph{Hannya
Shingyō} ist ein Lehrtext, ein Sutra. Zuerst dürfte es mündlich vorgetragen
worden sein, als Ansprache an die Schüler. Mithin wird man darin -- ganz
sprachunabhängig -- auch rhetorische Elemente finden: Einen Interesse weckenden
\emph{Einstieg} etwa. Oder eine aufrüttelnde \emph{Kernthese}, die allmähliche
\emph{Entfaltung} ihrer Feinheiten, und die sich daran anschließende Begründung
der \emph{Konsequenzen}. Und natürlich einen einprägsamen \emph{Schluss}. Wäre es
nicht schön, wenn ein \emph{Hannya-Shingyō-Lerntext} auch das noch erkennen
ließe?

Gleichwohl müsste die Übertragung immer genau bleiben, von der Bedeutung und der
syntaktischen Struktur her\footnote{Doris Wolter hat dankenswerterweise
verschiedene Übersetzungen ins Deutsche zusammengetragen.
(\cite[vgl.][\nopage]{Wolter2010a}) Vergleicht man diese Versionen, offenbaren
sich erhebliche Unterschiede. Insbesondere das letzte Drittel des \emph{Hannya
Shingyōs} scheint dabei zu 'poetischen' Übertragungen einzuladen.
Angesichts der existentiellen philosophischen Dimension des Zen-Buddhismus im
Allgemeinen und des \emph{Hannya Shingyō} im Besonderen ist das schlicht
unzufriedenstellend.}. Sie sollte so wenig als möglich interpretieren.

Es gibt wunderbare Übersetzungen: z.B. die von
Deshimaru\footcite[vgl.][]{Deshimaru1988a}, die eher ein philosophischer
Hintergrundbericht sein will, als eine pure Übersetzung. Oder die universitär
abgesicherte, elegante Übertragung von Scheid\footcite[vgl.][]{Scheid2016a}.
Oder die wortgetreue von Boeck\footcite[vgl.][]{Boeck2016a}.

Nur liefern sie alle leider keinen mehrsprachigen, sinnhaft gegliederten
Lerntext, der bei aller Worttreue auch noch die elegante Rhetorik des Originals
erahnen ließe. Wie wäre es also mit folgender Variante?

\newpage
\section{Das Hannya Shingyo als Lerntext} 

\sffamily

\begin{center}
\begin{tabular}{r|rl|rl|rl}
~ & \multicolumn{6}{l}{\textsc{Der Titel:}}\\
\hline
{\tiny\texttt{001}}&
  \multicolumn{2}{l|}{\cnbsmi{摩}  \cnbsmi{訶} \cnbsmi{般} \cnbsmi{若}} &
  \multicolumn{2}{l|}{\textbf{ma kā}  \textbf{han nya}} &
  \textrm{\emph{[Die]}}& \textrm{maha prajñā \emph{(= höchste Weisheit,)}}\\
{\tiny\texttt{002}}&
  ~ & \cnbsmi{波} \cnbsmi{羅} \cnbsmi{蜜} \cnbsmi{多} & 
  ~ & \textbf{ha ra mi tā} & 
  {\tiny \textrm{($\rightarrow$)}} & 
    \textrm{pāramitā \emph{(die über sich hinausführt,)}}\\
{\tiny\texttt{003}}& 
  ~ & \cnbsmi{心} \cnbsmi{經} &
  ~ & \textbf{shin gyō} & 
  \textrm{\emph{[als das]}} & \textrm{essentielle Sutra \emph{[schlechthin]}} \\
\hline
~ & \multicolumn{6}{l}{\textsc{Das Manifest:}}\\
\hline
{\tiny\texttt{004}}& 
~ & ~  & ~ & ~ &
  \textrm{Indem} {\tiny \textrm{($\rightarrow$)}} & \textrm{\emph{[ein der]}} \\
{\tiny\texttt{005}}&
  ~ & \cnbsmi{觀} \cnbsmi{自} \cnbsmi{在} & 
  ~ & \textbf{kan ji zai} & 
  {\tiny \textrm{($\rightarrow$)}} &
    \textrm{freien Sicht \emph{[zugewandter]}} \\
{\tiny\texttt{006}}&
  ~ & \cnbsmi{菩} \cnbsmi{薩} \cnbsmi{。}& 
  ~ & \textbf{bo} \textbf{sa}\tiny{tsu}\textbf{.}& 
  ~ & \textrm{\emph{[lebender Buddha = ein]} Bodhisattva} \\  
{\tiny\texttt{007}}& 
  ~ & \cnbsmi{行} \cnbsmi{深} &
  ~ & \textbf{gyō} \textbf{jin} & 
  ~ & \textrm{tief \emph{[und gründlich]} praktizierend} \\  
{\tiny\texttt{008}}& 
  ~ & \cnbsmi{般} \cnbsmi{若} & 
  ~ & \textbf{han nya} & 
  ~ & \textrm{\emph{[die]} Prajñā \emph{(= Weisheit,)}} \\  
{\tiny\texttt{009}} &
  ~ & \cnbsmi{波} \cnbsmi{羅} \cnbsmi{蜜} \cnbsmi{多}& 
  ~ & \textbf{ha ra mi ta} & 
  ~ & \textrm{Pāramitā \emph{(die über sich hinausführt,)}} \\  
{\tiny\texttt{010}}&
  ~ & ~  & ~ & ~ &  ~ & \textrm{\emph{[lebt]}} \\
{\tiny\texttt{011}}&
  \cnbsmi{時}&\cnbsmi{。} &
  \textbf{ji}. & ~ &
  {\tiny \textrm{($\rightarrow$)}} & ~ \\
{\tiny\texttt{012}}& 
  ~ & ~ & ~ & ~ & ~ & \textrm{\emph{[kommt es bei ihm zum]}} \\
{\tiny\texttt{013}}& 
  ~ & \cnbsmi{照} \cnbsmi{見} &
  ~ & \textbf{shō ken} &
  ~ & \textrm{erleuchteten Sehen \emph{[, dass die]}} \\  
{\tiny\texttt{014}}& 
  ~ & \cnbsmi{五} \cnbsmi{蘊} & 
  ~ & \textbf{go on} & 
  {\tiny \textrm{($\rightarrow$)}} & \textrm{5 Skandas}  \\
{\tiny\texttt{015}}&
  ~ & \cnbsmi{皆} \cnbsmi {空} \cnbsmi{。} &
  ~ & \textbf{kai kū}. & 
  ~ & \textrm{alle leer \emph{[sind]}} \\
{\tiny\texttt{016}}&
  \cnbsmi{度} & ~ &
  \textbf{do} & ~ &
  \textrm{\emph{[und]} so} & ~ \\
{\tiny\texttt{017}}&
  ~ & \cnbsmi{一} \cnbsmi{切} &
  ~ & \textbf{is sai} &
  ~ & \textrm{entfernt \emph{[er]}} \\
{\tiny\texttt{018}}&
  ~ & \cnbsmi{苦} \cnbsmi{厄} \cnbsmi{。} & ~ &
  \textbf{ku yaku}. & 
  ~ & \textrm{Leiden \emph{[und]} Unheil.} \\
\hline
  ~ & \multicolumn{6}{l}{\textsc{Die Kernthese:}}\\
\hline
{\tiny\texttt{019}}&
  \multicolumn{2}{l|}{\cnbsmi{舍} \cnbsmi{利} \cnbsmi{子}\cnbsmi{。}}  &
  \multicolumn{2}{l|}{\textbf{sha ri shi}.} & ~ &
  \textrm{Shariputra!}\\
\hline  
{\tiny\texttt{020}}&
  ~ & ~ & ~ & ~ & 
  \multicolumn{2}{l}{\textrm{\emph{[Die 1. der 5 Skandas, nämlich die]}}} \\
{\tiny\texttt{021}}&
  ~ & \cnbsmi{色} & 
  {\tiny \textrm{($\rightarrow$)}} & \textbf{shiki} &
  ~ & \textrm{Erscheinung} \\  
{\tiny\texttt{022}}&
  \cnbsmi{不} & \cnbsmi{異} & 
  \textbf{fu} & \textbf{i} &
  \textrm{\emph{[ist]} nicht} & \textrm{getrennt \emph{[von]}} \\  
{\tiny\texttt{023}}&
  ~ & \cnbsmi{空} \cnbsmi{。} &
  {\tiny \textrm{($\rightarrow$)}} & \textbf{kū}.  &
  {\tiny \textrm{($\rightarrow$)}} & \textrm{kū, \emph{[der Leere]}} \\
\hdashline
{\tiny\texttt{024}}&
  ~ & \cnbsmi{空} &
  ~ & \textbf{kū} & 
  \textrm{\emph{[und]}} & \textrm{kū, \emph{[die Leere]}} \\
{\tiny\texttt{025}}&
  \cnbsmi{不} & \cnbsmi{異} &
  \textbf{fu} & \textbf{i} &
  \textrm{\emph{[ist]} nicht} & \textrm{getrennt \emph{[von]}} \\  
{\tiny\texttt{026}}&
  ~ & \cnbsmi{色} \cnbsmi{。} &
  ~ & \textbf{shiki}. &
  ~ & \textrm{\emph{[der]} Erscheinung.} \\
\hline
{\tiny\texttt{027}}&
  ~ & ~ & ~ & ~ & \multicolumn{2}{l}{\textrm{~\emph{Ja, mehr noch:}}}  \\  
{\tiny\texttt{028}}&
  ~ & \cnbsmi{色} &
  ~ & \textbf{shiki} & 
  ~ & \textrm{\emph{[Die]} Erscheinung} \\  
{\tiny\texttt{029}}&
  \cnbsmi{即} & \cnbsmi{是} & 
  \textbf{soku} & \textbf{ze} &
  \textrm{ist} & \textrm{eigentlich} \\  
{\tiny\texttt{030}}&
  ~ & \cnbsmi{空} \cnbsmi{。} &
  ~ & \textbf{kū}. &
  ~ & \textrm{kū, \emph{[die Leere]}} \\
 \hdashline
 {\tiny\texttt{031}}&
  ~ & \cnbsmi{空} &
  ~ & \textbf{kū} & 
  \textrm{\emph{[und]}} & \textrm{kū, \emph{[die Leere]}} \\
{\tiny\texttt{032}}&
  \cnbsmi{即} & \cnbsmi{是} & 
  \textbf{soku} & \textbf{ze} &
  \textrm{ist} & \textrm{eigentlich} \\  
{\tiny\texttt{033}}&
 ~ & \cnbsmi{色} \cnbsmi{。} &
 ~ & \textbf{shiki}. &~
 ~ & \textrm{\emph{[die]} Erscheinung.} \\
 \hline
 {\tiny\texttt{034}}&
    ~ & ~ & ~ & ~ & \multicolumn{2}{l}{
    \textrm{\emph{[Und bei den anderen 4 Skandas, also beim]}}}\\
 {\tiny\texttt{035}}&
  ~ & \cnbsmi{受} &
  ~ & \textbf{ju} &
  ~ & \textrm{Empfinden,} \\
{\tiny\texttt{036}}&
  ~ & \cnbsmi{想} &
  ~ & \textbf{sō} &
  ~ & \textrm{Wahrnehmen,} \\
 {\tiny\texttt{037}}&
  ~ & \cnbsmi{行} &
  ~ & \textbf{gyō} &
  ~ & \textrm{Wollen \emph{[und]}} \\
 {\tiny\texttt{038}}&
  ~ & \cnbsmi{識} &
  ~ & \textbf{shiki}. &
  ~ & \textrm{Unterscheiden}, \\
{\tiny\texttt{039}}&
  \cnbsmi{亦} & \cnbsmi{復} \cnbsmi{如} \cnbsmi{是} \cnbsmi{。} &
  \textbf{yaku} & \textbf{bu nyo ze}. &
  \textrm{auch \emph{[da]}} & \textrm{ist \emph{[es]} wieder gleich}. \\
\hline
  ~ & \multicolumn{6}{l}{\textsc{Die ex negativo Definition von \textrm{kū}:}}\\
\hline
{\tiny\texttt{040}} &
  \multicolumn{2}{l|}{\cnbsmi{舍} \cnbsmi{利} \cnbsmi{子} \cnbsmi{。}} &
  \multicolumn{2}{l|}{\textbf{sha ri shi}.} &
  ~ & \textrm{Shariputra!}\\
\hline
{\tiny\texttt{041}}&
  \cnbsmi{是} & \cnbsmi{諸} &
  \textbf{ze} & \textbf{sho} &
  \textrm{\emph{[Es]} ist} & \textrm{alles} \\
{\tiny\texttt{042}}&
  ~ & \cnbsmi{法} &
  ~ & \textbf{hō} &
  ~ & \textrm{Seiende} \\
{\tiny\texttt{043}}&
  ~ & \cnbsmi{空} \cnbsmi{相} \cnbsmi{。} &
  ~ & \textbf{kū sō}. &
  ~ & \textrm{\emph{[ein]} Aspekt \emph{[von]} kū}: \\
\hdashline
{\tiny\texttt{044}}&
  \cnbsmi{不} & \cnbsmi{生} &
  \textbf{fu} & \textbf{shō} & 
  \textrm{nicht} & \textrm{geboren \emph{[bzw.]} geschaffen}, \\
{\tiny\texttt{045}}&
  \cnbsmi{不} & \cnbsmi{滅} \cnbsmi{。} &
  \textbf{fu} & \textbf{metsu}. &
  \textrm{nicht} & \textrm{gestorben \emph{[bzw.]} ausgelöscht},\\
\hdashline
{\tiny\texttt{046}}&
  \cnbsmi{不} & \cnbsmi{垢} &
  \textbf{fu} & \textbf{ku} &
  \textrm{nicht} & \textrm{befleckt}, \\
{\tiny\texttt{047}}&
  \cnbsmi{不} & \cnbsmi{淨} &
  \textbf{fu} & \textbf{jō}. &
  \textrm{nicht} & \textrm{rein}, \\
\hdashline
\end{tabular}

\begin{tabular}{r|rl|rl|rl}
\hdashline
{\tiny\texttt{048}}&
  \cnbsmi{不} & \cnbsmi{增} &
  \textbf{fu} & \textbf{zō} &
  \textrm{nicht} & \textrm{zunehmend}, \\
{\tiny\texttt{049}}&
  \cnbsmi{不} & \cnbsmi{減} &
  \textbf{fu} & \textbf{gen}. &
  \textrm{nicht} & \textrm{abnehmend}. \\
\hline
{\tiny\texttt{050}}&
  \cnbsmi{是} & \cnbsmi{故} &
  \textbf{ze} & \textbf{ko} &
  \textrm{Mithin} & {\tiny ($\rightarrow$)} \textrm{gibt es}\\
{\tiny\texttt{051}}&
  ~ & \cnbsmi{空} \cnbsmi{中} \cnbsmi{。} &
  ~ & \textbf{kū chū}. &
  ~ & \textrm{in kū} \\
{\tiny\texttt{052}}&
  ~ & ~  & ~ & ~ &  ~ & \textrm{\emph{[keines der 5 Skandhas, also]}} \\
{\tiny\texttt{053}}&
  \cnbsmi{無} & \cnbsmi{色} \cnbsmi{。} &
  \textbf{mu} & \textbf{shiki} &
  \textrm{kein} & \textrm{Erscheinen}, \\
{\tiny\texttt{054}}&
  \cnbsmi{無} & \cnbsmi{受} &
  \textbf{mu} & \textbf{ju} & 
  \textrm{kein} & \textrm{Empfinden,} \\
{\tiny\texttt{055}}&
  ~ & \cnbsmi{想} &
  ~ & \textbf{sō} &
  ~ & \textrm{Wahrnehmen,} \\
{\tiny\texttt{056}}&
  ~ & \cnbsmi{行} &
  ~ & \textbf{gyō} &
  ~ & \textrm{Wollen \emph{[oder]}} \\
{\tiny\texttt{057}}&
  ~ & \cnbsmi{識} \cnbsmi{。} &
  ~ & \textbf{shiki}. & 
  ~ & \textrm{Unterscheiden}, \\
\hdashline
 {\tiny\texttt{058}}&
  \cnbsmi{無} & \cnbsmi{眼} &
  \textbf{mu} & \textbf{gen} &
  \textrm{keine} & \textrm{Augen,} \\
{\tiny\texttt{059}}&
  ~ & \cnbsmi{耳} &
  ~ & \textbf{ni} &
  ~ & \textrm{Ohren,} \\
{\tiny\texttt{060}}&
  ~ & \cnbsmi{鼻} &
  ~ & \textbf{bi} &
  ~ & \textrm{Nase,} \\
{\tiny\texttt{061}}&
  ~ & \cnbsmi{舌} &
  ~ & \textbf{ze}\tiny{tsu}&
  ~ & \textrm{Zunge,} \\
{\tiny\texttt{062}}&
  ~ & \cnbsmi{身} &
  ~ & \textbf{shin} &
  \textrm{\emph{[keinen]}} & \textrm{Tastsinn \emph{[und]}} \\
{\tiny\texttt{063}}&
  ~ & \cnbsmi{意} \cnbsmi{。} &
  ~ & \textbf{i}. & 
  \textrm{\emph{[kein]}} & \textrm{Denkvermögen}, \\
\hdashline
{\tiny\texttt{064}}&
   \cnbsmi{無} & \cnbsmi{色} &
   \textbf{mu} & \textbf{shiki} &
   \textrm{keine} & \textrm{Farbe,} \\
{\tiny\texttt{065}}&
   ~ & \cnbsmi{聲} &
   ~ & \textbf{shō} &
   \textrm{\emph{[keinen]}} & \textrm{Klang,} \\
{\tiny\texttt{066}}&
   ~ & \cnbsmi{香} &
   ~ & \textbf{kō} &
   ~ & \textrm{Geruch,} \\
{\tiny\texttt{067}}&
  ~ & \cnbsmi{味} &
  ~ & \textbf{mi} &
  ~ & \textrm{Geschmack,} \\
{\tiny\texttt{068}}&
  ~ & \cnbsmi{觸} &
  ~ & \textbf{soku} & 
  \textrm{\emph{[keine]}} & \textrm{Berührung \emph{[und]}} \\
{\tiny\texttt{069}}&
  ~ & \cnbsmi{法} \cnbsmi{。} &
  ~ & \textbf{hō}.&
  \textrm{\emph{[keinen]}} & \textrm{Gedanken}. \\
\hline
{\tiny\texttt{070}}&
  ~ & ~ & ~ & ~ & ~ & \textrm{\emph{[Also gibt es in kū]}} \\
\hdashline
{\tiny\texttt{071}}&
  \cnbsmi{無} & \cnbsmi{眼} \cnbsmi{界} \cnbsmi{。} &
  \textbf{mu} & \textbf{gen kai} &
  \textrm{nicht} & \textrm{die sichtbare Welt} {\tiny ($\rightarrow$)} \\
{\tiny\texttt{072}}&
  \cnbsmi{乃}\cnbsmi{至} & ~ & 
  \textbf{nai shi} & ~ &
  \textrm{\emph{[und]} {\tiny ($\rightarrow$)}} & 
   \textrm{darum insbesondere [auch]}\\
{\tiny\texttt{073}}&
  \cnbsmi{無} & \cnbsmi{意} \cnbsmi{識} \cnbsmi{界} \cnbsmi{。}&
  \textbf{mu} & \textbf{i shiki kai}.&
  \textrm{nicht} & 
    \textrm{die Welt der Vorstellungen {\tiny ($\rightarrow$)}}, \\
\hdashline
{\tiny\texttt{074}}&
  \cnbsmi{無} & \cnbsmi{無} \cnbsmi{明} \cnbsmi{。} &
  \textbf{mu} & \textbf{mu myō} &
  \textrm{kein} & \textrm{Nicht-Wissen \emph{[und]}} \\
{\tiny\texttt{075}}&
  \cnbsmi{亦} & ~ & 
  \textbf{yaku} & ~ &
  \textrm{auch} & ~ \\  
{\tiny\texttt{076}}&
  \cnbsmi{無} & \cnbsmi{無} \cnbsmi{明} \cnbsmi{盡} \cnbsmi{。} &
  \textbf{mu} & \textbf{mu myō jin.} &
  \textrm{kein} & \textrm{Ende vom Nicht-Wissen} {\tiny ($\rightarrow$)}, \\
\hdashline
{\tiny\texttt{077}}&
  \cnbsmi{乃}\cnbsmi{至} & ~ & 
  \textbf{nai shi} & ~ &
  \textrm{\emph{[und]} {\tiny ($\rightarrow$)}} & 
   \textrm{darum insbesondere \emph{[auch]}}\\
\hdashline
{\tiny\texttt{078}}&
  \cnbsmi{無} & \cnbsmi{老} \cnbsmi{死} \cnbsmi{。} &
  \textbf{mu} & \textbf{rō shi} &
  \textrm{kein} & \textrm{Altern und Tod \emph{[und]}} \\
{\tiny\texttt{079}}&
  \cnbsmi{亦} & ~ &
  \textbf{yaku} & ~ &
  \textrm{auch} & ~ \\
{\tiny\texttt{080}}&
  \cnbsmi{無} & \cnbsmi{老} \cnbsmi{死} \cnbsmi{盡} \cnbsmi{。} &
  \textbf{mu} &
  \textbf{rō shi jin}. &
  \textrm{kein} & \textrm{Ende von Altern und Tod {\tiny ($\rightarrow$)}}\\
\hline
{\tiny\texttt{081}}&
  \cnbsmi{無} & \cnbsmi{苦} & 
  \textbf{mu} & \textbf{ku} &
  \textrm{kein} & \textrm{Leiden,} \\  
{\tiny\texttt{082}}&
  ~ & \cnbsmi{集} &
  ~ & \textbf{shū} & 
  ~ & \textrm{Anhäufen,} \\  
{\tiny\texttt{083}}&
  ~ & \cnbsmi{滅} &
  ~ & \textbf{metsu} & 
  ~ & \textrm{Verlöschen \emph{[und]}} \\  
{\tiny\texttt{084}}&
  ~ & \cnbsmi{道} \cnbsmi{。} & 
  ~ & \textbf{dō}. &
  \textrm{\emph{[keinen]}} & \textrm{Weg,} \\
\hdashline 
{\tiny\texttt{085}}&
  \cnbsmi{無} & \cnbsmi{智} &
  \textbf{mu} & \textbf{chi} &
  \textrm{keine} & \textrm{Erkenntnis \emph{[und]}} \\  
{\tiny\texttt{086}}&
  \cnbsmi{亦} & ~ &
  \textbf{yaku} & ~ &
  \textrm{auch} & ~ \\
{\tiny\texttt{087}}&
  \cnbsmi{無} & \cnbsmi{得} \cnbsmi{。} &
  \textbf{mu} & \textbf{toku}. &
  \textrm{keinen} & \textrm{Gewinn}, \\  
{\tiny\texttt{088}}&
  \cnbsmi{以} & ~ &
  \textbf{i} & ~ &
  \textrm{weil} &  \textrm{\emph{[kū]}} \\  
{\tiny\texttt{089}}&
  \cnbsmi{無} & \cnbsmi{所} \cnbsmi{得}  &
  \textbf{mu} & \textbf{sho tok}\tiny{u}  &
  \textrm{kein} & \textrm{Ort \emph{[des]} Gewinnens \emph{[ist]}.} \\
\hline
  ~ & \multicolumn{6}{l}{\textsc{Die praktische Konsequenz:}}\\
\hline
{\tiny\texttt{090}}&
  \cnbsmi{故}\cnbsmi{。} & 
    \cnbsmi{菩} \cnbsmi{提} \cnbsmi{薩} \cnbsmi{捶}\cnbsmi{。}  &
  \textbf{ko}. & \textbf{bo dai sat ta.} &
  \textrm{Darum} & 
    \textrm{\emph{[gilt:] [Ein]} Bodhisattva \emph{[zu sein,]}}\\
{\tiny\texttt{091}}&
  ~ & \cnbsmi{依} &
  ~ & \textbf{e} & 
  ~ & \textrm{bedingt \emph{[die]}} \\
\newline
{\tiny\texttt{092}}&
  ~ & \cnbsmi{般} \cnbsmi{若}  &
  ~ & \textbf{han nya} &
  ~ & \textrm{Prajñā \emph{(= Weisheit,)}} \\  
{\tiny\texttt{093}}&
  ~ & \cnbsmi{波} \cnbsmi{羅} \cnbsmi{蜜} \cnbsmi{多} &
  ~ & \textbf{ha ra mi ta} &
  ~ & \textrm{Pāramitā \emph{(die über sich hinausführt)}.} \\
\hdashline
{\tiny\texttt{094}}&
  \cnbsmi{故}\cnbsmi{。} & 
    \cnbsmi{心} \cnbsmi{無} \cnbsmi{罫} \cnbsmi{礙} \cnbsmi{。} &
  \textbf{ko.} & \textbf{shin mu kei ge} &
  \textrm{Darum} & 
    \textrm{\emph{[wird sein]} Geist nicht behindert.} \\
\hdashline
{\tiny\texttt{095}}&
  ~ & \cnbsmi{無} \cnbsmi{罫} \cnbsmi{礙} & 
  ~ & \textbf{mu kei ge} & 
 \multicolumn{2}{l}
  {\textrm{\emph{[Und da der]} nicht behindert \emph{[wird]},}}\\
{\tiny\texttt{096}}&
  \cnbsmi{故}\cnbsmi{。} & \cnbsmi{無} \cnbsmi{有} & 
  \textbf{ko.} & \textbf{mu u }& 
  \textrm{darum} & \textrm{hat er \emph{[- der Bodhisattva -]} keine} \\
{\tiny\texttt{097}}&
  ~ & \cnbsmi{恐} \cnbsmi{怖}\cnbsmi{。} &
  ~ & \textbf{ku fu} &
  ~ & \textrm{Furcht}. \\ 
\hline
\end{tabular}

\begin{tabular}{r|rl|rl|rl}
\hdashline
{\tiny\texttt{098}}&
  ~ & \cnbsmi{遠} \cnbsmi{離} &
  ~ & \textbf{on ri} &
  \textrm{\emph{[Das]}} & \textrm{übersteigend\emph{[, was er sich]}} \\      
{\tiny\texttt{099}}&
  ~ & \cnbsmi{一} \cnbsmi{切} &
  ~ & \textbf{is sai} & 
  ~ & \textrm{entfernt \emph{[hat -- nämlich]} }\\
{\tiny\texttt{100}}&
  ~ & \cnbsmi{顛} \cnbsmi{倒} &
  ~ & \textbf{ten dō} &
  ~ & \textrm{Täuschungen \emph{[und]}} \\      
{\tiny\texttt{101}}&
  ~ & \cnbsmi{夢} \cnbsmi{想} \cnbsmi{。} &
  ~ & \textbf{mu sō}. &
  ~  & \textrm{Illusionen \emph{[--]}} \\      
{\tiny\texttt{102}}&
  ~ & \cnbsmi{究} \cnbsmi{竟} &
  ~ & \textbf{ku gyō} &
  ~ & \textrm{erreicht \emph{[er]} schließlich } \\      
{\tiny\texttt{103}}&
  ~ & \cnbsmi{涅} \cnbsmi{槃} \cnbsmi{。}&
  ~ & \textbf{ne han}. &
  ~  & \textrm{\emph{[das]} Nirvana.} \\      
\hline   
{\tiny\texttt{104}}&
  ~ & \cnbsmi{三} \cnbsmi{世} &
  ~ & \textbf{san ze} &
  \textrm{\emph{[Zudem]}} & \textrm{\emph{[gilt seit]} drei Zeitaltern} \\      
{\tiny\texttt{105}}&
  ~ & \cnbsmi{諸} \cnbsmi{佛} \cnbsmi{。} &
  ~ & \textbf{sho butsu} &
  \textrm{\emph{für}} & 
    \textrm{alle Buddhas: \emph{[ihre Buddhaschaft]}} \\     
 {\tiny\texttt{106}}&
  ~ & \cnbsmi{依} &
  ~ & \textbf{e} &
  ~ & \textrm{bedingt \emph{[die]}} \\  
{\tiny\texttt{107}}&
  ~ & \cnbsmi{般} \cnbsmi{若}  &
  ~ & \textbf{han nya} &
  ~ & \textrm{Prajñā \emph{(= Weisheit, )}} \\  
{\tiny\texttt{108}}&
  ~ & \cnbsmi{波} \cnbsmi{羅} \cnbsmi{蜜} \cnbsmi{多} &
  ~ & \textbf{ha ra mi ta} &
  ~ & \textrm{Pāramitā \emph{(die über sich hinausführt)}.} \\
\hline  
{\tiny\texttt{109}}&
  \cnbsmi{故}\cnbsmi{。} & \cnbsmi{得} &
  \textbf{ko}. & \textbf{toku} &
  \textrm{Darum} & \textrm{gewinnen sie die} \\  
{\tiny\texttt{110}}&
  ~ & \cnbsmi{阿} \cnbsmi{耨} \cnbsmi{多} \cnbsmi{羅} & 
  ~ & \textbf{a noku ta ra} &
  {\tiny ($\rightarrow$)} & \textrm{anuttara} \textrm{\emph{(= höchste)}} \\
{\tiny\texttt{111}}&
  ~ & \cnbsmi{三} \cnbsmi{藐} &
  ~ & \textbf{san myaku} &
  {\tiny ($\rightarrow$)} & \textrm{samyak} \textrm{\emph{(= vollkommene)}} \\      
{\tiny\texttt{112}}&
  ~ & \cnbsmi{三} \cnbsmi{菩} \cnbsmi{提} \cnbsmi{。} &
  ~ & \textbf{san bo dai}. &
  {\tiny ($\rightarrow$)} & \textrm{sambodhi} \textrm{\emph{(= Erleuchtung)}} \\ 
 \hline 
{\tiny\texttt{113}}&
  \cnbsmi{故} & \cnbsmi{知} &
  \textbf{ko} & \textbf{chi} &
  \textrm{Darum} &  \textrm{wisse \emph{[nun Du Deinerseits:]}} \\  
{\tiny\texttt{114}}&
  ~ & \cnbsmi{般} \cnbsmi{若} &
  ~ & \textbf{han nya} &
  \textrm{\emph{[Das]}} & \textrm{Prajñā} {\tiny ($\rightarrow$)} \\  
{\tiny\texttt{115}}&
  ~ & \cnbsmi{波} \cnbsmi{羅} \cnbsmi{蜜} \cnbsmi{多}&
  ~ & \textbf{ha ra mi ta}.&
  ~ & \textrm{Pāramitā} {\tiny ($\rightarrow$)} \\ 
{\tiny\texttt{116}}&
  \cnbsmi{是} & \cnbsmi{大} \cnbsmi{神} \cnbsmi{咒} \cnbsmi{。} &
  \textbf{ze} & \textbf{dai jin shu}. &
  \textrm{ist} & \textrm{\emph{[ein]} großes wunderbares Mantra}; \\  
{\tiny\texttt{117}}&
  \cnbsmi{是} & \cnbsmi{大} \cnbsmi{明} \cnbsmi{咒} &
  \textbf{ze} & \textbf{dai myō shu}. &
  \textrm{\emph{[es]} ist} & \textrm{\emph{[ein]} großes leuchtendes Mantra}, \\ 
{\tiny\texttt{118}}&
  \cnbsmi{是} & \cnbsmi{無} \cnbsmi{上} \cnbsmi{咒} \cnbsmi{。} &
  \textbf{ze} & \textbf{mu jō shu}. &
  \textrm{\emph{[es]} ist} & 
    \textrm{\emph{[das]} {\tiny ($\rightarrow$)} höchste Mantra} \\
{\tiny\texttt{119}}&
  \cnbsmi{是} & \cnbsmi{無} \cnbsmi{等} \cnbsmi{等} \cnbsmi{咒} \cnbsmi{。}  &
  \textbf{ze} &  \textbf{mu tō dō shu}.  &
  \textrm{\emph{[es]} ist} & \textrm{\emph{[das]} nicht übersteigbare Mantra} \\  
{\tiny\texttt{120}}&
  ~ & \cnbsmi{能} & 
  ~ & \textbf{nō} &  
  ~ \textrm{\emph{[es]}} & \textrm{dient \emph{[dem]}} \\
{\tiny\texttt{121}}&
  ~ & \cnbsmi{除} \cnbsmi{一} \cnbsmi{切} & 
  ~ & \textbf{jo is sai} & 
  ~ & \textrm{Beseitigen \emph{[und]} Abschneiden} \\
{\tiny\texttt{122}}&
  ~ & \cnbsmi{苦} \cnbsmi{。} & 
  ~ & \textbf{ku}. & 
 ~ & \textrm{\emph{[von]} Leiden.} \\
\hline 
  ~ & \multicolumn{6}{l}{\textsc{Das Fazit \ldots}}\\
\hline
{\tiny\texttt{123}}&
  ~ & ~ & ~ & ~ &  & \textrm{\emph{[Und weil dies]}} \\      
{\tiny\texttt{124}}&
  ~ & \cnbsmi{真} \cnbsmi{實} &
  ~ & \textbf{shin jitsu}  &
  ~ & \textrm{wirklich \emph{[und]} {\tiny ($\rightarrow$)}} \\  
{\tiny\texttt{125}}&
  ~ & \cnbsmi{不} \cnbsmi{虛} \cnbsmi{。}&
  ~ & \textbf{fu ko} &
  ~ & \textrm{nicht unwahr \emph{[ist,]}} \\
{\tiny\texttt{126}}&
  \cnbsmi{故} & ~ &
  \textbf{ko} & ~ &
  \textrm{darum} & \textrm{\emph{[wird die]} }\\  
{\tiny\texttt{127}}&
  ~ & \cnbsmi{說} & 
  ~ & \textbf{setsu} &
  ~ & \textrm{Bedeutung \emph{[der]}} \\  
 {\tiny\texttt{128}}&
  ~ & \cnbsmi{般} \cnbsmi{若} &
  ~ & \textbf{han nya} &
  ~ & \textrm{Prajñā} \\  
{\tiny\texttt{129}}&
  ~ & \cnbsmi{波} \cnbsmi{羅} \cnbsmi{蜜} \cnbsmi{多}&
  ~ & \textbf{ha ra mi ta} &
  ~ & \textrm{Pāramitā} \\ 
{\tiny\texttt{130}}&
  ~ & \cnbsmi{咒} &
  ~ & \textbf{shu}. & 
  \textrm{\emph{[als]}} & \textrm{Mantra} \\
{\tiny\texttt{131}}&
  \cnbsmi{即} & ~ & 
  \textbf{soku} & ~ &
  \multicolumn{2}{l}{\textrm{eigentlich \emph{[auch durch die]}}} \\  
{\tiny\texttt{132}}&
  ~ & \cnbsmi{說} &
  ~ & \textbf{setsu} & 
  ~ & \textrm{Bedeutung \emph{[des nun}}\\
{\tiny\texttt{133}}&
  ~ & \cnbsmi{咒} &
  ~ & \textbf{shu}  & 
  ~ & \textrm{\emph{folgenden]} Mantras} \\
{\tiny\texttt{134}}&
  ~ & \cnbsmi{曰} &
  ~ & \textbf{watsu} & 
  ~ & \textrm{ausgesagt:} \\ 
\hline
\end{tabular}


\begin{tabular}{r|rl|rl|rl}
\hline
  ~ & \multicolumn{6}{l}{\textsc{\ldots in Form eines Mantras:}}\\
\hline
{\tiny\texttt{135}}&
  ~ & ~ & ~ & ~ & ~ & \textrm{\emph{Lasst uns}} \\
{\tiny\texttt{136}}&
  ~ & \cnbsmi{羯} \cnbsmi{諦}&
  ~ & \textbf{gya tei} &
  ~ & \textrm{hinübergehen,} \\  
{\tiny\texttt{137}}&
  ~ & \cnbsmi{羯} \cnbsmi{諦}&
  ~ & \textbf{gya tei} &
  ~ & \textrm{hinübergehen,} \\  
{\tiny\texttt{138}}&
  \cnbsmi{波} \cnbsmi{羅} & \cnbsmi{羯} \cnbsmi{諦}&
  \textbf{ha ra} & \textbf{gya tei} &
  \multicolumn{2}{l}{\textrm{mit anderen hinübergehen,}} \\
{\tiny\texttt{139}}&
  \cnbsmi{波} \cnbsmi{羅} \cnbsmi{僧} & \cnbsmi{羯} \cnbsmi{諦}&
  \textbf{ha ra sō} & \textbf{gya tei} &
  \multicolumn{2}{l}{\textrm{mit anderen vollständig hinübergehen,}} \\
 \hdashline
 {\tiny\texttt{140}}&
  \cnbsmi{菩} \cnbsmi{提} \cnbsmi{薩} & \cnbsmi{婆} \cnbsmi{訶}&
  \textbf{bo ji} & \textbf{so wa ka} &
  \textrm{\emph{auf dem}} & \textrm{Weg \emph{zur} Vollendung.} \\
 \hline
   ~ & \multicolumn{6}{l}{\textsc{Punkt}}\\
 \hline
 {\tiny\texttt{141}}& 
  ~ & \cnbsmi{般} \cnbsmi{若} &
  ~ & \textbf{han nya} &
  \textrm{\emph{[So die]}}& \textrm{prajñā \emph{(= Weisheit)}}\\
{\tiny\texttt{142}}& 
  ~ & \cnbsmi{心} \cnbsmi{經} &
  ~ & \textbf{shin gyō}. & 
  \textrm{\emph{[als]}} & \textrm{essentielles Sutra} \\
 \hline 
%\end{longtable}
\end{tabular}
\end{center}
\rmfamily

\section{Die Gestaltung} 

In der linken Spalte meiner Lernversion des \emph{Hannya Shingyōs} steht der
chinesische Text. Er folgt dem universitär abgesicherten Text von
Scheid\footcite[vgl.][\nopage]{Scheid2016a} und ist -- entsprechend der
europäischen Tradition -- von links nach rechts und von oben nach unten zu
lesen. Er unterscheidet sich von den chinesischen Texten, die die anderen,
hier zitierten Autoren präsentieren, höchstens in der Punktion.

Die mittlere Spalte meiner Lernversion präsentiert den japanischen Text in
europäischer Umschrift. Sie folgt -- mit vier Ausnahmen -- dem Text von
Deshimaru\footcite[vgl.][30]{Deshimaru1988a} und ist ebenfalls von links nach
rechts und von oben nach unten zu lesen: Die erste Ausnahme betrifft das Wort
\emph{bo satsu} in Zeile [006], die zweite Ausnahme das Wort \emph{ze(tsu)} in
Zeile [061], und die dritte die Phrase \emph{mu sho toku} in Zeile [089]. In
diesen Fällen habe ich die Teile sehr klein gesetzt, die in der Sangha, zu der
ich mich hingezogen fühle\footcite[vgl.][\nopage]{DaiShinZen2016a}, nicht
gesprochen werden. Inhaltlich entsteht dadurch keine Veränderung, phonetisch nur
eine geringe: das auslautend \emph{u} wird im Japanischen fast nicht gesprochen,
jedenfalls noch weniger als das deutsche Auslaut-e in \emph{Stange} oder
\emph{Karte}\footnote{In meiner Sangha wird das \emph{mu sho toku} aus Zeile
[089] mit dem folgenden \emph{ko. bo dai satta} zu \emph{mu sho tokko bo dai
satta} verschmolzen. Das ist etwas ungünstig: Es verschleiert einerseits, dass
das \emph{toku} aus Zeile [089] eine Wiederaufnahme aus Zeile [087] ist, also
wiederum für \emph{Gewinn} steht. Andererseits erschwert es, das folgende
\emph{ko} als eine deduktive Konjunktion wahrzunehmen. Wenn man die
Zusammenhänge kennt, lässt sich aber auch diese Sprechweise aus der japanischen
'u'-Lautung ableiten.}. Die vierte Ausnahme betrifft die Groß- und
Kleinschreibung: ich habe die konsequente Kleinschreibung der Version von Scheid
übernommen. Die Großschreibung nach einem Punkt signalisiert harte syntaktische
Abschlüsse, die semantisch so nicht stimmen.

Meine deutsche Version des \emph{Hannya Shingyōs} folgt in der Regel der
anregenden, wortweisen Übersetzung von
Boeck\footcite[vgl.][\nopage]{Boeck2016a}, allerdings im Abgleich mit den
Erläuterungen von Deshimaru und Scheid. Mein eigenes Zutun wollte von Anfang an
nicht mehr bieten als eine geschickte Anordnung, bei der eine möglichst
wortgetreue Übersetzung zeilenmäßig in der Nähe der zu übersetzenden Phrase
bleibt. Das \emph{Hannya Shingyō} sollte in sinnhaften Einheiten lernbar gemacht
werden. Um das zu erreichen, habe ich die großen syntaktischen Freiheiten der
deutschen Sprache genutzt: im Zweifel habe ich die etwas geschrobenere
Formulierung mit genauer Zuordnung der eleganteren, aber entfernteren
vorgezogen.

Um meine eher syntaktisch motivierten Zutaten als solche zu kennzeichnen, habe
ich sie in eckige Klammern eingeschlossen und kursiv gesetzt. Der deutsche Text
sollte sich mit diesen Zutaten schlüssig von links nach rechts und oben nach
unten lesen lassen. Unmarkierte deutsche Wörter sollten in der Zeile stehen, in
denen auch die chinesischen und japanischen Korrelate stehen - jedoch nicht
immer in derselben Reihenfolge, wie die Originale.

Und noch zwei letzte typographische Aufschlüsselung: 

\begin{enumerate}
  \item Die chinesische Schrift ist eine Begriffsschrift. Trotzdem enthält sie
  auch syntaktische Konnektoren, etwa die Negationen \emph{mu} (= \cnbsmi{無})
  und  \emph{fu} (= \cnbsmi{不}), die additive Konjunktion \emph{yaku} (=
  \cnbsmi{亦} = auch), die einfache Schlussfolgerung \emph{ko} (= \cnbsmi{故} =
  darum) oder die betonte Schlussfolgerung \emph{nai shi} (= \cnbsmi{乃}
  \cnbsmi{至} = darum insbesondere)\footnote{Boeck übersetzt \emph{fu} mit der
  deutschen Vorsilbe \emph{un-} und \emph{mu} mit der expliziten Negation
  \emph{nicht}. \emph{yaku} übersetzt er ebenfalls als auch. \emph{ko} übersetzt
  er wörtlich als \emph{Ursache}. Und \emph{nai shi} übersetzt er als \emph{dann
  extrem}, was ich als \emph{darum inbesondere}
  übernehme.\cite[vgl][\nopage]{Boeck2016a}}. Diese Patikel strukturieren den
  Text logisch. Deshalb habe ich sie in der linearen Anordnung jeweils nach
  links herausgezogen. Im selben Sinne habe ich auch einige andere, gliedernde
  Partikel optisch arrangiert.
  \item Im Text erscheint gelegentlich ein verweisender Pfeil
  \emph{$\rightarrow$}. Zu diesen Zeilen gibt es eine Erläuterung der
  Übersetzung. Die Zeilennummern werden im Kapitel mit den Übersetzungshinweisen
  als Referenz benutzt.
\end{enumerate}

\section{Die Übersetzung} 

Einige Entscheidungen habe ich im folgenden erläutert. Mit ist natürlich klar,
dass eine wirklich wissenschaftliche Aufbereitung viele Aspekte und Behauptungen
nachweisen müsste, auf die ich hier ohne Nachweis zurückgreife. Sie sind das
Ergebnis der Arbeit der anderen Autoren. Ihnen gebührt dafür Respekt,
Anerkennung und Dank, nicht mir. In einer späteren Version werde ich die
Nachweise sicher nachholen. Bis dahin möge man mir nachsehen, dass ich einfach
nur eine besser zu lernende Version erstellen wollte.

\begin{description}

  \item[001-003:] Das \emph{Hannya Shingyō} ist ursprünglich in Sanskrit
  geschrieben, von dort ins Chinesische übertragen und von da aus ist es dann
  noch einmal ins Japanische übersetzt worden. Das Chinesische selbst ist eine
  Begriffsschrift, sodass sich die Übersetzung ins Japanische auf die Definition
  einer 'anderen' Aussprache konzentrieren konnte. Allerdings hatte die
  chinesische Version einige ursprüngliche Formulierung als 'wörtliche Zitate'
  bewahrt. Dabei ist die Aussprache des Sanskrit mit chinesischen Silben
  lautlich nachgebildet worden. Die Übertragung ins Japanische hat diese Idee
  übernommen. Damit entsteht jedoch eine 'Doppeldeutigkeit'. Denn die das
  Sanskrit mehr oder minder gut nachbildenden japanischen Wörter und Silben
  haben natürlich eine eigene unabhängige Bedeutung. Dem entsprechend wird
  gelegentlich gesagt, die Übertragungen hätten die Bedeutung des \emph{Hannya
  Shingyōs} \enquote{vertieft}\footcite[vgl.][56]{Deshimaru1988a}. Das
  \emph{Hannya Shingyō} als Name des Textes ist jedenfalls das erste Zitat aus
  dem Sanskrit.

  \item[005-006:] Der Ausdruck \emph{kan ji zai bo sa} bildet auch ein solches
  lautliches Zitat, allerdings in etwas \emph{verschleierter Form}: Von der
  (chinesischen) Schrift her soll er den Namen vom \emph{Boddhisattva
  Avalokiteshvara} 'wiedergeben'. Die tatsächliche japanische Aussprache -- so
  Deshimaru -- reichere diesen Namen dann 'semantisch' an: Die Silben \emph{bo
  sa} bezögen sich demnach direkt auf den Titel \emph{Boddhisattva} (=
  \emph{\enquote{jemand, der Satori erlangt hat}}, während die Silben \emph{kan}
  (= \emph{beobachten}) und \emph{ji zai} (= \emph{Freiheit}) diesen zu einer
  Beschreibung erweitern: \emph{\enquote{das lebende Wesen, das Satori und die
  wahre Freiheit erlangt hat}}\footcite[vgl.][57]{Deshimaru1988a}. Darum kann
  man den Namen nicht unübersetzt in einen deutschen Text übernehmen: es wird
  hier eben nicht über eine konkrete Einzelperson gesprochen. Vielmehr fungiert
  diese konkrete Person als Typus. Die so verallgemeinerte Aussage erlaubt es
  dem Hörer, sich einbezogen zu fühlen. Um das im Deutschen nachzubilden, nutze
  ich den unbestimmten Artikel und folge ansonsten der Deutung von
  Deshimaru\footcite[vgl.][57 et passim]{Deshimaru1988a}.

  \item[004,011:] \emph{ji} (= \cnbsmi{時}) soll \emph{Zeit} bedeuten und wird
  als Konjunktion zumeist mit \emph{als} oder \emph{während} übersetzt. Im
  deutschen kennen wir zwei Arten der 'zeitlichen' Verbindung zweier Fakten. Die
  eine betont eher die Zufälligkeit, die andere die Ursächlichkeit:
  \emph{\underline{als} ich Zucker aß, bekam ich Kopfschmerzen} meint etwas
  anderes als, \emph{\underline{indem} ich Zucker aß, bekam ich Kopfschmerzen}.
  Im \emph{Hannya Shingyō} ist eine ursächliche Verknüpfung gemeint: \emph{Das
  Praktizieren der Höchsten Wahrheit führt zu der Erkenntnis, dass \ldots}. Das
  Wort \emph{indem} markiert diese ursächliche Beziehung gut.

  \item[014:] Die 5 Skandhas -- nämlich \emph{Erscheinung, Empfindung,
  Wahrnehmung, Wollen bzw. Handeln und Bewusstsein} -- bilden eine zentrale
  Achse des Textes: zuerst wird ihr Oberbegriff \emph{go on} (= \cnbsmi{五}
  \cnbsmi{蘊}) genannt (14). Danach wird von jeder einzelnen gesagt, sie sei
  nicht nur nicht getrennt von \emph{kū}, sondern sie sei \emph{kū} (020-039).
  Schließlich wird auch gesagt, dass es sie in \emph{kū} ansich nicht gäbe
  (050-054), genauso wenig, wie entsprechenden Organe (055-060) oder deren
  Resulte (061-66). Dem liegt ein Weltbild zugrunde, das sicher nicht mehr
  unseres ist. Deshalb ist es angemessen, den fremden Begriff 'Skandha' als
  \emph{Fremdwort} in die Übersetzung zu übernehmen. Allerdings: die Pointe des
  \emph{Hannya Shingyōs}, dass es das, was dieses fremde Weltbild beschreibt, in
  \emph{kū} nicht gäbe, ließe sich umstandlos auch mit unserem heutigen physisch
  / psychischen Weltbild formulieren. Man muss sich also die 'veraltete'
  Sichtweise nicht zu eigen machen, um das\emph{Hannya Shingyō} zu verstehen und
  seine Aussage zu bejahen. Das \emph{Hannya Shingyō} ist -- so gesehen -- sehr
  modern.
  
  \item[021:] Es ist üblich, \emph{shiki} mit \emph{Form} zu übersetzen. Das
  wird der rhetorischen Form des Textes aber nicht gerecht: \emph{shiki} ist die
  erste der 5 Skandas. Die anderen 4 werden in den Zeilen [035-038] aufglistet.
  Die Übersetzung von \emph{shiki} muss auch das 1. Skandha schon als Teil einer
  Reihe erscheinen lassen. Dazu eignet sich das Wort \emph{Form} nicht.

  \item[021-033:] Außerdem wird diese ganze Sentenz gelegentlich zu der Aussage
  verknappt, \emph{Form sei Leere und Leere sei Form}. Damit geht eine -- auch
  rhetorisch entscheidende -- Pointe des Originals verloren: Zuerst sagt das
  \emph{Hannya Shingyō}, \emph{shiki}, die \emph{Erscheinung} sei nicht getrennt
  von \emph{kū}. Dies muss den Hörer verwirren. Denn das normale Verständnis
  besagt doch wohl eher, dass es sich dabei um verschiedene Dinge handelt. Und
  mit diesem Erwartungshorizont spielt der Text. Denn er setzt danach --
  sozusagen -- 'noch eins drauf': Er verschäft die Situation, in dem er sagt,
  dass die \emph{Erscheinung} und \emph{kū} nicht nur nicht getrennt seien,
  sondern dass das eine -- \emph{gewissermaßen}, also nicht ganz uneingeschränkt
  -- auch das andere \emph{sei}. Rhethorisch gesehen präsentiert das
  \emph{Hannya Shingyō} also zuerst eine 'steile' These, die es im folgenden
  wird erläutern und begründen müssen. Auf jeden Fall -- und das ist der
  rednerische Zweck dieses Vorgehens -- hat es mit dieser Konstruktion die
  Aufmerksamkeit seiner Hörer geweckt. Darum ist es notwendig, diese
  rhethorische Verschärfung auch in der Übersetzung zu erhalten.
  
  \item[022:] Oft wird \emph{i} mit \emph{verschieden} übersetzt. Das wird dem
  Original nicht gerecht. Denn tatsächlich geht es im folgenden Text [044-087],
  in dem \emph{kū} ex negativo definiert wird, um nichts anderes, als die
  Feststellung von Unterschieden. Die Pointe des \emph{Hannya Shingyōs} ist
  aber, dass \emph{kū} trotz aller Verschiedenartigkeit dennoch -- irgendwie --
  mit den 5 Skandhas zusammenfällt, also trotz aller Verschiedenartigkeit nicht
  getrennt ist von \emph{shiki}. Darum habe ich mich für das Übersetzung
  \emph{getrennt} entschieden; es unterstreicht die intellektuelle Brisanz des
  \emph{Hannya Shingyōs}.

  \item[023ff:] Es ist üblich, \emph{kū} mit dem Wort \emph{Leere} zu
  übersetzen. Allerdings bringt das Wort \emph{Leere} eigene Konnotationen mit,
  die dem eigentlich Gemeinten entgegenstehen. Das Problem schillernder Begriffe
  kennt pikanterweise sogar das \emph{Hannya Shingyō} selbst, mehr noch: es
  spielt sogar mit dem Phänomen: Es nimmt nämlich einen dem Gemeinten
  nahestehenden, vermeintlich klaren Begriff \emph{kū} und schärft diesen
  mittels Aussagen darüber, was das Gemeinte alles \emph{nicht} ist. Solch ein
  Verfahren nennt man eine \emph{Ex-Negativo-Definition}. Tatsächlich besteht
  das \emph{Hannya Shingyō} im Kern aus einer Liste von negierenden Abgrenzungen
  [044-087]. Aus diesem Grund ist es besser, nicht das auch durch die
  europäische Philosophie aufgeheizte Wort \emph{Leere} durch vielfache
  Wiederholgungen zum Kern zu machen, sondern das Original -- also \emph{kū} -- 
  zu verwenden und dessen Bedeutung gerade über Negationen klarwerden zu lassen.

  \item[050:] \emph{ze ko} (= \cnbsmi{是} \cnbsmi{故}) steht für \emph{sein
  Ursache}. Während ich später in Zeile [095ff] \emph{ko} konsequent als
  \emph{darum} übersetze, um den repitiven Charakter zu erhalten, wähle ich hier
  - zu Beginn der Deduktion - das stärkere und elegantere \emph{mithin} als
  Übersetzung.
  \item[072:] \emph{nai shi} besagt für sich genommen \emph{dann extrem}. Es
  geht also um eine besonders wichtige Schlussfolgerung. Solch ein sprachliches
  Konstrukt kennen wir auch im Deutschen, nämlich die einleitende Formel:
  \emph{Darum ist/wird/\ldots inbesondere \ldots}.

  \item[071-073:] Die Kombination \emph{gen kai} (= \cnbsmi{眼} \cnbsmi{界}) steht
  wörtlich für \emph{[Auge Welt]}, die Sequenz \emph{i shiki kai} (= \cnbsmi{意}
  \cnbsmi{識} \cnbsmi{界}) hingegen für \emph{Denkvermögen Unterscheiden Welt}.
  Erstere meint also die sichtbare, die erscheinende Welt, letztere die Welt der
  trennenden Vorstellungen und Konzepte. Auch in dieser Gegenüberstellung trifft
  man indirekt die fünf Skandas wieder: Zeile [058] hat schon \emph{gen} (= das
  \emph{Auge}) dem ersten Skandha \emph{shiki} (= \cnbsmi{色} =
  \emph{Erscheinen}) aus Zeile [053] als Organ zugeordnet. Für das fünfte
  Skandha, das Unterscheiden als intellektuelles Tun -- japanisch ebenfalls
  \emph{shiki} genannt -- wird ein anderes Zeichen benutzt als für das erste
  Skandha, nämlich \cnbsmi{識} (Zeile [057]. Und eben dieses zweite \emph{shiki}
  erscheint auch in Zeile [073]. Die rhetorische Konstruktion 'von \emph{gen
  kai} bis \emph{shiki kai} spannt also indirekt erneut den ganzen Bogen über
  alle fünf Skandhas auf.

  \item[074-080:] Die rhetorische Konstruktion \emph{Es gibt in kū nicht XYZ}
  und \emph{Es gibt in kū kein Ende von XYZ} ist besonders aufreizend für
  (europäische) Logiker: Ersteres negiert die Existenz von XYZ; letzteres setzt
  seine Existenz voraus und betont diese durch den impliziten Hinweis auf seine
  Ewigkeit, ausgedrückt durch eine doppelte Verneinung. Damit widersetzt sich
  das \emph{Hannya Shingyō} der formalen Logik, in dem es dem europäischen
  Verständnis sein \emph{tertium datur} entgegenstellt, nicht ohne diese Logik
  allerdings selbst souverän zu benutzen. Dem Zen entsprechend ist das kein
  Widerspruch, sondern geradezu der Sinn allen Tuns: alle gedanklichen
  Konstrukte müssen aufgehoben werden, wenn \emph{kū} selbst im  Akt der
  Erleuchtung erfahrbar werden soll.

  \item[111-113:] Auch die Sentenz \emph{anokutara sanmyaku sanbodai} ist eine
  zitierende Sanskritnachahmung und meint \emph{höchste, vollkommene
  Erleuchtung}\footcite[vgl.][\nopage Anm. 10]{Scheid2016a}. Welches der Worte
  was bedeutet, habe ich den Quellen bisher nicht entnehmen können. Meine
  Zuordnung ist also willkürlich, folgt aber der Tradition.

  \item[115-121:] Hier findet eine rhetorisch geniale Umdeutung statt, die eine
  große Auswirkung auf den Buddhismus hat: Bisher war der Begriff \emph{han nya
  ha ra mi ta} beschreibend. Er stand für die \emph{die höchste Weisheit, die
  über sich hinausführt}. Jetzt wird der Ausdruck zum Namen des Textes selbst:
  indem er mehrfach als herausgehobenes \emph{Mantra} bezeichnet wird,
  verschiebt sich seine Bedeutung: der Terminus \emph{han nya ha ra mi ta} wird
  zum Namen des Textes. Und in dem diesem dann auch noch eine Wirkung
  zugesprochen wird, wird seine Rezitation zu einem Mittel. Kein Wunder also,
  dass alle Buddhisten diesen Text rezitieren: es steckt in ihm selbst.
 
  \item[119:] Die Phrase \emph{mu jō shu} verwendet wieder einmal eine der im
  \emph{Hannya Shingyō} so gern genutzte 'negative Zuschreibungen':
  \emph{mu} (= \cnbsmi{無}) ist die bekannte Verneinigung; und \emph{shu} (=
  \cnbsmi{咒}) steht für das \emph{Mantra}. Also wird \emph{jō} (= \cnbsmi{上})
  ein Attribut sein, das negiert dem Objekt \emph{Mantra} zugsprochen wird:
  Ein chinesisch-deutsches Internetlexikon sagt, das \cnbsmi{上} auch für
  \emph{von unten nach oben, aufwärts} bzw. \emph{vorwärts gehen}
  steht\footcite[vgl.][\nopage]{babla2016a}. Eine gute Übersetzung würde auch an
  dieser Stelle -- auf der Basis dieser Primärbedeutung -- die bevorzugte
  Methode der Eingrenzung ohne direkte Spezifikation bewahren; sie würde diese
  'ZEN gemäße' Art des 'Denkens' auch hier verdeutlichen. Hier fehlt mir noch
  eine gute Idee für die Umsetzung.

  \item[125-126:] \emph{shin jitsu}(= \cnbsmi{真} \cnbsmi{實}) soll
  \emph{Realität} meinen, und \emph{fu ko} (= \cnbsmi{不} \cnbsmi{虛} ) für
  \emph{nicht/keine Unwahrheit} stehen. Ersteres übersetze ich mit
  \emph{wirklich}, letzteres müsste dann \emph{wahr} heißen. Ich belasse
  letzteres aber bei \emph{nicht unwahr}, um die Neigung des \emph{Hannya
  Shingyōs} zur (doppelten) Verneinung zu erhalten.
\end{description}

\section{Die Deutung}

Ohne Frage, das Hannya Shingyo ist ein Rezitationstext, der sich bruchlos in das
didaktische System des Zen-Budhismus einfügt. So gesehen, dient er - wie alles -
der Befreiung, der Erleuchtung. Hinnerk Polenski hat dieses 5 stufige System
erläutert und vorgeführt: am Anfang steht die silbenweise Rezitation aus dem
Hara, gefolgt von der phrasen- bzw. mantrabezogenen Rezitation und der
Rezitation des ganzen Textes im absoluten Samadhi, in der absoluten Versenkung.
Auf der vierten Stufe kommt das Tempo hinzu, es werden möglichst lange Teile auf
einem Ausatmenstrom rezitiert (positives Samadhi) und zuletzt wird dieses
zwecks eines noch spezielleren, ganzheitlichen Körpereinsatzes mit einem
Obertongesangsstil kombiniert\footnote{Hinnerk Polenski hat dies im Rahmen
eines Teishos im März 2017 entwickelt. Der Mitschnitt ist in YouTube noch
nicht freigegeben. Sobald das passiert ist, trage ich den Link hier nach.}.
Dieses didaktische System verleiht dem Hannya Shingyo fünf autonome Dimensionen.

Es gibt allerdings - davon unabhängig und für den Weg der Erkenntnis auch sicher
nicht so essentiell - eine sechste. oder besser gesagt: eine 0. Dimension. Sie
nähert sich dem Text von seiner Bedeutung her  und schlüsselt ihn - im Anklang
zur europäischen Geistestradition - rhetorisch auf:

Eine gute Rede enthält vier respektive fünf Teile: Sie versucht zunächst die
Aufmerksamkeit des Publikums zu wecken (\emph{exordium/prooemium}) und schildert
dann den Sachverhalt (\emph{narratio}). Optional gliedert sie danach die
folgende Argumentation (\emph{propositio}) und argumentiert anschließend -
entsprechend der Gliederung - für die Glaubwürdigkeit der Sache
(\emph{argumentatio}). Zuletzt zieht sie die Schlussfolgerung und zwar durchaus
appelativ (\emph{peroratio/conclusio)}: denn alleinger Zweck einer Rede ist es,
den Hörer für den eigenen Standpunkt zu gewinnen\footnote{Die Rhetorik
entsprechend zu referieren, ist eine eigene aufwendig Arbeit, die der
Überprüfbarkeit wegen zwar zu leisten wäre, deren Erkenntnisgewinn in unserem
Zusammenhang aber eher gering wäre. Deshalb referiere ich an dieser Stelle
exemplarisch eine wirklich allgemein erreichbare Quelle:
\texttt{https://de.wikipedia.org/wiki/Rhetorik\#Redeteile} }.

Man kann also auch sagen, dass eine gute Rede einen \emph{door opener} enthält,
eine Kernthese präsentiert, diese entfaltet und dann das Publikum für diese
These zu gewinnen versucht, am besten, in dem sie gleich auch eine \emph{take
home message} mitgibt. Und genau diese Teile finden wir auch im Hannya Shingyo:


% insert the bibliographical data here
\bibliography{bibfiles/hsResourcesDe}

\end{document}
